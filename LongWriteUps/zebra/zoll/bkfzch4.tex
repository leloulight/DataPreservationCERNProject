%  $Header: ch1.tex,v 1.11 93/08/02 21:16:12 goossens Exp $
\Filename{H1-FZ-technical-details}
\chapter{Technical details}

\Filename{H2-FZ-FZINFO-obtain-status-information}
\section{FZINFO - obtain status information}

\Shubr{FZINFO}{(LUN)}

Gets information about an FZ unit \Rarg{LUN} one may
which will load the communication area

\begin{verbatim}
      COMMON /FZSTAT/ INFLUN, INFSTA, INFOFZ(40)
\end{verbatim}

with

\begin{verbatim}
   INFLUN = LUN  if success,
          = 0    if unit LUN was not found, in which case
                 the rest of the information is useless.

   INFSTA = a copy of the status word and
   INFOFZ = the first 40 data words of the FZ control bank for LUN.
\end{verbatim}

Details about the significance of the returned information are
found below.

%This routine is new with Zebra version 3.62, October 1989.

The lay-out of the control-bank has changed with version 3.66.

\Filename{H2-FZ-control-bank-description}
\section{FZ control bank description}

\begin{verbatim}
link 2  points to the associated text buffer

@:   address is LQFy, shortened to '@'      y = F or I or X
@ -5  LUNy

@ +0  MSTATy   status word of the control bank
      MEDIUy  bit   1-3:  medium:   0   disk,   1 T  tape, normal
                                      2 K disk,   3 TK tape, user
                                      4 C disk,   5 TC tape, channel
                                      6 M memory

      IFIFOy  bit   4-6:  file fmt:  0   native
                                      1 X exchange sequential
                                      2 D direct-access
                                      3 M memory
                                      4 A alfa

      IDAFOy  bit     7:  data fmt:  0 native,  1 exchange

      IACMOy  bit  8-10:  access:   0 F Fortran
                                      1 Y special
                                      2 L c library
                                      3 C channel

              bit    11:  read  permission
              bit    12:  write permission
              bit 13-14:  NEOF = 0, 1, 2, or 3
              bit    15:  option S for separate d/ss
      IUPAKy  bit    16:  option U for unpacked d/ss

@  1  IADOPy  Channel:  adr of user routine
              Memory:   current memory pointer

@  2  IACTVy  last activity:

      =  0  file unused              10  switch input to output
                                         or: rewind to over-write
                                         or: new output file connected
         1  read start-of-run        11  write start-of-run
         2  read d/s                 12  write d/s
         3  read end-of-run          13  write d/s, buffer flushed
         4  read Zebra  end-of-file  14  write end-of-run
         5  read system end-of-file  15  write end-of-file
         6  read end-of-data         16  write end-of-data
         7  attempted read beyond    17  attempted write beyond EoD
         8  rewind after read        18  rewind after write for input
            or: new input file connected

@  3  INCBFy  incr. to go to the buffer param.:  LBPARy = LQFy + INCBFy
@  4  LOGLVy  logging level

@  5  MAXREy  native mode:    maximum logical record size
              exchange file format:  physical record size
@  6
@  8  Memory: LMEM:  User memory starts at IQ(LMEM)
@  9  Memory: NWMEM: Size of the user memory

@+11  number of system file marks
@ 12  number of Zebra EoF signals
@ 13  number of   end-of-run records
@ 14  number of start-of-run records
@ 15  number of pilot records
@ 16  number of non-empty d/s
@ 17  number of     empty d/s
@ 18  number of errors
@ 19  number of Mwords read/written
@ 20  number of  words read/written (up to 1 M)
@ 21  number of good logical records
@ 22  number of good physical records (exchange file format)
@ 23  number of steering blocks (exchange file format)
@ 24  number of conversion problems (exchange data format)

@ 26  input: last read status returned
@ 27  input: LRTYP status of last record read
@ 28  number of pilots printed by last statistic message
@ 29  current run number
@ 30  input:  pending:  0 nothing,  1 EoF,  2 st/end of run

@ 31  DsA word 1:  record # for current d/s
@ 32  DsA word 2:  offset
@ 33  current record number for direct-access
      if -ve: FZINXT called, reset from previous
@ 34  saved DsA word 1 of the direct access table
@ 35  word 2
\end{verbatim}

\subsubsection*{Control bank for input}

\begin{verbatim}
@ 41  NWTB, number of table words
@ 42  NWBK, number of words of bank material
@ 43  LENTRY, entry adr to d/s

@ 50  NWIOI, number of words in I/O characteristic
@ 51 -> 66   I/O characteristic for last user header vector

@ 68  NWSEG, number of words in the segment table
@ 69 ->128   last segment table read

@130  NTBE, number of early table words (only if native file format)
@131 -> 170  early table

  buffer parameters in the control bank,  LBPARI = LQFI + INCBFI

      -9  non-zero if last LR abended
          next physical record ready in the buffer if N4ENDI not zero
      -7  expected next PhR number
      -6  zero if current block is steering,
            = 1 if last fast block in burst, =2 if last-but-one, etc
      -5  number of fast blocks to follow current block
      -4  N4SKII, # of words to be skipped before next transmission
      -3  N4RESI, # of words still to be read for current LR
      -2  N4DONI, # of words of the current buffer already done
      -1  N4ENDI, # of words in buffer before start of next LR
                  if = buffer-size: record continues in next PhR
LBPARI+0  maximum size of buffer, file words
      +1  expected size of PhR, machine words
      +2  displacement from LBPAR to start-of-buffer LSTO
      +3  off-set from start-of-buffer to read position for packed records
      +4  off-set for writing

          LSTO = LBPARI + IQ(LBPARI+2)

 LSTO -1  space for left half of double-precision number spanning PhRs
 LSTO +0  start-of-buffer: read (if 32-bit machine) or unpack into here
\end{verbatim}

\subsubsection*{Control bank for output}

\begin{verbatim}
@ 37  limit: Mega-words
@ 38  limit: words
@ 39  original parameter from FZLIMI

  buffer parameters in the control bank,  LBPARX = LQFX + INCBFX

      -6  zero if current block is steering,
            = 1 if last fast block in burst, =2 if last-but-one, etc
      -5  number of fast blocks to follow current block
      -4
      -3  N4RESX, # of words still to be done for current LR
      -2  N4DONX, # of words of the current buffer already done
      -1
LBPARX+0  maximum size of buffer, file words
      +1  size of PhR, machine words
      +2  displacement from LBPAR to start-of-buffer LSTO
      +3  off-set for input
      +4  off-set for output

          LSTO = LBPARX + IQ(LBPARX+2)

      -1  free to allow packing with shift
 LSTO +0  start-of-buffer for accumulation

      last buffer-word + 1:
          space for right half of a double-precision number
                spanning physical records
\end{verbatim}

Filename{H2-FZ-ALFA-exchange-format}
\section{ALFA exchange format}

To allow files in exchange mode to travel on networks which
cannot handle binary file transfer, a conversion to and from
card-images is provided.
This uses a sub-set of the 64-character ASCII set
to represent the binary contents of a file
in exchange file and data format.

The easiest way, from the coding point-of-view,
would be to generate a hexadecimal dump of the binary file.
But this is too simple in several respects:

\begin{itemize}
\item markers must be provided to mark the start and the end of the
      original physical record, to be used for recovery of read errors
      and for fast skipping of unwanted records.
\item the hexadecimal dump of one word would give 8 characters,
      taken from a set of 16 characters.
      In this case 4 bits are represented by one character.
      This is wasteful,
      and we use a set of 32 characters,
      each representing 5 bits, and hence we need 7 characters
      for each word (at most).
      If one were to represent 6 bits by one character,
      one could represent one word on 6 characters,
      but one would need a 64 character set;
      this we consider to be too tight.
\item compression of sets of consecutive identical words, and
\item special dense representations of short integers and exact
      floating point numbers should be provided, to make the ALFA file
      as short as possible,
       in order to save network transmission time.
\end{itemize}

\subsubsection*{File format}

The file produced or expected by \Rind{FZOUT}/\Rind{FZIN} 
in 'ALFA mode' consists
of card-images of 80 columns exactly, written by FORTRAN formatted
WRITE statements.
Column 1 of all lines is blank,
except for the first and the last line representing
the start and the end of an original physical record.
Blocking, if any, is under control of the user with the JCL.
On an IBM the character set is EBCDIC,
on a CDC Display code, on most other machines it is ASCII.
Translation, if any, is expected to happen in the network
stations.
Such files are not written to tape.

\subsubsection*{Number representation}

An original 32-bit binary word is unpacked onto seven 5-bit bytes:

\begin{verbatim}
      I(1) = bits 31,32
      I(2) = bits 26 -> 30
          ...
      I(6) = bits  6 -> 10
      I(7) = bits  1 ->  5
\end{verbatim}

In general this word is represented by 7 characters:

The first character \Lit{C(1)}, called the type-character,
combines the information \Lit{I(1)} with more information about possible
shortening of the representation.

The remaining 6 bytes \Lit{I(j)} are translated into the characters \Lit{C(j)}
whose \Lit{CETA} values are \Lit{I(j)+1}:

\begin{verbatim}
      value  0 1 2 ... 17 18 19 20 21 22 23 24 25 26 27 28 29 30 31
      char.  A B C      R  S  T  U  V  W  X  Y  Z  0  1  2  3  4  5
\end{verbatim}


Small integers contain several leading bytes of value zero
(or 31 for negative integers),
the number \Lit{N} of such bytes is encoded into \Lit{C(1)},
and the characters \Lit{C(2 to N+1)} are not output.
A similar scheme is applied to words with trailing bytes
\Lit{I(j to 7)} of value zero.

Very small positive integers, with value less than 10,
are given a one character representation,
and the encoding for \Lit{C(1)} is arranged such that the integers
0 to 9 stand for themselves.

The construction of \Lit{C(1)} is shown by this figure:

\begin{verbatim}
       I(1)  ooooo 11111 22222 33333 ooooo oooooooooo 33333
       I(2)  xxxxx xxxxx xxxxx xxxxx ooooo oooooooooo -----
       I(3)  xxxxx xxxxx xxxxx xxxxx xoooo oooooooooo x----
       I(4)  xxxxo xxxxo xxxxo xxxxo xxooo oooooooooo xx---
       I(5)  xxxoo xxxoo xxxoo xxxoo xxxoo oooooooooo xxx--
       I(6)  xxooo xxooo xxooo xxooo xxxxo oooooooooo xxxx-
       I(7)  xoooo xoooo xoooo xoooo xxxxx o123456789 xxxxx

  main type  1     2     3     4     5     0          6
   sub-type  01234 01234 01234 01234 01234            01234

  type-code              11111 11111 22222 2222333333 33334
             01234 56789 01234 56789 01234 6789012345 67890

 type-char.  ABCDE FGHIJ KLMNO PQRST UVWXY 0123456789 +-*/(
\end{verbatim}

\Lit{'x'} stands for any value not zero,
except in Main type 6 where \Lit{'-'} stands for a value of 31,
and where \Lit{'x'} stands for any value not 31.

One can see that main types 1 to 4 cover the general case and also
shortening for trailing zero bytes.
Main types 5 and 6 cover positive and negative short integers,
very small positive integers have main type zero.

\subsection*{Compression of sets of consecutive identical numbers}

This is done by following the first number of the set by the
special character \Lit{'='}, not a member of the ordinary
type-characters \Lit{A} to \Lit{Z},  \Lit{0} to \Lit{9},
\Lit{+} to \Lit{(}, followed by the short integer \Lit{N} 
(represented like all other integers).
This signals to the reading program that the last number has to
be repeated \Lit{N+1} times.

\subsubsection*{Examples:}

\begin{verbatim}
    XBA=YU   XBA and YU are the small integers 32 and 20,
             hence this stands for 22 words containing 32.

    /45=9    /45 and 9 are the small integers -33 and 9,
             hence this stands for 11 words containing -33.

    0=XMO    0 and XMO are the small integers 0 and 398,
             hence this stands for 400 words of all zeros.
\end{verbatim}

This surely is unreadable, but it is not meant for the human eye.

\subsection*{Non-repetition of identical type characters}

If a set of consecutive numbers all have the same type character,
the first number is preceded by the special character \Lit{'['},
the type is omitted for all numbers except the first,
and the last number is normally followed by the special character \Lit{']'}
(the characters \Lit{'['} and \Lit{'='} also terminate a set of
same type-characters).

\subsubsection*{Example:}

\begin{verbatim}
   [XBA MO MP MQ  [YU  V  W  A  B  C  X  Y  Z  0  1  2  3  4  5]
\end{verbatim}

represent the numbers

\begin{verbatim}
    32 398 399 400 20 21 22  0  1  2 23 24 25 26 27 28 29 30 31
\end{verbatim}

(the blanks are typed for readability, they are not present on the file).
\end{verbatim}

\subsection*{Start and End of physical record}

The first line of the dump of each physical record carries the
special character \Lit{'>'} in column 2, normally also in col. 1.
Column 3 of the first line has \Lit{'0'} or \Lit{'1'} if the record is a
fast or a control record.
The last line carries the symbol \Lit{'<'} in column 1.
The last number of the record, normally on the last line,
is followed by \Lit{'<'}, followed by the two check-sum numbers.
The check-sums are obtained by addition,
separately for bits 17 to 32 and bits 1 to 16,
of the binary value of each number written.
A second \Lit{'<'} could be given instead of the 2 check-sum numbers
to suppress the check.

\Filename{H2-FZ-Coding-Zebra-user-IO}
\section{Coding Zebra user I/O}

On UNIX machines, and if Zebra has been installed with the
\Rind{FZLIBC} option, one can select user I/O for a particular
file by giving the \Ropt{K} option to \Rind{FZFILE}.
As with the \Ropt{L} option this causes the I/O requests to be
channelled to the routines \Rind{CFPUT}, 
\Rind{CFGET}, etc.

To code user I/O one has to take the source of these
routines and add a new branch, or two, into each one
to do what one wants to do.

One finds the source of these routines on the Pam file
KERNFOR in \Lit{P=CCGENCF}; there is also the Fortran dummy
routine \Lit{P=TCGEN, D=CFWEOF} to write system file-marks.
Careful, it is possible that there is a machine-specific
implementation of these routines; for example if one
wanted to do this on the Sun one should check that
KERNSUN does not contain versions of these routines
(it does not right now).

The second parameter for each routine is \Lit{MEDIUM},
and on entry its value selects the wanted branch.
\Lit{MEDIUM} is 0,1,2,3 for \Lit{L,TL,K,TK} selected 
with \Rind{FZFILE} for the that file.
Be careful not to destroy the existing path.

\Filename{H2-FZ-Byte-inversion-on-VAX}
\section{Byte inversion on the VAX}

The problem arises because the VAX, and also DECstation,
loads computational registers with bytes of increasing address
starting at the right moving to the left, whilst tape, disk, and
memory are mapped according to increasing byte address.
For example, supposing we had an exchange format record
starting with the following numbers:

\begin{verbatim}
  1 - bit pattern:          hex  0123CDEF
  2 - integer:        292   hex  00000124
  3 - floating:       1.0   hex  3F800000
  4 - Hollerith:     ABCD   hex  41424344
  ...
\end{verbatim}

the record would start with the following bytes:

\begin{verbatim}
   number:    1  2  3  4   5  6  7  8   9 10 11 12  13 14 15 16  ...
   value :   01 23 CD EF  00 00 01 24  3F 80 00 00  41 42 43 44  ...
\end{verbatim}

Reading the record to memory will transmit the bytes in this
order to increasing byte addresses, but under the VAX optic
showing association to registers we have to write it as:

\begin{verbatim}
   number:   ...  16 15 14 13  12 11 10  9   8  7  6  5   4  3  2  1
   value :   ...  44 43 42 41  00 00 80 3F  24 01 00 00  EF CD 23 01
\end{verbatim}

This is right as it is only for Hollerith
(for which this scheme has been designed, of course),
but it is upside-down for integers and bit patterns;
floating-point has to be converted anyway.

Since all the control-information is integer,
and Hollerith is relatively rare,
\Rind{FZIN} on the VAX for simplicity (and speed)
converts the whole record with one call (to VXINVB),
inverting the 4 bytes of each word, giving:

\begin{verbatim}
   number:   ...  16 15 14 13  12 11 10  9   8  7  6  5   4  3  2  1
   value :   ...  41 42 43 44  3F 80 00 00  00 00 01 24  01 23 CD EF
\end{verbatim}

Conversion from Exchange Data Format to VAX internal
transforms only floating-point and Hollerith,
the integers being ready, giving:

\begin{verbatim}
   number:   ...  16 15 14 13  12 11 10  9   8  7  6  5   4  3  2  1
   value :   ...  44 43 42 41  00 00 40 80  00 00 01 24  01 23 CD EF
\end{verbatim}

\Rind{FZOUT} on the VAX goes through the inverse process.

Byte inversion in \Rind{FZIN}/\Rind{FZOUT} operates only for exchange data format;
bytes are not swopped for native data format, not even with the
exchange file format.

\endinput

