% file : bkjz91.tex   26 July  9.30
\thispagestyle{empty}     % empty head
\markboth{}{26 July 91}   % empty foot
\begin{center} {\large\bf Zebra Reference Manual} \vspace{8mm} \end{center}
\begin{center} {\LARGE\bf book JZ91} \vspace{2mm} \end{center}
\begin{center} {\Large\bf Processor support} \vspace{4mm} \end{center}
\begin{center}
  {\large\bf Zebra version 3.67 \vspace{2mm} \\
             July 1991          \vspace{2mm} \\
             J. Zoll}
\end{center}
\vspace*{20pt}
\begin{description}
  \in{30mm}
  \item[Chapter 1] Basic calling sequences
  \begin{itemize}
     \in{30mm}
     \item[1.1] JZIN/JZOUT - simplest use
     \item[1.2] JZIN   - processor entry, general use
     \item[1.3] JZINIT - initialize the JZ91 package
     \item[1.4] JZTELL - count processor conditions
     \item[1.5] JZZERO - zero the down call bank
     \item[1.6] JZROOT - reset to processor level zero
     \item[1.7] JZEND  - print processor usage statistics
     \item[1.8] Titles JZAN - processor constants
     \item[1.9] Titles JZFL - processor flags
  \end{itemize}
  \item[Chapter 2] Extra features
  \begin{itemize}
     \in{30mm}
     \item[2.1] JZIN   - extra features
     \item[2.2] JZINIT - extra features
     \item[2.3] JZSETF - set processor flag by program
     \item[2.4] JZLOG - processor logging
     \item[2.5] JZWIND - unwind the processor stack
     \item[2.6] JZTRAC - print processor trace-back
     \item[2.7] Receiving the working space
     \item[2.8] Note on processor timing
     \item[2.9] Off-line initialization of a processor
  \end{itemize}
  \item[Appendix] JZ91 data structure and bank descriptions
\end{description}
\newpage
\vspace*{40pt}
{\large\bf Acknowledgement}

This package is derived from the HYDRA package JQ81

A.Norton, J.Zoll, HYDRA Topical Manual, book JQ81, CERN Program Library
\cleardoublepage  % continue on next odd page
\markboth{Principles}{Principles}
\vspace*{20pt}
\lile{8mm}
\begin{center}
  {\large\bf{Principles}}\\
\end{center}
% \smark{Principles}
\lspa

\vspace*{2pt}
{\large\bf Purpose}

The MZ package of ZEBRA helps the user to organize his data.
The purpose of the present JZ package is to assist him
in structuring his program.
It allows to formalize the concept of 'program module'
beyond the mere subroutine
and it provides the back-up service for these modules.

It is at the design stage of a program,
rather than later,
that the advantage of the JZ package will be most strongly felt,
since it provides a frame-work for the design;
again just like with the data structures of ZEBRA.

The program we are talking about will be designed as
a collection of modules called 'processors'.
The art consists in designing processors with interfaces
as simple and logical as possible,
and entirely documentable.

A given processor has a given task
which formalizes into a transformation of the input data structure
or rather sub-structure.
The result may be a modification of the input structure,
or a new output structure,
or just printed output and the like.

The processor is controlled by what is essentially a parameter list.
Normally this list contains pointers to the sub-structure
the processor is to work with.
Since links have to be held on relocatable memory
the parameter list is passed in a special purpose bank,
the 'call bank',
containing reference links and data words.
This call bank is filled with the input parameters
by the higher level code which calls the processor.
The processor takes them from there and also places back
into the same call bank any output parameters,
in particular links to the output data structure,
if any has been lifted.
Clearly the content of the call bank is part of
the specification of the processor.

A processor may call other processors.
This is not to say that a good design should aim
at having processors at several levels.
On the contrary, the fewer levels one can do with
the better, of course.
Also, one should not write trivial processors
where simple subroutines will do.

By convention, every processor is entitled to have
the ZEBRA working space near the beginning of Q
freely to itself.
As a result a processor calling another processor
looses at that moment the content of its working space.
Its dimensions are saved by JZ,
and they are automatically restored when control
comes back to the calling processor,
ie. it does not have to call MZWORK again.
As an extra facility, JZ91 may be asked to save and restore
also the contents of the first so many links
and of the first so many data words of the working space.
\newpage
{\large\bf JZ91 Services}

JZ91 provides the following services
to an application software organized into processors:

--- handling of 'call banks' serving to transmit parametric
information of the link and data types
between processors at levels n-1, n, and n+1.
For the processor at call-depth n the 'up' call-bank,
pointed to by the system link LQUP,
assures the communication with the higher level at
call-depth n-1;
and the 'down' call-bank,
pointed to by LQDW,
communicates to the lower level at call-depth n+1.
Call banks of equal size are pre-lifted,
one for each level of a definite number of levels,
they stay permanently in memory.

--- handling of 'processor constants',
being part of the environmental data for each processor,
fixed during the run.
If a processor needs any constants at all,
it may initialize them itself,
this then being the default initialization.
By using titles, loaded with TZINIT described in book TZ,
this default can be over-ruled.
The system link LQAN gives Access to these Numbers thus :
\bva
              IQ(LQAN)   number of constants
               Q(LQAN+1) first constant
               . . .
\end{verbatim}

--- handling of 'processor conditions' which may be signalled from
any processor with  CALL JZTELL (J),
J being a small integer normally from 1 to 10.
This provides for simple counters over the whole run
grouped by processors.

--- handling of statistics of processor usage,
like number of times entered and time spent.
The number of times entered is accessible to
the processor in  IQ(LQSV+2).

--- saving the size of the working space,
on special request also the contents,
on down-call to the next processor
and restoring it on up-return.

--- handling of 'processor flags' for test runs during
program development.
The flags may be used to drive debug operations of
a processor without having to recompile it.
The flags for a given processor are defined by the user
on titles JZFL and they
are copied on entry to the processor into the vector JQFLAG,
ready for inspection by the code in the processor;
non-initialized flags are set to zero.
This is only available with the program-development
version of JZ91;
the production version presets all flags to zero
and leaves them thus for the whole run.

This 'environment' information is held in
the bank of 'support variables',
one bank for each processor,
which is permanently in store as part of the JZ91 data structure.

Communication between the processors and JZ91 is via :
\bva
            COMMON /JZUC/LQJZ,LQUP,LQDW,LQSV,LQAN, JQLEV,JQFLAG(10)
\end{verbatim}
JZ91 operates in and for one store only,
which must be the user's main store,
normally the primary store.
Links in the call bank can point only into this store.

\lile{-8mm}
\chapter{Basic calling sequences}

\section{JZIN/JZOUT - simplest use}
\smark{JZIN/JZOUT - basic}

Processor AA transfers control to processor BB with
a simple Fortran
\hspace{.2mm}
CALL BB,
having readied the contents of its down call-bank
at LQDW:
\bvb
          . . .
          LQ(LQDW-1) =     load parameters of the link type
          LQ(LQDW-2) =
          . . .
          IQ(LQDW+1) =     load parameters of the data type
          IQ(LQDW+2) =
          . . .
          CALL BB          transfer control
          . . .
\end{verbatim}
\lspa
In the simplest case the processor BB does
not call itself another processor,
does not have processor constants,
and does not use processor flags.
It would then look like this :
\bvb
          SUBROUTINE BB

    +CDE, Q.       this is supposed to declare the store and also /JZUC/
         +,   links, data, last

          CALL JZIN ('BB  ',0,0,0)
          CALL MZWORK (0,data(1),last,0)

              processor body

          CALL JZOUT ('BB  ')
          RETURN
          END
\end{verbatim} \lspa
By calling JZIN the processor causes switching of the environment,
gaining access to its own data,
in particular to its call-bank via the system link LQUP,
thus LQ(LQUP-1) is its first link parameter.

The inverse switching is done by JZOUT.
The processor name has to be given to JZOUT explicitely.
This handshake is a check against forgotten calls.

The call to MZWORK must come after the call to JZIN
because JZIN saves the working space parameters of AA,
and hence they must still be intact.

For efficiency, JZIN and JZOUT, and also other routines,
expect to receive the processor name IAM with 4 characters exactly,
with blank-fill if necessary.
\newpage
\section{JZIN - processor entry, general use}
\smark{JZIN - general}

Processor AA transfers control to processor BB with
a simple Fortran
\hspace{.2mm}
CALL BB,
having readied the contents of its down call-bank
at LQDW.
To trigger swopping of the processor environment,
the first executable statement in the processor BB should be

\Subr{CALL JZIN (IAM, IFDOWN, NAN, 0)}
\bvb
with    IAM  processor ID, a text string of 4 characters exactly
                           of type CHARACTER*4

     IFDOWN  flag indicating whether this processor does or does not
             call other processors :  = 0 no  / = 1 yes

        NAN  number of processor constants used

          0  zero; non-zero gives access to the extra features
                   described in para. 2.1
\end{verbatim}
\lspa
JZIN saves the environment of the upper processor
and then sets up the environment of the
new processor.
If this does not yet exist,
it calls the internal service routine JZLIFT to create
the bank of support variables,
digesting the titles for this processor, if any.
JZIN returns the initialisation status thus :
\bvb
   IQUEST(1) =  -ve : just initialized without JZAN title
                  0 : just initialized with    JZAN title
                +ve : normal running
\end{verbatim}
\lspa
Thus a processor using processor constants can check
this condition like in this {\bf example} :

\bvc
      SUBROUTINE BB

+CDE, Q.                     declaring the store and /JZUC/

      CHARACTER    IAM*4
      PARAMETER   (IAM = 'BB  ')

      CALL JZIN (IAM,0,3,0)
      IF (IQUEST(1))         11, 17, 21

C--       Initialize constants if and only if not set from title
   11 Q(LQAN+1) = .0025
      Q(LQAN+2) = .3

C--       Complete initialization calculating derived constants
   17 Q(LQAN+3) = .5 * SIN (Q(LQAN+2))

   21 CALL MZWORK (...)

          ...    processor body

      CALL JZOUT (IAM)
      RETURN
      END
\end{verbatim}
\lspa
The 3rd and the 4th parameter to JZIN are looked at only on first
contact for each processor.
\newpage
Note that over-ruling with the JZAN title only works
if the processor is programmed to handle it.
This is done in this example where statement 11
is reached only if there is no title.

JZIN readies for the new processor these links :
\bvb
      COMMON /JZUC/LQJZ,LQUP,LQDW,LQSV,LQAN, JQLEV,JQFLAG(10)

          LQJZ  the header bank supporting the JZ91 data-structure

          LQUP  the upper call bank

          LQDW  the  down call bank, if needed, else = zero.

          LQSV  the bank of support variables;
                IQ(LQSV+1) contains system information,
                IQ(LQSV+2) is 1 for first entry, 2 for second, etc.

          LQAN  the processor constants inside the SV bank :
                IQ(LQAN)   number of constants
                 Q(LQAN+1)  first constant
                 Q(LQAN+2) second constant
                        . . .
      and also :

          JQLEV       is the current call depth level, = zero for the root

          JQFLAG(10)  receives a copy of the flags for this processor.
\end{verbatim}
\lspa
Note that the data in this common /JZUC/ must not be changed
by the user, JQFLAG excepted.

JZIN goes to ZFATAL if IFDOWN is non-zero and
the lowest possible call-depth has been reached.

There are always 10 flag words.
Words not explicitely initialized with a JZFL title
for a given processor are always zero.
This feature is only available with the program development version,
in the production version of ZEBRA the flags are
all initialized to zero and never change,
JZFL titles are dropped by JZINIT and are otherwise ignored.

If you need the working space also in the initialization
part of the processor, look out :
you cannot place the CALL MZWORK before the
CALL JZIN,
nor immediately after, because MZWORK destroys IQUEST.

\newpage
\section{JZINIT - initialize the JZ91 package}
\smark{JZINIT}

The highest processor level,
at call depth zero, is called the 'root'.
The MAIN program is necessarily at this level.
The root level is handled as a processor,
with the ID given in IAMR to JZINIT.
This is used to associate the titles JZAN and JZFL, if any.
The root gets 10 processor constants and 10 JZTELL counters,
unless the extra features of para.~2.2 are used.

Before using the JZ91 package one has to initialize it with

\Subr{CALL JZINIT (IXSTOR, CHIAMR, CHOPT, MAXLEV, NLCALL, NDCALL, 0)}
\bvb
     IXSTOR  the index of the processing store,
             (or the index of any division in this store)
             may be zero if the primary store is used

     CHIAMR  the processor ID of the root,
             type CHARACTER*4, string of 4 characters

      CHOPT  character string of options :

                T  timing selected
                Q  quiet, no log output
                E  error messages only

     MAXLEV  maximum call-depth number,
             eg. =1 if only the root calls processors

     NLCALL  maximum number of links in all call banks

     NDCALL  maximum number of data words in all call banks

          0  zero; non-zero gives access to the extra features
                   described in para. 2.2
\end{verbatim}
\lspa

JZINIT will create the long-range division JZ91 in the store
signalled by IXSTOR for holding the JZ91 data structure,
which contains all JZ data, like the call banks, the SV banks, etc.
This store must be the store where the user does his processing;
the links LQJZ,LQUP,... will be declared by JZINIT to be a link-area
for this store.
Links in call banks can only point into this store.

Titles JZAN and JZFL, if any, must have been read into the
title-structure of this same store before JZINIT is called,
because it will re-format or re-link them for use.

All call banks are pre-lifted by JZINIT,
all of the same maximum size as specified by NLCALL and NDCALL,
one call bank for each of the MAXLEV levels.
They are permanent banks,
being continously re-used.

Accounting the execution time of the processors individually
is an option which could be expensive in real time
on some computers.

JZINIT returns IQUEST(1) just like JZIN.

\newpage
\section{JZTELL - count processor conditions}
\smark{JZTELL}

To signal condition J in the current processor one may
\Subr{CALL JZTELL (J)}

which bumps the counter J=1,2,...,NCD contained
in the support variables.
The first or the last counter are bumped for
underflow or overflow in J.

NCD, the number of counters available in the SV bank,
is normally 10.
If more are needed the extra features of JZIN have
to be used,
as explained in para.~2.1.

\section{JZZERO - zero the down call bank}
\smark{JZZERO}

When filling the down call bank for the next processor to be
called it is safer to clear the unused part of this bank to zero,
with

\Subr{CALL JZZERO (NL, ND)}
\bvb
with     NL  leave the links 1 to NL untouched and reset links
             NL+1 to the end to zero;
         ND  same for the data words, words 1 to ND untouched.
\end{verbatim}
\lspa
Note that JZIN does already clear the new down call bank to zero.

\section{JZROOT - reset processor level to root}
\smark{JZROOT}

If recovery to 'next event' is done with transfer to QNEXT
(see book MZ, para.~3.04),
QNEXT should reset the processor level to 'root' with :

\Subr{CALL JZROOT}

No harm is done by calling JZROOT on first entry to QNEXT,
when the level is already zero.

\section{JZEND - print processor usage statistics}
\smark{JZEND}

To get this printed on the log file, one calls from ZEND :

\Subr{CALL JZEND}

The  apparent number of calls to the root reflects the
number of times that JZROOT did actually have to unwind
the processor stack,
except for the initial entry with JZINIT.
\newpage
\section{Titles JZAN - processor constants}
\smark{Titles JZAN constants}

See TZINIT in book TZ for input of titles into
the dynamic store.

For each processor whose constants are to be initialized
via the titles,
thus over-ruling the default in the processor itself,
one title should be given :

\bvb
   word 1  processor ID in A4 format

        2  constant 1
        3  constant 2
      ...  ...
      n+1  constant n
\end{verbatim}
\lspa

The number constants given should agree with
the number given as NAN to JZIN.
A discrepancy causes a diagnostic message.
If some constants are derived by the processor from other
constants,
as in the example of para.~1.2,
their places have to be kept open by giving dummy zeros.

If several titles are given for the same processor,
the first title coming in the title input stream is taken,
later ones are dropped.

\bvc
Example :

      *DO  JZAN  -E11 -C21/72    #. Constants for central detector decoding
       MAIN               :CDRC
       GLOBAL T0             0.
       DE/DX SCALE           4.
       MAX BASE              15.
       S-SAMPL LENGTH        8.
       A-COEFF               0.89
       MIN P-LENGTH          2.
       T-SLEW CONSTANT       1000.
       AV INV RAW E1         0.00585
       DEDXOFFSET             68.0
       T-COR FOR S/W         -35.0

      *DO  JZAN  -C11/72
                :V0FI
       DPPMAX        30.0    #.  maximum DELTA P / P
       Y2TMAX         6.0
       ETAMAX        99.0
       AMBFLG         1.0    #.  do not remove ambiguities
       DISVMN         0.0
       DISZMX         0.3
       STDIMP         1.0
       PENALT         0.1

      *DO  JZAN  -E11 -C21/72
                          :SERC
       DUMMY (RUN NO.)         0.
       TBIN TRAK+BGR           2.
       CDIN TRAK+DIGIT+BGR     3.
       FDIN NOT CALLED         0.
       DUMMY                   6* 0.
\end{verbatim}
\lspa
\newpage
\section{Titles JZFL - processor flags}
\smark{Titles JZFL flags}

Zero, one or several titles may be given,
each containing flags for one or several processors,
given as one data group for one processor which looks thus :

\bvb
    word  a  processor ID in A4 format
        a+1  flag word 1
        a+2  flag word 2
          . . .                 (n=0 is possible,
                                     it blocks later settings)
        a+n  flag word n
      a+n+1  END     - the termination literal,
                       may be omitted for the last group.


Any JZFL title looks then like this :

    word  1  first word of first group of n1 words
                           ie. n1-2 flags
       n1+1  first word of second group
          . . .
\end{verbatim}
\lspa

If several flag settings are given for the same processor,
the first one is taken and further are dropped.
\bvb
Examples :

*DO  JZFL
    :IMRE   0   0   0   0   0   1   0   0   :END
    :VEFI  99   5   0   0   0   1   1   0   :END
    :V0FI   0   0   0   0   0   0   0   0   :END
    :TMER   0   0   0   0   0   0   0       :END
    :VEMO   0   0   0   0   0   1           :END
    :XCAL   0             :END
    :TFIT   #B1110110001  :END

      For single-group titles it is more economic to omit
      the end terminator (no re-formatting needed) :

*DO  JZFL
    :VEMO   0   0   0   0   0   1
*DO  JZFL
    :XCAL   0
*DO  JZFL
    :TFIT   #B1110110001

\end{verbatim}
\lspa
\chapter{Extra features}

\section{JZIN - extra features}
\smark{JZIN - extra features}

These are requested by giving a LIST as the fourth parameter
to JZIN rather than zero.
The first word of LIST must indicate the length of the list.
Each further word selects the features described:

\bvb
  LIST(2)  = NCD the number of JZTELL counters to be provided for the processor
                 (the default is 10)

  LIST(3)  = NLS the number of working space links and
  LIST(4)  = NDS the number of working space data to be saved
                 (the defaults are 0)
\end{verbatim}
\lspa
When processor AA calls BB it looses the working space
since BB has the right to use it freely,
only the size of the working space is saved by JZIN and
restored by JZOUT.
With these 2 options JZIN is requested to save links 1,...,NLS
and/or data words 1,...,NDS into the bank of support variables
on down-call.
JZOUT will restore them on up-return.

Saving working space data is intended to be used with
{\bf small} amounts of data only,
otherwise this costs time and also memory.
\bvb
Examples :

      DIMENSION LIST(2)
      DATA LIST /1,24/    selects NCD=24; NLS and NDS remain zero

      DIMENSION LIST(3)
      DATA LIST /2,4,3/   selects NCD=4 and NLS=3; NDS remains zero
\end{verbatim}
\lspa
\section{JZINIT - extra features}
\smark{JZINIT - extra features}

This is handled in analogy with the extra features of JZIN :
\bvb
  LIST(2)  = NANR the number of processor constants for the root (default is 10)
  LIST(3)  = NCD the number of JZTELL counters for the root (default is 10)
  LIST(4)  = NLSR the number of working space links and
  LIST(5)  = NDSR the number of working space data to be saved (defaults are 0)
  LIST(6)  = NACCE extra system accounting words in all SV banks,
             this is for monitoring to be used only by experts (default is 0)
\end{verbatim}
\lspa
\newpage
\section{JZSETF - set processor flag by program}
\smark{JZSETF - set flag}

To change the value of flag JFL in processor CHID one can call:

\Subr{CALL JZSETF (CHID, JFL, VALUE)}

This routine acts only if the flag JFL actually exists
in the processor CHID,
ie. if a title JZFL of at least JFL flags has been given.
In case of do-nothing it returns IQUEST(1)=0.

In case of succesful operation it returns 3 values in
IQUEST(1/3) in this order:
\bvb
  1  LFL  adr of the flag JFL is IQ(LFL+JFL)
  2  NFL  length of the flag vector
  3  OLD  previous content of the changed flag
\end{verbatim}
\lspa

\section{JZLOG - processor logging}
\smark{JZLOG - logging}

This gives control over the amount of information printed
about the operation of the processors~:

\Subr{CALL JZLOG (CHOPT)}

CHOPT is a CHARACTER string whose individual letters select
particular outputs :
\bvb
      Q :  suppress all messages
      E :  print error messages only
      N :  reset to normal logging

      T :  monitor each call to JZTELL

      A :  monitor each call to JZIN
      B :          and dump the call bank
      C :          and dump the parameters

      X :  monitor each call to JZOUT
      Y :          and dump the call bank
\end{verbatim}
Options B and C imply A, option Y implies X.

The implementation of the effect of options B, C, Y, is waiting
for other new code in Zebra.
\bvb
Examples :

          CALL JZLOG ('E')

          CALL JZLOG ('TBCY')      maximum logging

          CALL JZLOG ('A')         log only entries

          CALL JZLOG ('N')         back to normal
\end{verbatim}
\lspa
\newpage
\section{JZWIND - unwind the processor stack}
\smark{JZWIND}

If one uses setjmp/longjmp to abnormally quit from some low level
processor to some higher level processor (other than the root)
the processor receiving the longjmp has to unwind the JZ91 stack
to itself by calling JZWIND giving its name IAM :

\Subr{CALL JZWIND (IAM)}

\section{JZTRAC - print processor trace-back}
\smark{JZTRAC}

This routine is called from ZPOSTM during error termination.

\Subr{CALL JZTRAC (MODE)}

It prints the processor names and the call-bank addresses,
it also optionally marks some banks as 'critical'
to get a full dump of these banks in DZSNAP.
Marking 'critical' depends on single bits in MODE :
\bvb
    bit 1  all SV banks in the chain
        2  all call banks in the chain
        3  all banks pointed to from the links
           in the current call banks at LQUP and LQDW.
\end{verbatim}
\lspa

\section{Receiving the working space}
\smark{Notes}

Sometimes the situation arises when the calling processor
wants to receive the working space of the called processor
intact;
it may for instance contain a large error matrix
which one does not want to be copied into a bank
just in case it may be needed.
This request is done in the calling processor
by setting status-bit 15 into the down call bank,
ie. CALL SBIT1 (IQ(LQDW),15).
JZOUT will see this flag,
it will reset it to zero,
and it will leave the working space unchanged.
\section{Note on processor timing}
\smark{Notes}

JZ91 uses the KERNLIB routine TIMED (Z 007)
for measuring the time spent in each processor.
TIMED is called every time the processor level changes
and the value returned is added into the right Q(LQSV+5).

If the user also wants to use TIMED to time a section
of his code inside (!) a processor,
he can do this.
But unless he follows the recomendation below,
he will invalidate the timing figures for the particular
processor.
To keep things right, he should do this :
\bvb
       CALL TIMED (T)
       Q(LQSV+5)=Q(LQSV+5) + T
          user code to be timed
       CALL TIMED (T)
       Q(LQSV+5)=Q(LQSV+5) + T
\end{verbatim}
\lspa
The first call marks the start time of the user code;
the time spent in the processor till this moment
is cumulated into the SV bank.
The second call brings the time interval of the user
code,
it too is cumulated.
Note that you cannot so measure an interval
across a processor call.
\newpage
\section{Off-line initialization of a processor}
\smark{Notes}

In case the initialization part of a processor is bulky
it may be convenient to split it off from the processor proper
into a separate subroutine to be called just once from the root,
so as to have it executed and then out of the way.
Suppose we have the processor BB,
and we split the initialization off into subroutine BBIN.
This might then look as follows :

\bvb
      SUBROUTINE BBIN

+CDE, Q.

      CALL JZIN ('BB  ',0,36,0)
      IF (IQUEST(1))         11, 17, 99

C--       Initialize constants if and only if not set from title
   11 Q(LQAN+1) = .0025
      Q(LQAN+2) = .3

C--       Complete initialization calculating derived constants
   17 Q(LQAN+3) = .5 * SIN (Q(LQAN+2))

      IQ(LQSV+2) = 0       to reset the entry count from 1 to zero
   99 CALL JZOUT ('BB  ')
      RETURN
      END
C=====================================================

      SUBROUTINE BB

+CDE, Q.

      CALL JZIN ('BB  ',0,0,0)
      IF (IQUEST(1).LE.0)  CALL ZFATAM ('BB, NOT INITIALIZED.')
      CALL MZWORK (...)

         processor body

      CALL JZOUT ('BB  ')
      RETURN
      END
C=====================================================

Note that in subroutine BB it is still wise to check on IQUEST(1)
in case the explicit  CALL BBIN from the main program has been lost.
\end{verbatim}
\lspa
\newpage
\markboth{appendix: bank JZ91}{appendix: bank JZ91}
\vspace*{4mm}
{\Large\bf Appendix: JZ91 data structure and bank descriptions}

\vspace*{4mm}
{\large\bf JZ91 - header bank}

address : LQJZ system link in /JZUC/

\bvb
    - 2*JQMLEV+6  current SV bank for depth JQMLEV
             ...  ...
              +7                            1
      - JQMLEV+6  current SV bank for depth 0  (root)

      - JQMLEV+5  down call bank for depth JQMLEV-1
             ...  ...
             - 7                           1
             - 6  down call bank for depth 0  (root)
             - 5  zero  (= LQUP for the root)

             - 4  fan-out bank for immediate access to SV banks
             - 3  linear chain of SV banks
             - 2  linear chain of pending JZFL derived title banks
links :      - 1  linear chain of pending JZAN title banks

data :         1  guard word
               2  NQLINK   working space parameters
               3  NDATA    at level 0 (= root)

               4  NQLINK
               5  NDATA    at level 1 (= 1 below the root)
             ...  ...

      2*JQMLEV+1  NDATA    at level JQMLEV-1
      2*JQMLEV+2  guard word

        JQMLEV is MAXLEV of JZINIT, the maximum call depth
\end{verbatim}
\lspa

{\large\bf Fan-out bank}

address : LQ(LQJZ-4)

\bvb
        link  -J  address of SV bank J

        data   1  N = length of the list,  J=1,...,N

             J+1  ID of SV bank J
\end{verbatim}
\lspa
\newpage
\markboth{appendix: bank JZSV}{appendix: bank JZSV}
{\large\bf JZSV - bank of support variables}

One such bank for each processor initialized

\bvb
links :
      -(NLSV+3)  saved working space link 1
            ...  ...
            - 4  saved working space link NLSV
            - 3  two links reserved for the user
            - 2
            - 1  reserved

data :
     LQSV   + 0  status word
                 bits  1/8 = LV, the processor has been init. for this level
                 bits 9/16 = JQNACC-9, extra account words, normally zero
                 bit 17 set : constants are title initialized
            + 1  processor ID in A4 format
            + 2  number of calls to this processor
            + 3  NLSV working space links to be saved
            + 4  NDSV working space data words to be saved
            + 5  cumulate time for current call
            + 6  longest time interval for this processor
            + 7  cumulated execution time for this processor
            + 8  intermediate time cumulator (to improve precision)
      [ + 9 ...  possibly extra accounting words ]

                 LCD = LQSV + JQNACC    (constant in /JQC/)

     LCD    + 0  NCD = number of conditions to be recorded
              1  count condition 1 and lower
              2  count condition 2
            ...  ...
            NCD  count condition NCD and higher

                 LAN = LCD + NCD + 1     --> LQAN

     LAN    + 0  NAN = number of processor constants
              1  constant 1
            ...  ...
            NAN  constant NAN

                 LDSV = LAN + NAN + 1

     LDSV   + 0  saved working space data word 1
            ...  ...
         NDSV-1  saved data word NDSV

                 LFL = LDSV + NDSV    only in P=QDEBUG version

     LFL    + 0  NFL = number of flag words
            + 1  flag word 1
            ...  ...
            NFL  flag word NFL
\end{verbatim}
\lspa
