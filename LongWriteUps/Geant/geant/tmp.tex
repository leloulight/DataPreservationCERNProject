%%%%%%%%%%%%%%%%%%%%%%%%%%%%%%%%%%%%%%%%%%%%%%%%%%%%%%%%%%%%%%%%%%%
%                                                                 %
%  GEANT manual in LaTeX form                                     %
%                                                                 %
%  Version 1.00                                                   %
%                                                                 %
%  Last Mod. 8 June 1993  17:20  MG                               %
%                                                                 %
%%%%%%%%%%%%%%%%%%%%%%%%%%%%%%%%%%%%%%%%%%%%%%%%%%%%%%%%%%%%%%%%%%%
\documentstyle[11pt,fleqn,epsfig,crngeant,bibunits,fontcmds]{cernman}
\newcommand{\Title}{GEANT User's Guide}%           Title for document
\psfigdriver{DVIPS}
\makeindex
\romanfont{times}
\PScommands% Initialize PS boxes
\newmathalphabet*{\mathtt}{cmtt}{m}{n}
\newmathalphabet*{\mathbf}{cmr}{b}{n}
\begin{document}
%  ==================== Front material ============================
\include{geafront}
%  ==================== Body of text ==============================
\pagenumbering{arabic}
\setcounter{page}{1}
 
%%%%%%   Catalog of Program packages and entries%%%%
 
\def\Rtnr{Catalog}%Dummy routine name to appear at bottom of page
\include{geantcat}
 
\let\LARGE\large
\let\Large\large
\let\DL\DLtt 

% Here come the different files to be included
 
%%     TEMP part     %%
 
\begin{bibunit}[unsrt]
\renewcommand{\bibname}{TEMP Bibliography}
%%%%%%%%%%%%%%%%%%%%%%%%%%%%%%%%%%%%%%%%%%%%%%%%%%%%%%%%%%%%%%%%%%%
%                                                                 %
%  GEANT manual in LaTeX form                                     %
%                                                                 %
%  Version 1.00                                                   %
%  Last Mod.  8 June 1993  1300   MG                              %
%                                                                 %
%%%%%%%%%%%%%%%%%%%%%%%%%%%%%%%%%%%%%%%%%%%%%%%%%%%%%%%%%%%%%%%%%%%
\Authors {G.N. Patrick, L.Urban}   \Origin {Same}
\Version{Geant 3.16}\Routid{PHYS341}
\Submitted {26.10.84}  \Revised{19.12.92}
\Makehead{Simulation  of   discrete Bremsstrahlung by electrons}

\section{Subroutines}

\Shubr{GBREME}{}

\Rind{GBREME} generates a photon from Bremsstrahlung of an electron by
treating it as a discrete process.
The secondary photon energy is sampled from a parameterization of
the Bremsstrahlung calculation of Seltzer and Berger
\cite{bib-SEL1} for electron energies below 10 GeV, and from the screened
Bethe-Heitler cross-section above 10 GeV.
Midgal corrections are applied in both cases. 
For the angular distribution of the photon, is calls the function
\Rind{GBTETH}.
 
Input :  via common block \FCind{/GCTRAK/}
 
Output:  via common block \FCind{/GCKING/}.
 
\Rind{GBREME} is called automatically from the tracking
routine \Rind{GTELEC} if, and when,
the parent electron reaches its radiation point during the
tracking stage of GEANT.
 
\Sfunc{GBTETH}{THETA = GBTETH(ENER,PARTM,EFRAC)}

\Rind{GBTETH} calculates the angular distribution for the $e^+ e^-$-pair
in pair production and for the photon in Bremsstrahlung.
In case of Bremsstrahlung it gives the
scaled angle for a photon radiated by an electron (mass {\tt PARTM}) of 
energy {\tt ENER}. The energy of photon is {\tt EFRAC} times
the initial energy of the electron.
\Rind{GBTETH} is called by \Rind{GBREME}.

\section {Method}
 
The Coulomb corrected Bethe-Heitler differential cross-section for
production of a photon of energy $\epsilon E$ 
($\epsilon = k/E$, where $k$ is the photon energy)
by an electron of primary energy $E$ is given as in 
\notHTML{\cite{bib-WILL,bib-BUTC,bib-EGS3}}\HTML{\cite{bib-WILL}, \cite{bib-BUTC}, \cite{bib-EGS3}}:
 
\begin{equation}
\frac{d\sigma(Z,E,\epsilon)} {d \epsilon} =
      \frac{r_0^2 \alpha Z[Z +\xi(Z)]}{\epsilon}
    \left\{[1+(1 -\epsilon)^2][\Phi_1(\delta )
   - F(Z)] -\frac{2} {3}(1 - \epsilon)[\Phi_2(\delta) - F(Z)]\right\}
\end{equation}
where, $\Phi_i(\delta)$ are the screening functions depending on the screening
variable $\delta$
\[ \begin{array}{ll}

\begin{array}{l}
\delta  =  \frac {\Dstm 136m}{\Dstm Z^{1/3}E} \end{array}
            \frac{\Dstm \epsilon}{\Dstm (1-\epsilon)}
              &        m = \mbox{electron mass} \\
& \\
\noindent \left.\begin{array}{l}
\Phi_1(\delta)  =  20.867 - 3.242\delta +0.625\delta^2  \\
\Phi_2(\delta)  =  20.209 - 1.930\delta - 0.086\delta^2  \\
\end{array}        \right\} & \; \delta\leq 1 \\
\begin{array}{l} \Phi_1(\delta) = \Phi_2(\delta) =  
21.12 - 4.184 \ln(\delta+0.952) \end{array} 
                    & \; \delta > 1  \\
&\\
\begin{array}{l} F(Z) \end{array}  = \left\{\begin{array}{l}
        4/3 \ln Z          \\
        4/3 \ln Z+4f_c(Z)  \\
        \end{array}\right.      
&  \begin{array}{l}
E<0.05 \; GeV      \\
E\geq 0.05 \; GeV  \\
\end{array}  \\
\\
\begin{array}{l}  \xi(Z)=\frac{\Dstm\ln {(1440/Z^{2/3})}}{\Dstm\ln
{(183/Z^{1/3})}-f_c(Z)} \end{array}  & \\
\\
\begin{array}{l} f_c(Z) \mbox{\qquad the Coulomb correction function}
\end{array}  \\
\end{array} \]

%all these arrays of one column are here just to line up the text
\begin{eqnarray*}
f_c(Z) & = & a(1/(1+a)+0.20206-0.0369a+0.0083*a^2-0.002a^3)\\
    a   & = & (\alpha*Z)^2                                     \\
\alpha  & = & 1./137.
\end{eqnarray*}
 The kinematically allowed region for $\epsilon$ is
\begin{equation}
\epsilon_c \equiv \frac{k_c}{E} \leq \epsilon \leq  1- \frac{m}{E} 
\equiv \epsilon_1
\end{equation}
where,  $k_c$ is the photon cut-off energy below which the Bremsstrahlung
is treated as a continuous energy loss ({\tt BCUTE} in the program).
The cross-section (1) can be decomposed as
(the normalization is not important !)
\begin{equation}
\frac{d \sigma}{d \epsilon} =  \sum^{2}_{i=1} \alpha_i f_i(\epsilon)
      g_i(\epsilon )
\end{equation}
\begin{flushleft}
$\begin{array}{l@{}l@{\hspace{4cm}}l}
\mbox{where} & & \\
& \alpha_1 = F_{10}\ln \frac{\Tstm \epsilon_1}{\Tstm \epsilon_c}&
  \alpha_2 = F_{20}\frac{\Tstm 3}{\Tstm 8}\
             \frac{\Tstm \epsilon_1^2 -\epsilon_c^2}{\Tstm 1-\epsilon_c}\\
& & \\
& f_1 (\epsilon) = \frac{\Tstm 1}
                 {\Tstm \epsilon \; \ln {(\epsilon_1/ \epsilon_c)}} &
  f_2(\epsilon) = \frac{\Tstm 2 \epsilon}{\Tstm \epsilon_1^2 -\epsilon_c^2}\\
& g_1(\epsilon)=\frac{\Tstm F_1}{\Tstm F_{10}}
                \frac{\Tstm 1-\epsilon}{\Tstm 1-\epsilon_c} M(\epsilon)&
g_2(\epsilon )  = \frac{\Tstm F_2}{\Tstm F_{20}} . C_M (\epsilon )\\
\mbox{and} & & \\
& F_1(\delta) = 3\Phi_1(\delta)-\Phi_2(\delta)-2F(Z)  &
  F_{10}=F_1(\delta_{min})                     \\
& F_2(\delta) = 2\Phi_1(\delta)-2F(Z)                 &
  F_{20} = F_2(\delta_{min})                   \\
\\
& \delta_{min} = \frac{\Tstm 136m}{\Tstm Z^{1/3}E}
                \frac{\Tstm \epsilon_c}{\Tstm 1- \epsilon_c}
\end{array}$
\end{flushleft}
$\delta_{min}$ is the minimum value of the variable $\delta$, and
$C_M(\epsilon)$  is the Midgal correction factor \cite{bib-MIGD}:
\[
C_M (\epsilon)  =\frac{1 + C_0 / \epsilon_1^2}
               {1 + C_0 / \epsilon^2} 
\]
 where
\[
C_0 =\frac{nr_0 \lambdar^2 }{\pi}
\]
\begin{DL}{MM}
\item[$n$]           electron density in the medium
\item[$r_0$]         classical electron radius
\item[$\lambdar$]    reduced Compton wavelength of the electron.
\end{DL}
This correction decreases the cross-section for low photon energy.
Using the decomposition (3) the sampling of $\epsilon$ can be done by
(for a set of random numbers: $0\leq r_i\leq 1$)
\begin{OL}
\item  selecting the integer i (1 or 2)\\
$ \begin{array}{ll}
\mbox{if  }r_0 \leq BPAR = \alpha_1/(\alpha_1+\alpha_2), & \mbox{then  } i=1 \\
\mbox{if  } r_0 > BPAR,                                  & \mbox{then  } i=2
\end{array}$
\item
sampling $\epsilon$ from $f_i(\epsilon) $ \\
$\begin{array}{ll}
\epsilon =\epsilon c(\epsilon_1/\epsilon_c)^{r1} & \mbox{when  } i=1 \\
\epsilon  =\sqrt{\epsilon_c^2 + r_1 (\epsilon m^2-\epsilon_c^2)} 
                                                 & \mbox{when  } i=2 
\end{array}$
\item
calculating the rejection function $g_i(\epsilon)$ and\\
if $ r_2>g_i(\epsilon)$  rejecting $\epsilon$, starting again from 1\\
if $ r_2\leq g_i(\epsilon)$ accepting $\epsilon$
\end{OL}
As in the case of the pair production after sampling $\epsilon$ from
$f_i(\epsilon)$,
we have to check that
\[
\delta  = \frac{136 m} {Z^{1/3}E} \frac{\epsilon} {1 - \epsilon_c}
       \leq \delta_{max}=
       \exp  \left(\frac{21.12-f(Z)} {4.184}\right)  - 0.952
\]
and, if this is not the case, we have to start again from step 1.
 
The decomposition (3) used here is simpler and more effective
than that used in the earlier versions of GEANT and in {\tt EGS},
and the method outlined above has no convergence problems.
 
After the successful sampling of $\epsilon$, \Rind{GBREME} generates the
polar angles
of the radiated photon with respect to an axis defined along the parent
electron's momentum.
The energy-angle distribution is given by 
Tsai~\notHTML{\cite{bib-TSAI,bib-TSAI-err}}\HTML{\cite{bib-TSAI}, \cite{bib-TSAI-err}}
as following:
\begin{eqnarray}
\frac{d \sigma}{dkd \Omega} 
& = & \frac{2 \alpha^{2}e^{2}}{\pi k m^{4}}
      \left\{ \left[ \frac{2y-2}{(1+l)^2}+\frac{12l(1-y)}{(1+l)^4}\right]
      (Z^{2}+Z) \nonumber  \right. \\
&   & \mbox{} + \left. \left[ \frac{2-2y-y^{2}}{(1+l)^2}- 
      \frac{4l(1-y)}{(1+l)^4}
      \right]
      \left[ X-2Z^{2}f((\alpha Z)^{2})\right]
      \right\} \nonumber \\
\end{eqnarray}
where $k$ is the photon energy, $E$ is the initial electron energy,
$y = k/E$ and $l = E^{2} \theta^{2}/m^{2}$. This distribution
is quite complicated for sampling and, furthermore, for a variable
$u = E \theta/m$, shows a very weak dependence on $Z$, $E(k)$,
$y = k/E$. Thus, the distribution can be approximated by a function
\begin{equation}
f(u) = C \left( u e^{-au} + d u e^{-3au} \right)
\end{equation}
where
\begin{eqnarray*}
C & = & \frac{9a^{2}}{9 + d} \\
a & = & 0.625 \\
d & = & 27.0 \\
\end{eqnarray*}
The sampling of the function $f(u)$ can be done in the following way
($r_{i}$ are uniformly distributed random numbers 
in [0,1]):
\begin{enumerate}
\item Choose between $u e^{-au}$ and $d u e^{-3au}$, \\
      weights: $9/(9+d)$ and $d/(9+d)$. \\
      If $r_{1} < 9/(9+d)$ then $a=b$, \\
      else $b=3a$.
\item Sample $u e^{-bu}$, \\
      $u=-(\ln r_{2} + \ln r_{3})/b$.
\end{enumerate}
The azimuthal angle, $\Phi$, is generated isotropically.
This information is used to calculate the momentum vector of the radiated
photon, to transform it to the GEANT coordinate system and 
to store the result into common block \FCind{/GCKING/}. Also, the momentum of the parent electron 
is updated.

\subsection{Restrictions}

\begin{enumerate}
\item
Effects due to breakdown of the Born approximation at low energies are ignored.
\item
Target materials composed of compounds or mixtures are treated identically
to elements using the effective atomic number $Z$ calculated in
\Rind{GSMIXT}
(this is not the case when computing the mean free path!).
\end{enumerate}
\putbib[cnasbibl,geabibl]
\end{bibunit}

%  ==================== Index material ============================

\setcounter{page}{1}%                                Reset page counter
\def\Rtnr{Index}%Dummy routine name to appear at bottom of page
\input{\jobname.ind} % index

\end{document}
