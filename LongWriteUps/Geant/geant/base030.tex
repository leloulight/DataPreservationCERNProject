%%%%%%%%%%%%%%%%%%%%%%%%%%%%%%%%%%%%%%%%%%%%%%%%%%%%%%%%%%%%%%%%%%%
%                                                                 %
%  GEANT manual in LaTeX form                                     %
%                                                                 %
%  Version 1.00                                                   %
%                                                                 %
%  Last Mod.  9 June 1993 19:30  MG                               %
%                                                                 %
%%%%%%%%%%%%%%%%%%%%%%%%%%%%%%%%%%%%%%%%%%%%%%%%%%%%%%%%%%%%%%%%%%%
\Authors{F.Carminati}      \Origin{GEANT3}
\Version{Geant 3.12}\Routid{BASE030}
\Submitted{01.10.84}  \Revised{05.08.93}
\Makehead{Overview of COMMON Blocks}
\section{Introduction}
 
The communication between program segments of the GEANT3 system
is ensured by the contents of the data structures and by the definition
of `long range' variables in several common blocks.
In addition, within the program segments,
the subroutines communicate with each other through explicit arguments
and through the common block variables.
 
The data structures are described in separate papers. Here, the
main features of the common blocks used in GEANT3 are summarized,
with special mention of the variables initialized in \Rind{GINIT}
and of the possibility of overriding them through data records
{\tt [BASE040]} or interactive commands {\tt [XINT]}.
In most of the cases there is a correspondance between a
given data structure and a given common block where the current contents of
the banks are stored.
The labelled common blocks are accessible through Patchy sequences
identified by the name of the {\tt COMMON}. They are defined in the Patch
\Rind {GCDES}.
 
{\bf Note:}
 
Unless otherwise specified, the long range variables are
initialized in \Rind{GINIT}. When non-zero, default values are
quoted between brackets. If the value may be modified
the keyword for the data record and for the interactive
command is also given in bold characters between brackets.
 
\subsection{Dynamic memory}
 
The GEANT3 data structures are stored in the
common \FCind{/GCBANK/} accessible through the following Patchy sequence:
\FComm{GCBANK}{Dynamic core for the GEANT data structures}
\begin{verbatim}
      PARAMETER (KWBANK=69000,KWWORK=5200)
      COMMON/GCBANK/NZEBRA,GVERSN,ZVERSN,IXSTOR,IXDIV,IXCONS,FENDQ(16)
     +             ,LMAIN,LR1,WS(KWBANK)
      DIMENSION IQ(2),Q(2),LQ(8000),IWS(2)
      EQUIVALENCE (Q(1),IQ(1),LQ(9)),(LQ(1),LMAIN),(IWS(1),WS(1))
      EQUIVALENCE (JCG,JGSTAT)
      COMMON/GCLINK/JDIGI ,JDRAW ,JHEAD ,JHITS ,JKINE ,JMATE ,JPART
     +      ,JROTM ,JRUNG ,JSET  ,JSTAK ,JGSTAT,JTMED ,JTRACK,JVERTX
     +      ,JVOLUM,JXYZ  ,JGPAR ,JGPAR2,JSKLT
C
\end{verbatim}
The \FCind{/GCLINK/} variables are pointers to the GEANT3 data structures in
the \FCind{/GCBANK/} common.
They belong to a permanent area declared in \Rind{GZINIT}.
\subsection{Other labelled COMMON blocks}
\FComm{GCCUTS}{Tracking thresholds}
\begin{verbatim}
 COMMON/GCCUTS/CUTGAM,CUTELE,CUTNEU,CUTHAD,CUTMUO,BCUTE,BCUTM,
 +             DCUTE,DCUTM,PPCUTM,TOFMAX,GCUTS(5)
\end{verbatim}
\begin{DLtt}{MMMMMMMMMM}
\item[CUTGAM]    Kinetic energy cut threshold for gammas
({\tt 0.001, CUTS})
\item[CUTELE]    Kinetic energy cut threshold for electrons
({\tt 0.001, CUTS})
\item[CUTNEU]    Kinetic energy cut threshold for neutral hadrons
({\tt 0.01, CUTS})
\item[CUTHAD]    Kinetic energy cut threshold for charged hadrons
({\tt 0.01, CUTS})
\item[CUTMUO]    Kinetic energy cut threshold for muons
({\tt 0.01, CUTS})
\item[BCUTE]     Kinetic energy cut threshold for electron
                 Bremsstrahlung ({\tt CUTGAM, CUTS})
\item[BCUTM]    Kinetic energy cut threshold for muon Bremsstrahlung
({\tt CUTGAM, CUTS})
\item[DCUTE]   Kinetic energy cut threshold for electron delta rays
({\tt CUTELE, CUTS})
\item[DCUTM]  Kinetic energy cut threshold for muon or hadron delta rays
({\tt CUTELE, CUTS})
\item[PPCUTM] Total energy cut threshold for \Pep\Pem pair production by
              muon ({\tt 0.002, CUTS})
\item[TOFMAX]  Tracking cut threshold on time of flight integrated
from primary interaction time ({\tt $10^{10}$, CUTS})
\item[GCUTS]   For user applications   ({\tt CUTS})
\end{DLtt}
{\bf Note:}
The cuts {\tt BCUTE, BCUTM} and {\tt DCUTE, DCUTM} are given
the respective default values {\tt CUTGAM} and {\tt CUTELE}.
Experienced
users can make use of the facility offered (command {\tt CUTS})
to change {\tt BCUTE, DCUTE, BCUTM} and {\tt DCUTM}.
 
\FComm{GCDRAW}{Variables used by the drawing package}
\begin{verbatim}
      COMMON/GCDRAW/NUMNOD,MAXNOD,NUMND1,LEVVER,LEVHOR,MAXV,IPICK,
     + MLEVV,MLEVH,NWCUT,JNAM,JMOT,JXON,JBRO,JDUP,JSCA,JDVM,JPSM,
     + JNAM1,JMOT1,JXON1,JBRO1,JDUP1,JSCA1,JULEV,JVLEV,
     + LOOKTB(16),
     + GRMAT0(10),GTRAN0(3),IDRNUM,GSIN(41),GCOS(41),SINPSI,COSPSI,
     + GTHETA,GPHI,GPSI,GU0,GV0,GSCU,GSCV,NGVIEW,
     + ICUTFL,ICUT,CTHETA,CPHI,DCUT,NSURF,ISURF,
     + GZUA,GZVA,GZUB,GZVB,GZUC,GZVC,PLTRNX,PLTRNY,
     + LINATT,LINATP,ITXATT,ITHRZ,IPRJ,DPERS,ITR3D,IPKHIT,IOBJ,LINBUF,
     + MAXGU,MORGU,MAXGS,MORGS,MAXTU,MORTU,MAXTS,MORTS,
     + IGU,IGS,ITU,ITS,NKVIEW,IDVIEW,
     + NOPEN,IGMR,IPIONS,ITRKOP,IHIDEN,
     + DDUMMY(18)
C
\end{verbatim}
\begin{DLtt}{MMMMMMMMMM}
\item[NUMNOD] number of nodes in non-optimized tree
\item[MAXNOD] max. number of nodes of non-optimized tree.
({\tt MIN(NLEFT,16,200)}).
\item[NUMND1] number of nodes in optimized tree
\item[LEVVER] vertical level in the tree currently scanned by tree routines
\item[LEVHOR] horizontal node in the tree currently scanned by
tree routines
\item[MAXV] max vertical levels in the tree to be scanned by tree routines
\item[IPICK] node selected by \Rind{GDTREE}
\item[MLEVV] number of vertical levels in the last tree scanned
\item[MLEVH] number of horizontal nodes in the last tree scanned
\item[NWCUT] max. workspace allocated by cut routines, ({\tt 5000})
\item[JNAM-JVLEV]  pointers used by the tree routines
\item[LOOKTB] colour look-up table, ({\tt LOOKTB(I)=I,I=1,16})
\item[GRMAT0] rotation matrix saved by \Rind{GDRVOL}, ({\tt unitary matrix})
\item[GTRAN0] translation matrix saved by \Rind{GDRVOL}, ({\tt 0.,0.,0.})
\item[IDRNUM] flag for \Rind{GDRAW}, set to 1 when called by \Rind{GDRVOL},
({\bf 0})
\item[GSIN] sine table (at $9^{\circ}$ steps)
\item[GCOS] cosine table (at $9^{\circ}$ steps)
\item[SINPSI] {\tt SIN(GPSI*DEGRAD)}
\item[COSPSI] {\tt COS(GPSI DEGRAD)}
\item[GTHETA] $\theta$ angle of the parallel projection of 3-dimensional
images on the screen (${\tt 45^{\circ}}$)
\item[GPHI]  $\phi$ angle of the parallel projection of 3-dimensional
images on the screen (${\tt 135^{\circ}}$)
\item[GPSI]  $\psi$ angle of rotation of the image on the screen
(${\tt 0^{\circ}}$)
\item[GU0]  U position (X in screen coordinates) of the origin of the drawing
screen in screen units ({\tt 10.})
\item[GV0]  V position (Y in screen coordinates) of the origin of the drawing
screen in screen units ({\tt 10.})
\item[GSCU]   scale factor for the U screen coordinate  ({\tt 0.015})
\item[GSCV]   scale factor for the V screen coordinate ({\tt 0.015})
\item[NGVIEW] flag informing \Rind{GDFR3D} and \Rind{GD3D3D} if
the view point has changed  ({\tt 0})
\item[ICUTFL] flag informing \Rind{GDRAW} if it was called by
{\it cut} routines
\item[ICUT] axis along which the cut is performed (1, 2 or 3, 0 if no cut)
\item[CTHETA] $\theta$ angle of cut supplied to \Rind{GDRAWX} (used by
\Rind{GDCUT})
\item[CPHI] $\phi$ angle of cut supplied to \Rind{GDRAWX} (used by
\Rind{GDCUT})
\item[DCUT] coordinate value (along axis {\tt ICUT)} at which the cut is
performed
\item[NSURF] number of surfaces stored in {\tt SURF}
\item[ISURF] pointer for array {\tt SURF}
\item[GZUA] zoom parameter (horizontal scale factor)  ({\tt 1.})
\item[GZVA] zoom parameter (vertical scale factor)  ({\tt 1.})
\item[GZUB] zoom parameter  ({\tt 0.})
\item[GZVB] zoom parameter  ({\tt 0.})
\item[GZUC] zoom parameter  ({\tt 0.})
\item[GZVC] zoom parameter  ({\tt 0.})
\item[PLTRNX] screen and plotter X range, {\tt PLTRNX} $\times$
{\tt PLTRNY} cm. ({\tt 20.})
\item[PLTRNY] screen and plotter Y range, {\tt PLTRNX} $\times$
{\tt PLTRNY} cm. ({\tt 20.})
\item[LINATT] current line attributes ({\tt colour=1, width=1, style=1,
fill=1})
\item[LINATP] permanent line attributes  ({\tt LINATT)}
\item[ITXATT] current text attributes  ({\tt colour = 1, width = 1})
\item[ITHRZ] string containing the status of {\tt THRZ} option of
            \Rind{GDOPT}  ({\tt 'OFF '})
\item[IPRJ] string containing the status of {\tt PROJ} option of
            \Rind{GDOPT}  ({\tt 'PARA'})
\item[DPERS] distance of the view point from
the origin (used with perspective) ({\tt 1000.})
\item[ITR3D]track being scanned (used together with {\tt THRZ} option)
\item[IPKHIT]flag for \Rind{GPHITS}, if
$>0$ then print only hit number, ({\tt 0})
\item[IOBJ]type of the object being drawn (detector, track, hit, etc.)
({\tt 0})
\item[LINBUF]flag informing \Rind{GDRAWV} if line buffering is wanted or
not ({\tt 0})
\item[MAXGU]current physical number of words for graphic unit banks
\item[MORGU]number of words to be pushed in graphic unit banks
\item[MAXGS]current physical number of words for graphic segment banks
\item[MORGS]number of words to be pushed in graphic segment banks
\item[MAXTU]current physical number of words for text unit banks
\item[MORTU]number of words to be pushed in text unit banks
\item[MAXTS]current physical number of words for text segment banks
\item[MORTS]number of words to be pushed in text segment banks
\item[IGU]pointer to current graphic unit bank
\item[IGS]pointer to current graphic segment bank
\item[ITU]pointer to current text unit bank
\item[ITS]pointer to current text segment bank
\item[NKVIEW]number of view data banks ({\tt 0})
\item[IGVIEW]current view bank number or 0 for screen ({\tt 0})
\item[NOPEN]unused ({\tt 0})
\item[IGMR]flag informing if {\tt APOLLO-GMR} is being used ({\tt 0})
\item[IPIONS]unused ({\tt 0})
\item[ITRKOP]string containing the status of {\tt TRAK} option of
\Rind{GDOPT} ({\tt 'LINE'})
\item[DDUMMY]array of dummy words
\end{DLtt}
\FComm{GCFLAG}{Flags and variables to control the run}
\begin{verbatim}
      COMMON/GCFLAG/IDEBUG,IDEMIN,IDEMAX,ITEST,IDRUN,IDEVT,IEORUN
     +        ,IEOTRI,IEVENT,ISWIT(10),IFINIT(20),NEVENT,NRNDM(2)
C
\end{verbatim}
\begin{DLtt}{MMMMMMMMMM}
\item[IDEBUG]Flag set internally to 1 to activate debug
output if {\tt IEVENT} (below)
\item[IDEMIN]     is greater or equal to {\tt IDEMIN} ({\tt DEBU})
\item[IDEMAX]     and less or equal to {\tt IDEMAX}   ({\tt DEBU})
\item[ITEST]Flag to request printing of {\tt IEVENT, IDEVT} and {\tt
NRNDM} (below) every {\tt ITEST} events ({\tt DEBU})
\item[IDRUN]Current user run number   ({\tt 1, RUNG})
\item[IDEVT]Current user event number  ({\tt 1, RUNG})
\item[IEORUN]Flag to terminate run if non-zero
\item[IEOTRI]Flag to abort current event if non-zero
\item[IEVENT]Current event sequence number ({\tt 1})
\item[ISWIT]Flags reserved for user in relation to debug ({\tt 0, SWIT})
\item[IFINIT]Flags used for initialisation
\item[NEVENT]Number of events to be processed  ({\tt 10000000, TRIG})
\item[NRNDM]Initial seeds for the random number generator. If
{\tt NRNDM(2)=0} the sequence number {\tt NRNDM(1)} is taken from a
predefined set of 215 indipendent sequences. Otherwise the random
number generator is initialised with the two seeds {\tt NRNDM(1), NRNDM(2)}.
({\tt 9876, 54321})
\end{DLtt}
\FComm{GCGOBJ}{CG package variables}
\begin{verbatim}
      PARAMETER (NTRCG=1)
      PARAMETER (NWB=207,NWREV=100,NWS=1500)
      PARAMETER (C2TOC1=7.7, C3TOC1=2.,TVLIM=1296.)
      COMMON /GCGOBJ/IST,IFCG,ILCG,NTCUR,NFILT,NTNEX,KCGST
     +             ,NCGVOL,IVFUN,IVCLOS,IFACST,NCLAS1,NCLAS2,NCLAS3
      COMMON /CGBLIM/IHOLE,CGXMIN,CGXMAX,CGYMIN,CGYMAX,CGZMIN,CGZMAX
C
\end{verbatim}
\begin{DLtt}{MMMMMMMMMM}
\item[NTRCG]
\item[NWB]
\item[NWREV]
\item[NWS]
\item[C2TOC1]
\item[C3TOC1]
\item[TVLIM]
\item[IST]
\item[IFCG]
\item[ILCG]
\item[NTCUR]
\item[NFILT]
\item[NTNEX]
\item[KCGST]
\item[NCGVOL]
\item[IVFUN]
\item[IVCLOS]
\item[IFACST]
\item[NCLAS1]
\item[NCLAS2]
\item[NCLAS3]
\item[IHOLE]
\item[CGXMIN]
\item[CGXMAX]
\item[CGYMIN]
\item[CGYMAX]
\item[CGZMIN]
\item[CGZMAX]
\end{DLtt}
\FComm{GCHILN}{Temporary link area for the CG package}
\begin{verbatim}
      COMMON/GCHILN/LARECG(2), JCGOBJ, JCGCOL, JCOUNT, JCLIPS,
     +              ICLIP1, ICLIP2
*
\end{verbatim}
\FComm{GCJLOC}{JMATE substructure pointers for current material}
\begin{verbatim}
      COMMON/GCJLOC/NJLOC(2),JTM,JMA,JLOSS,JPROB,JMIXT,JPHOT,JANNI
     +                  ,JCOMP,JBREM,JPAIR,JDRAY,JPFIS,JMUNU,JRAYL
     +                  ,JMULOF,JCOEF,JRANG
C
\end{verbatim}
 
See {\tt [CONS199]}.
\FComm{GCJUMP}{Pointers for the jump package}
\begin{verbatim}
      PARAMETER    (MAXJMP=30)
      COMMON/GCJUMP/JUDCAY, JUDIGI, JUDTIM, JUFLD , JUHADR, JUIGET,
     +              JUINME, JUINTI, JUKINE, JUNEAR, JUOUT , JUPHAD,
     +              JUSKIP, JUSTEP, JUSWIM, JUTRAK, JUTREV, JUVIEW,
     +              JUPARA
      DIMENSION     JMPADR(MAXJMP)
      EQUIVALENCE  (JMPADR(1), JUDCAY)
*
\end{verbatim}
\FComm{GCKINE}{Kinematics of current track}
 
\begin{verbatim}
      COMMON/GCKINE/IKINE,PKINE(10),ITRA,ISTAK,IVERT,IPART,ITRTYP
     +      ,NAPART(5),AMASS,CHARGE,TLIFE,VERT(3),PVERT(4),IPAOLD
C
\end{verbatim}
\begin{DLtt}{MMMMMMMMMM}
\item[IKINE]  user integer word  ({\tt 0, KINE})
\item[PKINE]  user array of real   ({\tt 0, KINE})
\item[ITRA]   Current track number
\item[ISTAK]Current stack track number
\item[IVERT]Current vertex number
\item[IPART]Current particle number
\item[ITRTYP]Current particle tracking type
\item[NAPART] Name of current particle (ASCII codes stored in an integer
array, 4 characthers per word)
\item[AMASS]  Mass of current particle
\item[CHARGE]Charge of current particle
\item[TLIFE]Average life time of current particle
\item[VERT]Coordinates of origin vertex for current track
\item[PVERT]Track kinematics at origin vertex ({\tt PVERT(4)} not used)
\item[IPAOLD]Particle number of the previous track.
\end{DLtt}
\FComm{GCKMAX}{Size of the \FCind{/GCKING/} stack}
\begin{verbatim}
      INTEGER MXGKIN
      PARAMETER (MXGKIN=100)
\end{verbatim}
\FComm{GCMUTR}{Auxiliary variables for the CG package}
\begin{verbatim}
      PARAMETER (MULTRA=50)
      CHARACTER*4 GNASH, GNNVV, GNVNV
      COMMON/GCMUTR/NCVOLS,KSHIFT,NSHIFT,ICUBE,NAIN,JJJ,
     +              NIET,IOLDSU,IVOOLD,IWPOIN,IHPOIN,IVECVO(100),
     +              PORGX,PORGY,PORGZ,POX(15),POY(15),POZ(15),GBOOM,
     +              PORMIR(18),PORMAR(18),IPORNT,
     +              ICGP,CLIPMI(6),CLIPMA(6),
     +              ABCD(4),BMIN(6),BMAX(6),CGB(16000),CGB1(16000),
     +              GXMIN(MULTRA),GXMAX(MULTRA),GYMIN(MULTRA),
     +              GYMAX(MULTRA),GZMIN(MULTRA),GZMAX(MULTRA),
     +              GXXXX(MULTRA),GYYYY(MULTRA),GZZZZ(MULTRA)
*
      COMMON/GCMUTC/   GNASH(MULTRA),GNNVV(MULTRA),GNVNV(MULTRA)
*
\end{verbatim}
\FComm{GCKING}{Kinematics of generated secondaries}
\begin{verbatim}
      COMMON/GCKING/KCASE,NGKINE,GKIN(5,MXGKIN),
     +                           TOFD(MXGKIN),IFLGK(MXGKIN)
C
      PARAMETER (MXPHOT=1000)
      COMMON/GCKIN2/NGPHOT,XPHOT(11,MXPHOT)
C
\end{verbatim}
\begin{DLtt}{MMMMMMMMMM}
\item[KCASE] Mechanism which has generated the secondary particles
\item[NGKINE]Number of generated secondaries
\item[GKIN(1,I)]x component of momentum of ${\rm I}^{th}$ particle
\item[GKIN(2,I)]y component of momentum
\item[GKIN(3,I)]z component of momentum
\item[GKIN(4,I)]Total energy
\item[GKIN(5,I)]Particle code
\item[TOFD(I)]Time offset with respect to current time of flight
\item[IFLGK(I)]Flag controlling the handling of track by {\tt GSKING/GSSTAK}
\begin{DLtt}{MMMMM}
\item[$<0$=]particle is stored in  the temporary stack {\tt JSTAK} and in
the data structure {\tt JKINE} attached to vertex {\tt -IFLGK(I)}
\item[0 =]({\bf D}) particle is stored in the temporary stack {\tt JSTAK}
for further tracking
\item[1 =] like {\tt 0} but
particle is stored in {\tt JVERTX/JKINE} structure as well
\item[2 =] entry in {\tt JKINE} already exists for this track
\end{DLtt}
\item[NGPHOT] number of \v{C}erenkov photons generated in the current
step
\item[XPHOT(1,I)] x position of the ${\rm I}^{th}$ photon
\item[XPHOT(2,I)] y position
\item[XPHOT(3,I)] z position
\item[XPHOT(4,I)] x component of momentum
\item[XPHOT(5,I)] y component of momentum
\item[XPHOT(6,I)] z component of momentum
\item[XPHOT(7,I)] momentum of the photon
\item[XPHOT(8,I)] x component of the polarisation vector
\item[XPHOT(9,I)] y component of the polarisation vector
\item[XPHOT(10,I)] z component of the polarisation vector
\item[XPHOT(11,I)] time of flight in seconds of the photon
\end{DLtt}
\FComm{GCLINK}{See \FCind{/GCBANK/} above}
\FComm{GCLIST}{Various system and user lists}
\begin{verbatim}
      COMMON/GCLIST/NHSTA,NGET ,NSAVE,NSETS,NPRIN,NGEOM,NVIEW,NPLOT
     +       ,NSTAT,LHSTA(20),LGET (20),LSAVE(20),LSETS(20),LPRIN(20)
     +             ,LGEOM(20),LVIEW(20),LPLOT(20),LSTAT(20)
C
\end{verbatim}
\begin{DLtt}{MMMMMMMMMM}
\item[NHSTA] Number of histograms declared on data record {\tt HSTA }
\item[NGET] Number of data structures declared on data record {\tt GET}
\item[NSAVE]Number of data structures declared on data record {\tt SAVE}
\item[NSETS]Number of items described on data record {\tt SETS}
\item[NPRIN]Number of items described on data record {\tt PRIN}
\item[NGEOM]Number of items described on data record {\tt GEOM}
\item[NVIEW]Number of items described on data record {\tt VIEW}
\item[NPLOT]Number of items described on data record {\tt PLOT}
\item[NSTAT]Number of items described on data record {\tt STAT}. Obsolete.
\item[LHSTA,\ldots LSTAT]Corresponding user lists of items
({\tt HSTA,\ldots,STAT})
\end{DLtt}
{\tt LSTAT(1)} is reserved by the system for volume statistics.
\FComm{GCMATE}{Parameters of current material}
\begin{verbatim}
      COMMON/GCMATE/NMAT,NAMATE(5),A,Z,DENS,RADL,ABSL
C
\end{verbatim}
\begin{DLtt}{MMMMMMMMMM}
\item[NMAT]  Current material number
\item[NAMATE]Name of current material (ASCII codes stored in an integer
array, 4 characthers per word)
\item[A]Atomic weight of current material
\item[Z]Atomic number of current material
\item[DENS]Density of current material in ${\rm g \: \: cm^{-3}}$
\item[RADL]Radiation length of current material
\item[ABSL]Absorption length of current material
\end{DLtt}
\FComm{GCMULO}{Energy binning and multiple scattering}
 
Precomputed quantities for multiple scattering and energy binning for
{\tt JMATE} banks. See also {\tt [CONS199]} for the energy binning and
{\tt [PHYS325]} for a description of the variables {\tt OMCMOL} and
{\tt CHCMOL}.
\begin{verbatim}
      COMMON/GCMULO/SINMUL(101),COSMUL(101),SQRMUL(101),OMCMOL,CHCMOL
     +  ,EKMIN,EKMAX,NEKBIN,NEK1,EKINV,GEKA,GEKB,EKBIN(200),ELOW(200)
\end{verbatim}
\begin{DLtt}{MMMMMMMMMM}
\item[SINMUL]  Not used any more
\item[COSMUL]  Not used any more
\item[SQRMUL]  Not used any more
\item[OMCMOL]  Constant $\Omega_0$ of the Moli\'ere theory
\item[CHCMOL]  Constant of the Moli\'ere theory
\item[EKMIN]   Lower edge of the energy range of the tabulated cross
sections ({\tt $10^{-5}$, ERAN})
\item[EKMAX]   Upper edge of the energy range of the tabulated cross
sections ({\tt $10^{4}$, ERAN})
\item[NEKBIN]    Number of energy bins to be used ({\tt 90, ERAN})
\item[NEK1]    {\tt NEKBIN+1}
\item[EKINV]   $1/ \left ( \log_{10}({\tt EKMAX})-
\log_{10}({\tt EKMIN}) \right )$
\item[GEKA]    {\tt NEKBIN*EKINV}
\item[GEKB]    {\tt 1-GEKA*EKBIN(1)}
\item[EKBIN]   $\log \left ( {\tt ELOW} \right ) $
\item[ELOW]    Low edges of the energy bins
\end{DLtt}
\FComm{GCMZFO}{I/O descriptors of GEANT banks}
\begin{verbatim}
      COMMON/GCMZFO/IOMATE,IOPART,IOTMED,IOSEJD,IOSJDD,IOSJDH,IOSTAK
     +             ,IOMZFO(13)
C
\end{verbatim}
\FComm{GCNUM}{Current number for various items}
\begin{verbatim}
      COMMON/GCNUM/NMATE ,NVOLUM,NROTM,NTMED,NTMULT,NTRACK,NPART
     +            ,NSTMAX,NVERTX,NHEAD,NBIT
      COMMON /GCNUMX/ NALIVE,NTMSTO
C
\end{verbatim}
\begin{DLtt}{MMMMMMMMMM}
\item[NMATE]      Number of Materials
\item[NVOLUM]     Number of Volumes
\item[NROTM]      Number of Rotation matrices
\item[NTMED]      Number of Tracking media
\item[NTMULT]     Number of tracks processed in current event
                 (including secondaries), reset to 0 for each event
\item[NTRACK]    Number of Tracks in {\tt JKINE} bank for current event
\item[NPART]     Maximum particle code
\item[NSTMAX]    Maximum number of tracks in stack {\tt JSTAK}
                 for current event, reset to 0 for each event
\item[NVERTX]   Number of Vertices in {\tt JVERTX} mother bank for current event
\item[NHEAD]    Number of data words in the {\tt JHEAD} bank ({\tt 10})
\item[NBIT]    Number of bits per word (initialized in \Rind{GINIT}
               via {\tt ZEBRA})
\end{DLtt}
\begin{DLtt}{MMMMMMMMMM}
\item[NALIVE]Number of particles to be tracked in the parallel tracking stack
(see {\tt [TRAK???]}
\item[NTMSTO]Total number of tracks tracked in the current event so far. Same
as {\tt NTMULT} in \FCind{/GCTRAK/}.
\end{DLtt}
\FComm{GCOMIS}{Variables for the COMIS package}
\begin{verbatim}
      COMMON/GCOMIS/ICOMIS,JUINIT,JUGEOM,JUKINE,JUSTEP,JUOUT,JULAST
*
\end{verbatim}
\FComm{GCONST}{Basic constants}
\begin{verbatim}
      COMMON/GCONST/PI,TWOPI ,PIBY2,DEGRAD,RADDEG,CLIGHT ,BIG,EMASS
      COMMON/GCONSX/EMMU,PMASS,AVO
C
\end{verbatim}
\begin{DLtt}{MMMMMMMMMM}
\item[PI]         $\pi$ ({\tt ACOS(-1)})
\item[TWOPI]      $2\pi$
\item[PIBY2]      $\pi/2$
\item[DEGRAD]    Degree to radian conversion factor ($\pi/180$)
\item[RADDEG]    Radian to degree conversion factor ($180/\pi$)
\item[CLIGHT]    Light velocity ($2.99792458 \times 10^{10}
\: cm \: sec^{-1}$)
\item[BIG]       Arbitrary large number ($10^{10}$)
\item[EMASS]     Electron mass ($0.5110034 \times 10^{-3} \: GeV$)
\item[EMMU]      Muon mass ($0.105659 \: GeV$)
\item[PMASS]     Proton mass ($0.93828 \: GeV$)
\item[AVO]       Avogadro's number $\times 10^{23}$ ($0.6022045$)
\end{DLtt}
 
\FComm{GCOPTI}{Control of Geometry optimisation}
\begin{verbatim}
      COMMON/GCOPTI/ IOPTIM
C
\end{verbatim}
\begin{DLtt}{MMMMMMMMMM}
\item[IOPTIM]Optimization flag
\begin{DLtt}{MMMMM}
\item[-1 =] No optimisation at all. \Rind{GSORD} calls disabled
\item[~0 =] No optimisation. Only user calls to \Rind{GSORD} kept
\item[~1 =] All non-\Rind{GSORD}ered volumes are ordered along the best axis
\item[~2 =] All volumes are ordered along the best axis
\end{DLtt}
\end{DLtt}
\FComm{GCPARA}{Control of parametrized energy deposition}
\begin{verbatim}
      PARAMETER (LSTACK = 5000)
      LOGICAL    SYMPHI, SYMTEU, SYMTED
C
      COMMON    /GCPARA/
     +                   JJLOST, EPSMAX, JJWORK,
     +                   IFOUNP, IFOUNT, IFNPOT,
     +                   SYMPHI, SYMTEU, SYMTED
C
\end{verbatim}
\begin{DLtt}{MMMMMMMMMM}
\item[LSTACK] Dimension of the Energy ray stack
\item[JJLOST] Number of Energy rays lost in each tracking step
\item[EPSMAX] Maximum number of radiation
lengths that an Energy ray can travel
\item[JJWORK] Actual size of the Energy ray stack
\item[IFOUNP] Number of Energy rays that change cell in $\phi$
direction
\item[IFOUNT] Number of Energy rays that change cell in $\theta$
direction
\item[IFNPOT] Number of Energy rays that change cell either in $\phi$
or in $\theta$
\item[SYMPHI] {\tt .TRUE.} if ${\tt PHIMAX-PHIMIN = 360^{\circ}}$
\item[SYMTEU] {\tt .TRUE.} if ${\tt TETMIN = 0^{\circ}}$
\item[SYMTED] {\tt .TRUE.} if ${\tt TETMAX = 180^{\circ}}$
\end{DLtt}
\FComm{GCPARM}{Control of parametrization}
\begin{verbatim}
      COMMON/GCPARM/IPARAM,PCUTGA,PCUTEL,PCUTNE,PCUTHA,PCUTMU
     +             ,NSPARA,MPSTAK,NPGENE
      REAL PACUTS(5)
      EQUIVALENCE (PACUTS(1),PCUTGA)
      PARAMETER (NWPPAR=14)
      PARAMETER (NWERAY=40)
C
\end{verbatim}
\begin{DLtt}{MMMMMMMMMM}
\item[IPARAM]Parametrization flag ({\tt 0, PCUT})
\begin{DLtt}{MMMMM}
\item[0 =]parametrization is not in effect, normal tracking will be used
\item[1 =]parametrization is in effect
\end{DLtt}
\item[PCUTGA]Parametrization threshold for photons ({\tt 0.,  PCUT})
\item[PCUTEL]Parametrization threshold for electrons and positrons
({\tt 0.,  PCUT})
\item[PCUTNE]Parametrization threshold for neutral hadrons
({\tt 0., PCUT})
\item[PCUTHA]Parametrization threshold for charged hadrons
({\tt 0., PCUT})
\item[PCUTMU]Parametrization threshold for muons
({\tt 0.,  PCUT})
\item[NSPARA] not used
\item[MPSTAK] Optimum size of the Energy ray stack ({\tt 2000})
\item[NPGENE] Number of Energy rays generated per primary particle
({\tt 20})
\item[NWPPAR] Number of words stored for each track to be parametrized
\item[NWERAY] Number of words stored for each Energy-ray
\end{DLtt}
\FComm{GCPHYS}{Control of physics processes}
\begin{verbatim}
   COMMON/GCPHYS/IPAIR, SPAIR, SLPAIR,ZINTPA,STEPPA
  +             ,ICOMP, SCOMP, SLCOMP,ZINTCO,STEPCO
  +             ,IPHOT, SPHOT, SLPHOT,ZINTPH,STEPPH
  +             ,IPFIS, SPFIS, SLPFIS,ZINTPF,STEPPF
  +             ,IDRAY, SDRAY, SLDRAY,ZINTDR,STEPDR
  +             ,IANNI, SANNI, SLANNI,ZINTAN,STEPAN
  +             ,IBREM, SBREM, SLBREM,ZINTBR,STEPBR
  +             ,IHADR, SHADR, SLHADR,ZINTHA,STEPHA
  +             ,IMUNU, SMUNU, SLMUNU,ZINTMU,STEPMU
  +             ,IDCAY, SDCAY, SLIFE ,SUMLIF,DPHYS1
  +             ,ILOSS, SLOSS, SOLOSS,STLOSS,DPHYS2
  +             ,IMULS, SMULS, SOMULS,STMULS,DPHYS3
 
\end{verbatim}
\begin{DLtt}{MMMMMMMMMM}
\item[IPAIR] Control variable for the \Pem/\Pep pair production process.
\item[SPAIR] Distance to the next pair production in the current material.
\item[SLPAIR] Distance travelled by the $\gamma$ when pair production occurs.
\item[ZINTPA] Number of interaction lengths to the next pair production.
\item[STEPPA] Interaction length for pair production for the current material
and energy.
\item[ICOMP] Control variable for the Compton scattering process.
\item[SCOMP] Distance to the next Compton scattering in the current material.
\item[SLCOMP] Distance travelled by the $\gamma$ when Compton scattering occurs.
\item[ZINTCO] Number of interaction lengths to the next Compton scattering.
\item[STEPCO] Interaction length for Compton scattering for the current material
and energy.
\item[IPHOT] Control variable for the photoelectric effect process.
\item[SPHOT] Distance to the next photoelectric effect in the current material.
\item[SLPHOT] Distance travelled by the $\gamma$ when photoelectric effect occurs.
\item[ZINTPH] Number of interaction lengths to the next photoelectric effect.
\item[STEPPH] Interaction length for photoelectric effect for the current material
and energy.
\item[IPFIS] Control variable for the $\gamma$-induced nuclear fission process.
\item[SPFIS] Distance to the next $\gamma$-induced nuclear fission in the current 
material.
\item[SLPFIS] Distance travelled by the $\gamma$ when $\gamma$-induced nuclear 
fission occurs.
\item[ZINTPF] Number of interaction lengths to the next $\gamma$-induced nuclear 
fission.
\item[STEPPF] Interaction length for $\gamma$-induced nuclear fission for the 
current material and energy.
\item[IDRAY] Control variable for the $\delta$-ray production process.
\item[SDRAY] Distance to the next $\delta$-ray production in the current material.
\item[SLDRAY] Distance travelled by the particle when $\delta$-ray production 
occurs.
\item[ZINTDR] Number of interaction lengths to the next $\delta$-ray production.
\item[STEPDR] Interaction length for $\delta$-ray production for the current 
material and energy.
\item[IANNI] Control variable for the positron annichilation process.
\item[SANNI] Distance to the next positron annichilation in the current material.
\item[SLANNI] Distance travelled by the positron when positron annichilation 
occurs.
\item[ZINTAN] Number of interaction lengths to the next positron annichilation.
\item[STEPAN] Interaction length for positron annichilation for the current 
material and energy.
\item[IBREM] Control variable for the Bremstrahlung process.
\item[SBREM] Distance to the next Bremstrahlung in the current material.
\item[SLBREM] Distance travelled by the particle when Bremstrahlung occurs.
\item[ZINTBR] Number of interaction lengths to the next Bremstrahlung.
\item[STEPBR] Interaction length for Bremstrahlung for the current material
and energy.
\item[IHADR] Control variable for the hadronic interaction process.
\item[SHADR] Distance to the next hadronic interaction in the current material.
\item[SLHADR] Distance travelled by the particle when hadronic interaction occurs.
\item[ZINTHA] Number of interaction lengths to the next hadronic interaction.
\item[STEPHA] Interaction length for hadronic interaction for the current material
and energy.
\item[IMUNU] Control variable for the $\mu$ nuclear interaction process.
\item[SMUNU] Distance to the next $\mu$ nuclear interaction in the current 
material.
\item[SLMUNU] Distance travelled by the $\mu$ when $\mu$ nuclear interaction 
occurs.
\item[ZINTMU] Number of interaction lengths to the next $\mu$ nuclear interaction.
\item[STEPMU] Interaction length for $\mu$ nuclear interaction for the current 
material and energy.
\item[IDCAY] Control variable for the decay in flight process.
\item[SDCAY] Distance to the next decay in flight in the current material.
\item[SLIFE] Distance travelled by the particle when decay in flight occurs.
\item[SUMLIF] Time to the next interaction point in $ct$ units.
\item[DPHYS1] Not used.
and energy.
\item[ILOSS] Control variable for the energy loss process.
\item[SLOSS] Step limitation due to the energy loss process.
\item[SOLOSS] Not used.
\item[STLOSS] Not used. Set equal to {\tt STEP} for backward compatibility.
\item[DPHYS2] Not used.
\item[IMULS] Control variable for the energy loss process.
\item[SMULS] Maximum step allowed by the multiple scattering simulation.
\item[SOMULS] Not used.
\item[STMULS] Not used. Set equal to step for backward compatibility.
\item[DPHYS3] Not used.
\end{DLtt}
For more details on {\tt IDRAY} and {\tt ILOSS} see {\tt [BASE040]}.
For all other variables see {\tt [PHYS010]}.
\FComm{GCPOLY}{Internal flags for polygon and polycone shapes}
\begin{verbatim}
      COMMON/GCPOLY/IZSEC,IPSEC
C
\end{verbatim}
\begin{DLtt}{MMMMMMMMMM}
\item[IZSEC]    Z  section number
\item[IPSEC]    $\phi$ sector number
\end{DLtt}
\FComm{GCPUSH}{Initial and incremental size of some mother banks}
\begin{verbatim}
      COMMON/GCPUSH/NCVERT,NCKINE,NCJXYZ,NPVERT,NPKINE,NPJXYZ
C
\end{verbatim}
\begin{DLtt}{MMMMMMMMMM}
\item[NCVERT] Initial size of mother bank {\tt JVERTX } ({\tt 5})
\item[NCKINE] Initial size of mother bank {\tt JKINE}  ({\tt 50})
\item[NCJXYZ] Initial size of mother bank {\tt JXYZ}  ({\tt 50})
\item[NPVERT] Increment for size of mother bank {\tt JVERTX}  ({\tt 5})
\item[NPKINE] Increment for size of mother bank {\tt JKINE}  ({\tt 10})
\item[NPJXYZ] Increment for size of mother bank {\tt JXYZ}  ({\tt 10})
\end{DLtt}
\FComm{GCRZ}{Direct access files control variables}
\begin{verbatim}
      COMMON/GCRZ1/NRECRZ,NRGET,NRSAVE,LRGET(20),LRSAVE(20)
      COMMON/GCRZ2/RZTAGS
      CHARACTER*8 RZTAGS(4)
C
\end{verbatim}
\begin{DLtt}{MMMMMMMMMM}
\item[NRECRZ] Record size (argument of {\tt RZMAKE})
\item[NRGET]  Number of data structures declared on data card {\tt RGET}
\item[NRSAVE] Number of data structures declared on data card {\tt RSAV}
\item[LRGET,LRSAVE] Corresponding user lists of items
\item[RZTAGS]Key names (argument of {\tt RZMAKE})
\end{DLtt}
\FComm{GCSCAL}{Scan geometry ZEBRA pointers}
\begin{verbatim}
      PARAMETER(MXSLNK=100)
      COMMON/GCSCAL/ ISLINK(MXSLNK)
      EQUIVALENCE (LSLAST,ISLINK(MXSLNK))
      EQUIVALENCE (LSCAN ,ISLINK(1)),(LSTEMP,ISLINK(2))
      EQUIVALENCE (LSPARA,ISLINK(3)),(LSERAY,ISLINK(4))
*
\end{verbatim}
\FComm{GCSCAN}{Scan geometry control parameters}
\begin{verbatim}
      PARAMETER (MSLIST=32,MAXMDT=3)
      COMMON/GCSCAN/SCANFL,NPHI,PHIMIN,PHIMAX,NTETA,TETMIN,TETMAX,
     +              MODTET,IPHIMI,IPHIMA,IPHI1,IPHIL,NSLMAX,
     +              NSLIST,ISLIST(MSLIST),VSCAN(3),FACTX0,FACTL,
     +              FACTR,IPHI,ITETA,ISCUR,SX0,SABS,TETMID(MAXMDT),
     +              TETMAD(MAXMDT)
     +             ,SX0S,SX0T,SABSS,SABST,FACTSF
     +             ,DLTPHI,DLTETA,DPHIM1,DTETM1
     +             ,FCX0M1,FCLLM1,FCRRM1
      LOGICAL SCANFL
      COMMON/GCSCAC/SFIN,SFOUT
      CHARACTER*80 SFIN,SFOUT
*
\end{verbatim}
\begin{DLtt}{MMMMMMMMMM}
\item[MSLIST] Dimension of {\tt ISLIST} array ({\tt 32})
\item[MAXMDT] Number of $\theta$ division types ({\tt 3})
\item[SCANFL] SCAN flag ({\tt .FALSE., SCAN, STURN})
\begin{DLtt}{MMMMMMMMMM}
\item[.TRUE.]creation of {\tt SCAN} geometry, geantinos will be tracked
\item[.FALSE.]normal tracking
\end{DLtt}
\item[NPHI] Number of $\phi$ divisions ({\tt 90, SCAN}, {\tt PHI})
\item[PHIMIN] Minimum $\phi$ in degrees (${\tt 0^{\circ}}$,
{\tt SCAN}, {\tt PHI})
\item[PHIMAX] Maximum $\phi$ in degrees (${\tt 360^{\circ}}$,
{\tt SCAN}, {\tt PHI})
\item[NTETA] Number of $\theta$ divisions ({\tt 90}, {\tt SCAN},
{\tt TETA})
\item[TETMIN] Minimum value of $\theta$
(${\tt 0^{\circ}}$, {\tt SCAN}, {\tt TETA})
\item[TETMAX] Maximum value of $\theta$ (18{\tt 0.,  SCAN}, {\tt $\theta$})
\item[MODTET] Type of $\theta$ division (1, {\tt SCAN}, {\tt $\theta$})
\begin{DLtt}{MMMMM}
\item[1 =] $\theta$ is expressed in terms of degrees
\item[2 =] $\theta$ is expressed in terms of pseudorapidity
\item[3 =] $\theta$ is expressed in terms of $\cos(\theta)$
\end{DLtt}
\item[IPHIMI] not used
\item[IPHIMA] not used
\item[IPHI1] internal index ({\tt PHIMIN})
\item[IPHIL] internal index ({\tt PHIMAX})
\item[NSLMAX] not used
\item[NSLIST] Number of volumes to be scanned ({\tt 1}, {\tt SCAL})
\item[ISLIST] List of volumes to be scanned ({\tt SCAL}, {\tt SLIST})
\item[VSCAN] Scan vertex origin ({\tt SCAP}, {\tt VERTEX})
\item[FACTX0] Scale factor for {\tt SX0} ({\tt 100.}, {\tt SCAP},
{\tt SFACTORS})
\item[FACTL] Scale factor for {\tt SABS} ({\tt 10.}, {\tt SCAP},
{\tt SFACTORS})
\item[FACTR] Scale factor for {\tt R} ({\tt 100.},
{\tt SCAP}, {\tt SFACTORS})
\item[IPHI]  $\phi$ bin of the current cell
\item[ITETA] $\theta$ bin of the current cell
\item[ISCUR] Pointer in {\tt LPHI} to first triplet of words for a
given {\tt ITETA} cell
\item[SX0] Sum of radiation lengths up to current {\tt R} boundary
\item[SABS] Sum of absorbtion lengths up to current {\tt R} boundary
\item[TETMID] Bound value for {\tt TETMIN} ({\tt 0., -10., -1.} if
{\tt MODTET} is 1, 2 or 3 respectively)
\item[TETMAD] Bound value for {\tt TETMAX} ({\tt 180., 10., 1.} if
{\tt MODTET} is 1, 2 or 3 respectively)
\item[SX0S] Sum of radiation lengths for the sensitive mediums in the
current cell
\item[SX0T] Sum of radiation lengths in the current cell
\item[SABSS] Sum of absorption lengths for the sensitive mediums in
the current cell
\item[SABST] Sum of absorbtion lengths in the current cell
\item[FACTSF] Scale factor for the sampling fractions ({\tt 1000.})
\item[DLTPHI] Bin in $\phi$, ({\tt PHIMAX-PHIMIN)/NPHI}
\item[DLTETA] Bin in $\theta$, ({\tt TETMAX-TETMIN)/NTETA}
\item[DPHIM1] ${\tt DLTPHI^{-1}}$
\item[DTETM1] ${\tt DLTETA^{-1}}$
\item[FCX0M1] ${\tt FACTX0^{-1}}$
\item[FCLLM1] ${\tt FACTL^{-1}}$
\item[FCRRM1] ${\tt FACTR^{-1}}$
\item[SFIN] not used
\item[SFOUT] not used
\end{DLtt}
\FComm{GSECTI}{Hadronic partial cross sections}
\begin{verbatim}
      COMMON/GSECTI/ AIEL(20),AIIN(20),AIFI(20),AICA(20),ALAM,K0FLAG
C
\end{verbatim}
\begin{DLtt}{MMMMMMMMMM}
\item[AIEL]Elastic cross sections. {\tt AIEL(I)} is the elastic cross section
for the ${\tt I}^{th}$ element composing the current material
\item[AIIN]Inelastic cross sections
\item[AIFI]Fission cross sections
\item[AICA]Nuclear capture cross sections
\item[ALAM]Total cross section
\item[K0FLAG]Obsolete
\end{DLtt}
\FComm{GCSETS}{Identification of current sensitive detector}
\begin{verbatim}
      COMMON/GCSETS/IHSET,IHDET,ISET,IDET,IDTYPE,NVNAME,NUMBV(20)
C
\end{verbatim}
\begin{DLtt}{MMMMMMMMMM}
\item[IHSET]   Set identifier. ASCII equivalent of 4 characters.
\item[IHDET]   Detector identifier. ASCII equivalent of 4 characters.
\item[ISET]    Position of set in bank {\tt JSET}
\item[IDET]    Position of detector in bank {\tt JS=LQ(JSET-ISET)}
\item[IDTYPE]  User defined detector type
\item[NVNAME]  Number of elements in {\tt NUMBV}
\item[NUMBV]   List of volume copy numbers to identify the detector
\end{DLtt}
\FComm{GCSHNO}{Symbolic codes for system shapes}
\begin{verbatim}
      PARAMETER ( NSBOX=1,  NSTRD1=2, NSTRD2=3, NSTRAP=4, NSTUBE=5,
     +  NSTUBS=6, NSCONE=7, NSCONS=8, NSSPHE=9, NSPARA=10,NSPGON=11,
     +  NSPCON=12,NSELTU=13,NSHYPE=14,NSGTRA=28, NSCTUB=29 )
\end{verbatim}
\FComm{GCSPEE}{Auxiliary variables for the CG package}
\begin{verbatim}
      COMMON/GCSPEE/S1,S2,S3,SS1,SS2,SS3,LEP,IPORLI,ISUBLI,
     +              SRAGMX,SRAGMN,RAINT1,RAINT2,RMIN1,RMIN2,
     +              RMAX1,RMAX2,PORJJJ,ITSTCU,IOLDCU,ISCOP,
     +              NTIM,NTFLAG,LPASS
*
\end{verbatim}
\begin{DLtt}{MMMMMMMMMM}
\item[S1]
\item[S2]
\item[S3]
\item[SS1]
\item[SS2]
\item[SS3]
\item[LEP]
\item[IPORLI]
\item[ISUBLI]
\item[SRAGMX]
\item[SRAGMN]
\item[RAINT1]
\item[RAINT2]
\item[RMIN1]
\item[RMIN2]
\item[RMAX1]
\item[RMAX2]
\item[PORJJJ]
\item[ITSTCU]
\item[IOLDCU]
\item[ISCOP]
\item[NTIM]
\item[NTFLAG]
\item[LPASS]
\end{DLtt}
\FComm{GCSTAK}{Control variables for parallel tracking}
\begin{verbatim}
      PARAMETER (NWSTAK=12,NWINT=11,NWREAL=12,NWTRAC=NWINT+NWREAL+5)
      COMMON /GCSTAK/ NJTMAX, NJTMIN, NTSTKP, NTSTKS, NDBOOK, NDPUSH,
     +                NJFREE, NJGARB, NJINVO, LINSAV(15), LMXSAV(15)
C
\end{verbatim}
\begin{DLtt}{MMMMMMMMMM}
\item[NWSTAK]
\item[NWINT]
\item[NWREAL]
\item[NWTRAC]
\item[NJTMAX]
\item[NJTMIN]
\item[NTSTKP]
\item[NTSTKS]
\item[NDBOOK]
\item[NDPUSH]
\item[NJFREE]
\item[NJGARB]
\item[NJINVO]
\item[LINSAV]
\item[LMXSAV]
\end{DLtt}
\FComm{GCTIME}{Execution time control}
\begin{verbatim}
      COMMON/GCTIME/TIMINT,TIMEND,ITIME,IGDATE,IGTIME
C
\end{verbatim}
\begin{DLtt}{MMMMMMMMMM}
\item[TIMINT] Total time left after initialization  ({\tt TIME})
\item[TIMEND] Time requested
for program termination phase ({\tt 1, TIME})
\item[ITIME] Number of events between two tests of time left
({\tt 1, TIME})
\item[IGDATE]Current date in integer format {\tt YYMMDD}
\item[IGTIME] Current time in integer format {\tt HHMM}
\end{DLtt}
\FComm{GCTMED}{Array of current tracking medium parameters}
\begin{verbatim}
      COMMON/GCTMED/NUMED,NATMED(5),ISVOL,IFIELD,FIELDM,TMAXFD,STEMAX
     +      ,DEEMAX,EPSIL,STMIN,CFIELD,PREC,IUPD,ISTPAR,NUMOLD
C
\end{verbatim}
\begin{DLtt}{MMMMMMMMMM}
\item[NUMED]  Current tracking medium number
\item[NATMED] Name of current tracking medium (ASCII codes stored in an integer
array, 4 characthers per word)
\item[ISVOL]
\begin{DLtt}{MMMMM}
\item[-1 =] Non-sensitive volume with sensitive volume tracking parameters
\item[~0 =] Non-sensitive volume
\item[~1 =] Sensitive volume
\end{DLtt}
\item[IFIELD]
\begin{DLtt}{MMMMM}
\item[0 =] No field
\item[1 =] User defined field (\Rind{GUFLD})
\item[2 =] User defined field (\Rind{GUFLD}) along z
\item[3 =] Uniform field ({\tt FIELDM}) along z
\end{DLtt}
\item[FIELDM] Maximum field
\item[TMAXFD] Maximum turning angle in one step due to the magnetic
field
\item[STEMAX] Maximum step allowed
\item[DEEMAX] Maximum fraction of energy loss in one step for ionization
\item[EPSIL] Boundary crossing accuracy
\item[STMIN] Minimum step size by energy loss or by multiple scattering
\item[CFIELD]Constant for field step evaluation
\item[CMULS]Effective step for boundary crossing ($0.1 \times {\tt EPSIL}$)
\item[IUPD]
\begin{DLtt}{MMMMM}
\item[0 =] New particle or new medium in current step
\item[1 =] No change of medium or particle
\end{DLtt}
\item[ISTPAR]
\begin{DLtt}{MMMMM}
\item[0 =] Global tracking parameters are used
\item[1 =] Special tracking parameters are used for this medium
\end{DLtt}
\item[NUMOLD] Number of the previous tracking medium
\end{DLtt}
\FComm{GCTRAK}{Track parameters at the end of the current step}
\begin{verbatim}
      PARAMETER (MAXMEC=30)
      COMMON/GCTRAK/VECT(7),GETOT,GEKIN,VOUT(7),NMEC,LMEC(MAXMEC)
     + ,NAMEC(MAXMEC),NSTEP ,MAXNST,DESTEP,DESTEL,SAFETY,SLENG
     + ,STEP  ,SNEXT ,SFIELD,TOFG  ,GEKRAT,UPWGHT,IGNEXT,INWVOL
     + ,ISTOP ,IGAUTO,IEKBIN, ILOSL, IMULL,INGOTO,NLDOWN,NLEVIN
     + ,NLVSAV,ISTORY
C
\end{verbatim}
\begin{DLtt}{MMMMMMMMMM}
\item[VECT] Current track parameters ($\rm x,y,z,p_x/p,p_y/p,p_z/p,p$)
\item[GETOT]Current particle total energy
\item[GEKIN]Current particle kinetic energy
\item[VOUT]Track parameters at the end of the step. Used internally by
GEANT.
\item[NMEC]Number of mechanisms active for current step
\item[LMEC]List of mechanism indices for current step
\item[NAMEC]List of mechanism names for current step
(ASCII codes stored in an integer, 4 characthers per word)
\item[NSTEP]Number of steps for current track
\item[MAXNST]Maximum number of steps allowed (default = 10000)
\item[DESTEP]Total energy lost in current step
\item[DESTEL]Same as {\tt DESTEP}. Kept for backward compatibility.
\item[SAFETY]Underestimated distance to closest medium boundary
\item[SLENG]Track length at current point
\item[STEP] Size of curent tracking step
\item[SNEXT]Distance to current medium boundary along the direction of
the particle
\item[SFIELD]Obsolete.
\item[TOFG]Current time of flight in $ct$ units.
\item[GEKRAT]Interpolation coefficient in the energy table {\tt ELOW}
\item[UPWGHT]User word for current particle
\item[IGNEXT]
\begin{DLtt}{MMMMM}
\item[0 =]{\tt SNEXT} has not been computed in current step
\item[1 =]{\tt SNEXT} has been computed in current step
\end{DLtt}
\item[INWVOL]
\begin{DLtt}{MMMMM}
\item[0 =]track is inside a volume
\item[1 =]track has entered a new volume or at the beginning of a new track
\item[2 =]track is exiting current volume
\item[3 =]track is exiting the setup
\end{DLtt}
\item[ISTOP]
\begin{DLtt}{MMMMM}
\item[0 =]particle will continue to be tracked
\item[1 =]particle has disappeared (decay, inelastic interaction \dots)
\item[2 =]particle has fallen below the cutoff energy or has interacted but
no secondaries have been generated.
\end{DLtt}
\item[IGAUTO]
\begin{DLtt}{MMMMM}
\item[0 =]tracking parameters are given by the user
\item[1 =]tracking parameters are calculated by {\tt GEANT}
\end{DLtt}
\item[IEKBIN]Current kinetic energy bin in table {\tt ELOW}
\item[ILOSL]Local energy loss flag (see \FCind{/GCPHYS/})
\item[IMULL]Local multiple scattering flag (see \FCind{/GCPHYS/})
\item[INGOTO]Volume which the particle will enter if continuing along
a straight line for {\tt SNEXT} centimeters
\item[NLDOWN]Lowest level reached down the tree (parallel tracking only)
\item[NLEVIN]Number of levels currently filled and valid in
             \FCind{/GCVOLU/}
\item[NLVSAV]Current level (parallel tracking only)
\item[ISTORY]User flag for current track history (reset to $0$ in
             \Rind{GLTRAC})
\end{DLtt}
List of mechanisms considered at tracking time:
\begin{verbatim}
      DATA MEC/'NEXT','MULS','LOSS','FIEL','DCAY','PAIR','COMP','PHOT'
     +        ,'BREM','DRAY','ANNI','HADR','ECOH','EVAP','FISS','ABSO'
     +        ,'ANNH','CAPT','EINC','INHE','MUNU','TOFM','PFIS','SCUT'
     +        ,'RAYL','PARA','PRED','LOOP','NULL','STOP'/
\end{verbatim}
 
\FComm{GCUNIT}{Description of logical units' }
\begin{verbatim}
   COMMON/GCUNIT/LIN, LOUT, NUNITS, LUNITS(5)
   COMMON/GCMAIL/CHMAIL
   CHARACTER 132 CHMAIL
\end{verbatim}
\begin{DLtt}{MMMMMMMMMM}
\item[LIN]Input unit to read data records
\item[LOUT]Line printer output unit
\item[NUNITS]Number of additional units
\item[LUNITS]List of additional units
\item[CHMAIL]Character string containing the message to be printed by
             \Rind{GMAIL}
\end{DLtt}
 
{\tt LIN} and {\tt LOUT} are defined in \Rind{GINIT} through {\tt ZEBRA}.
{\tt NUNITS} and {\tt LUNITS} are reserved
for user-declared {\tt ZEBRA} files.
\FComm{GCVOLU}{Multi-level current volume description}
\begin{verbatim}
      COMMON/GCVOLU/NLEVEL,NAMES(15),NUMBER(15),
     +LVOLUM(15),LINDEX(15),INFROM,NLEVMX,NLDEV(15),LINMX(15),
     +GTRAN(3,15),GRMAT(10,15),GONLY(15),GLX(3)
C
\end{verbatim}
\begin{DLtt}{MMMMMMMMMM}
\item[NLEVEL] Level at which the last search stopped.
\item[NAMES]Volume names at each level.
(ASCII codes stored in an integer, 4 characthers per word)
\item[NUMBER]Volume copy numbers at each level.
\item[LVOLUM]System volume numbers at each level.
\item[LINDEX]Physical tree volume indices at each level.
\item[INFROM]
\item[NLEVMX]
\item[NLDEV]
\item[LINMX]
\item[GTRAN]x,y,z offsets of the cumulative coordinate
transformation from the master system to the system at each level.
\item[GRMAT]Rotation matrix elements for the cumulative
transformation from the master system to the system at each level.
${\tt GRMAT(10,LEVEL)}=0$ indicates the null rotation.
\item[GONLY] Uniqueness flags at each level.
\item[GLX]Current point in local coordinates system (local use only!)
\end{DLtt}
\FComm{GCVOL2}{Back-up for \FCind{/GCVOLU/}}
\FComm{GCXLUN}{Logical units number for the interactive version}
\begin{verbatim}
      COMMON/GCXLUN/LUNIT(128)
*
\end{verbatim}
\begin{DLtt}{MMMMMMMMMM}
\item[LUNIT]Logical units numbers
\end{DLtt}
