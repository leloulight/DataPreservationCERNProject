\chapter{Installing ready-built libraries and modules}

This chapter describes how to retrieve and install ready-built libraries
and modules from the asis server at CERN. More details on asis
can be found in section \ref{sect-ASIS}.

\section{Retrieving tar files for Unix systems}

Compressed {\it tar} files are kept in the directory 
\index{tar}
cernlib/@sys/pro/tar, where {\bf @sys} should be replaced
by the Transarc name for your system, e.g. {\bf rs\_aix32} for
the RS6000 running AIX 3.2. Examples of the Transarc
naming convention are given in table \ref{table-@SYS} on
page \pageref{table-@SYS}.
This directory contains a number
of files with the extension {\bf .contents}, each of which
describes the contents of the corresponding file with
extension {\bf .tar.gz}.

{\bf N.B. the} {\it tar} {\bf files are only created when a release
is made (typically a few days after the release). If you wish to
install a version of the CERN library that has {\it not yet been
released} please follow the instructions given in \ref{sect-NEW}}.

\begin{DLtt}{12345678901234567890}
\item[cernbin.contents]The binaries from the {\bf bin} directory, e.g. {\bf pawX11}.
\item[cernglib.contents]The graphics libraries, e.g. libgrafGKS.a, libgrafX11.a, libgraflib.a,
libpawlib.a
\item[cernlib.contents]Libraries such as libkernlib.a, libpacklib.a, libmathlib.a and libphtools.a.
\item[cernmgr.contents]The CERN {\bf mgr} tree, required if you wish to reinstall all or part
of CERNLIB locally.
\item[cernsrc.contents]The contents of the {\bf src} directory, i.e. the {\bf .car} 
and {\bf .cra} files.
\item[cmz.contents]The {\bf CMZ} distribution kit.
\item[gcalor.contents]Various cross section files used by GEANT.
\item[geant315.contents]The GEANT 3.15 distribution.
\item[mclibs.contents]The Monte-Carlo libraries, e.g. JETSET, PHOTOS etc.
\item[patchy.contents]The PATCHY distribution kit.
\end{DLtt}

\subsection{Retrieving the complete distribution}

Procede as follows:

\begin{enumerate}
\item
Connect to asisftp.cern.ch via {\bf ftp}.
Use username {\bf anonymous}, password {\bf your e-mail address}, e.g. jamie@zfatal.cern.ch.
\item
Go to the appropriate directory for your machine type and the release that
you wish to retrieve, e.g. {\bf cd cernlib/hp700\_ux90/94a/tar}. You will see
\index{tar}
a message like the following:

\begin{XMPt}{Welcome message}

ftp> cd 93d/tar
250-
250- This directory contains compressed files of CERNlib release 93d for HP/UX 9.0
250-
250- Files ending in .tar.gz have been compressed using gzip. gzip/gunzip for
250- HP/UX are available in this directory in the gzip.tar file. Get this first
250- and untar in a directory in the search path. Also take a new copy of 
250- plitar; this will use gzip -d ( equivalent to gunzip ) to uncompress files
250- ending in .tar.gz.
250- 
250-
250-Please read the file README
250-  it was last modified on Thu Nov  4 18:16:40 1993 - 83 days ago
250 CWD command successful.
ftp> 

\end{XMPt}
\item
Follow the instructions given in the welcome message, e.g.
{\bf retrieve and read any README files}.
\item
Go to {\bf binary} mode and retrieve all compressed tar files.
\index{tar}
\item
Unpack using \plitar{}.
\end{enumerate}

\subsection{Retrieving the source files}

The complete sources can be obtained by retrieving and unpacking
only the cernsrc.tar.gz file. Assuming that you wish to retrieve
individual sources, procede as follows:

\begin{enumerate}
\item
Connect to asisftp.cern.ch using anonymous \ftp{} as shown
above.
\item
Change directory to {\bf cernlib/share/94a/src/car} (or the corresponding
directory for the appropriate release).
\item
Retrieve the {\bf .car} files of interest, e.g.
{\bf paw.car}, {\bf hbook.car}, etc.
\end{enumerate}

\subsection{Retrieving the libraries}

One may obtain all of the libraries or individual sets as 
shown above. To obtain individual libraries only, procede as
follows:

\begin{enumerate}
\item
Connect to asisftp.cern.ch using anonymous \ftp{} as shown
above.
\item
Change directory to {\bf cernlib/@sys/94a/lib} (or the corresponding
directory for the appropriate release).
\item
Switch to {\bf binary} mode.
\item
Retrieve the {\bf .a} files of interest, e.g.
{\bf libpacklib.a}, {\bf libpawlib.a}.
\end{enumerate}

\subsection{Retrieving the binaries}

One may obtain all of the binaries as shown above.
shown above. To obtain individual binaries only, procede as
follows:

\begin{enumerate}
\item
Connect to asisftp.cern.ch using anonymous \ftp{} as shown
above.
\item
Change directory to {\bf cernlib/@sys/94a/bin} (or the corresponding
directory for the appropriate release).
\item
Switch to {\bf binary} mode.
\item
Retrieve the files of interest, e.g.
{\bf pawX11}, {\bf kxterm}, etc.
\end{enumerate}

\section{Retrieving backup savesets for VMS systems}

The backup savesets for VMS systems are currently compressed
using the {\tt gzip} program. If you have access to this 
program, it is more efficient in terms of network load
and CPU load on the {\tt asisftp.cern.ch} server to transfer
the compressed file and uncompress it on your machine.

If you do {\bf not} have access to this program, simply
leave of the {\tt .gz} suffix when performed the transfer
and the files will be uncompressed on the fly.

\subsection{GZIP/GUNZIP}

\index{GZIP}
\index{GUNZIP}

The executables for GZIP and GUNZIP may be found in
VXCERN::USR:[LOCAL.EXE] (VAX/VMS) and 
VXCERN::USR:[LOCALAXP.EXE] (OpenVMS).

Unlike Unix systems, which add an extension .gz when compressing
a file, -GZ is appended to the file name.

\begin{XMPt}{Running GZIP or GUNZIP on VMS systems}
AXCRNC? gzip -v login.com

DISK$CD:[JAMIE]LOGIN.COM;1:	61.8% -- replaced with DISK$CD:[JAMIE]LOGIN.COM-gz

AXCRNC? gunzip -v login.com

DISK$CD:[JAMIE]LOGIN.COM-gz:	61.8% -- replaced with DISK$CD:[JAMIE]LOGIN.COM
\end{XMPt}

The following savesets were built for release 94A:

\begin{DLtt}{12345678901234567890}
\item[crncmz.bck.gz]CMZ
\item[crnlib.bck.gz]The .OLBs (KERNLIB, MATHLIB, PACKLIB etc.) and .EXEs.
\item[crnpat.bck.gz]The PATCHY executables.
\item[crnsrc.bck.gz]The source files.
\end{DLtt}

Post 94A releases will be made in the same way as for Unix systems, i.e.

\begin{DLtt}{12345678901234567890}
\item[cernbin.tar.gz]The modules (binaries)
\item[cernglib.tar.gz]The graphics libraries      
\item[cernlib.tar.gz]The non-graphics libraries       
\item[cernmgr.tar.gz]The installation scripts
\item[cernsrc.tar.gz]The source files (.CAR and .CRA)
\item[cmz.tar.gz]CMZ
\item[geant321-src.tar.gz]GEANT 3.21 source
\item[geant321.tar.gz]GEANT libraries etc.
\item[mclibs.tar.gz]The Monte Carlo libraries
\item[patchy.tar.gz]The PATCHY modules
\end{DLtt}

{\bf N.B. Having retrieved (and uncompressed) the savesets,
the correct block size must be set using the RESIZE command,
available from VXCERN::CERN:[PRO.EXE]RESIZE.COM}.
\index{resize}

\begin{XMPt}{Setting the correct blocksize}

AXCRNC? resize -s 32256 crnlib.bck
resize: setting record size of CRNLIB.BCK to 32256 bytes...
AXCRNC?

\end{XMPt}

\index{SET FILE/ATTRIBUTES}
On systems running VMS 6.1 or higher, the resize command
should be replaced by the {\bf SET FILE/ATTRIBUTES} command,
e.g.

{\underline SET FILE/ATTRIBUTES=LRL=32256 CRNLIB.BCK}

\index{MULTINET}

With version 3.3 (February, 1994) of TGV Multinet TCP/IP software for OpenVMS,
it is possible to set the record size for binary transfers in FTP (previously
the only allowed sizes were 512 and 2048 bytes).  The commands are
RECORD-SIZE nnnnn when the ftp session originates on the openVMS computer, and
QUOTE SITE RMS RECSIZE nnnnn when ftp is started from the non-VMS end.
Here are examples of the two cases.  

\begin{XMPt}{For a transfer starting on an OpenVMS computer}

$ FTP SHIFT.CERN.CH
SHIFT.CERN.CH> USER yourname
<Password required for yourname.
Password:
<User leeiv logged in.
SHIFT.CERN.CH> VERSION
DUKPHY.PHY.DUKE.EDU MultiNet FTP user process 3.3(109)
SHIFT.CERN.CH> BINARY
Type: Image, Structure: File, Mode: Stream
SHIFT.CERN.CH> RECORD-SIZE 32400
SHIFT.CERN.CH> RECORD-SIZE
Record size for IMAGE files: 32400
SHIFT.CERN.CH> GET myfile.rzhist myfile.rzhist
SHIFT.CERN.CH> QUIT
$

\end{XMPt}

\begin{XMPt}{For a transfer starting from a non-Multinet computer}

> ftp dukphy.phy.duke.edu
Connected to dukphy.phy.duke.edu.
220 DUKPHY.PHY.DUKE.EDU MultiNet FTP Server Process 3.3(14) at Mon 13-Jun-94
Name: yourname
331 User name (yourname) ok. Password, please.
Password:
ftp> binary
200 Type I ok.
ftp> quote site rms recsize 32400
200 IMAGE file record size now 32400 bytes
ftp> put myfile.rzhist myfile.rzhist
ftp> quit

\end{XMPt}

\subsection{Unpacking the BACKUP savesets}

Having uncompressed and "resized" the BACKUP savesets, use
the BACKUP command to unpack the files into the appropriate
directories on your system.

\begin{XMPt}{Unpacking the BACKUP savesets}

back/log disk$scratch:[pubmf.work.jamie]crnexe.bck/save disk$cernlib:[*...]
%BACKUP-S-CREDIR, created directory DISK$CD:[PRO.EXE]
%BACKUP-S-CREATED, created DISK$CD:[PRO.EXE]ARIADNE.EXE;12
%BACKUP-S-CREATED, created DISK$CD:[PRO.EXE]ASTUCE.EXE;53

...

\end{XMPt}


\section{Retrieving tar files for VM/CMS systems}

\chapter{Retrieving individual source files}
\label{sect-NEW}

As the compressed tar files are only produced following a release
of the complete libraries, it may be necessary to retrieve individual
source files, e.g. if one wishes to install a new version of
a library or modules. In this case, one should transfer the
appropriate {\bf .car} and {\bf .cra} files. 

{\bf N.B. please also check ensure that you copy the latest versions
of any files in the} {\it mgr} tree.

\begin{XMPt}{Retrieving individual source files}

zfatal:/home/cp/jamie/hepdb (10) cd /tmp
zfatal:/tmp (11) ftp asisftp.cern.ch
Connected to asisftp.cern.ch.
220 asisftp FTP server (Version 2.0WU(14) Fri Sep 17 15:39:37 MET DST 1993) ready
.
Name (asisftp.cern.ch:jamie): anonymous
331 Guest login ok, send your complete e-mail address as password.
Password:
230-                   ________________________________________
230-                   Application Software Installation Server
230-                   ________________________________________
230-
230-   Welcome to the ASIS ftp server, developed by the CERN Computing and
230-   Networking Division to serve the High Energy Physics research community.
230-
230-   ftp clients may abort due to improper handling of such introductory
230-   messages. A dash (-) as the first character of your pw will suppress it.
230-
230-   The CERNlib software, located in the "cernlib" directory, is covered by
230-   CERN copyright. Before taking any material from this directory, please
230-   read the copyright notice "cernlib/copyright".
230-
230-   Please contact cernlib@cernvm.cern.ch for site registration. General
230-   support questions should be addressed to asis-support@asis01.cern.ch. 
230-
230 Guest login ok, access restrictions apply.
ftp> 
ftp> cd cernlib/share/94a/src/car
250----------------------------------------------------------------------------
250-     CERNlib release 94a:    scheduled date      March 1994
250- 
250-     directory /cern/94a/src/car     containing Patchy sources *.car
250-                                                Patchy cradles *.cra
250- 
250-                                     compressed Patchy sources *.zcar
250----------------------------------------------------------------------------
250-
250 CWD command successful.
ftp> 
ftp> dir paw.car
200 PORT command successful.
150 Opening ASCII mode data connection for /bin/ls.
-rw-r--r--   1 cernlib  software  4527809 Feb  1 10:47 paw.car
226 Transfer complete.
ftp> get paw.car
200 PORT command successful.
150 Opening ASCII mode data connection for paw.car (4527809 bytes).
226 Transfer complete.
4648428 bytes received in 44.68 seconds (101.6 Kbytes/s)
ftp> 

\end{XMPt}

{\bf N.B. as many of the CERN library packages depend on each other,
you may require new versions of other packages as well.}

\chapter{Overview of the installation procedures}
The CERNLIB installation procedures perform the following steps:
\begin{itemize}
\item
Extract the appropriate source code from the master file,
e.g. generate the VMS or Unix specific version.
\item
Split the extracted code into separate files (typically on Unix machines).
\item
Compile the code
\item
Generate the libraries
\end{itemize}

In some cases, additional steps are required, e.g. processing
of the {\bf KUIP} {\it Command Definition File} by the KUIP
compiler, or the assembly of individual components into a 
larger library (e.g. PACKLIB is composed of 
\HBOOK{}~\cite{bib-HBOOK}, KUIP~\cite{bib-KUIP}, ZEBRA~\cite{bib-ZEBRA} etc.).
Currently, source code is typically extracted using the program
{\bf YPATCHY}. If splitting is required, this is performed by
{\bf FCASPLIT}. The basic functionality of these two programs
is described below.
\section{YPATCHY}
The complete functionality of PATCHY is described in the
PATCHY reference manual~\cite{bib-PATCHY}. An introductory
guide is given in the Patchy for beginners guide~\cite{bib-SIMPATCH}.

Rather than attempt to describe all of the features of 
PATCHY here, we will take the
specific example of the \HBOOK{} package.

To install \HBOOK{}, two files are required. These are the \HBOOK{}
source file, typically kept in {\tt /cern/pro/src/car/hbook.car}
on Unix systems at CERN, and the so-called {\it cradle}.

The cradle for \HBOOK{}, in {\tt /cern/pro/src/car/hbook.cra},
is as follows.

\begin{XMPt}{Cradle for the installation of \HBOOK{}}

+EXE.
+USE,*HBOOK,$PLINAME.
+ASM,24.
+USE,QXNO_SC ,T=I,IF=QX_SC.
+USE,QX_SC   ,T=I,IF=QXNO_SC.
+USE,QXCAPT  ,T=I,IF=QXNO_SC,QX_SC.
+PAM,11,T=C,A.$CERN_ROOT/src/car/hbook
+QUIT.

\end{XMPt}

Other cradles may be more complicated, but this will help describe
the basic ideas. Let us examine each line of this cradle in turn.

\begin{enumerate}
\item
+EXE.

This tells YPATCHY to write out all 'material', typically source code,
that has been selected. The material is written to 
the so-called Assembled Material file, or {\bf ASM} file for short.
Different streams exist for various types of material, as described
below. They are all initially connected to the default stream 
on unit 21.

\item
+USE,*HBOOK,\$PLINAME.

This, and other +USE statements, select the material of interest.
Multi-level selection is possible. For example, *HBOOK will trigger
all of the things in the hbook.car file in the PATCH *HBOOK.

\$PLINAME is set by the installation procedures. Typically, it is
the machine type, e.g. DECS for DECstation, IBMRT for RS6000 etc.

By convention, an asterix is used to indicate a so-called {\bf pilot}
patch, which will contain other {\bf +USE} statements. The flags 
selected by \$PLINAME are normally used to select machine specific
features, as shown below.

\begin{XMPt}{Flagging machine specific features using PATCHY}

+SELF,IF=IBMRT.
*
*     RS 6000 specific code
*
+SELF.

\end{XMPt}

The same effect can be achieved using the C preprocessor,
as is shown below.

\begin{XMPt}{Flagging machine specific features with C preprocessor statements}

#ifdef IBMRT

/* RS 6000 specific code
 */

#endif /* IBMRT */

\end{XMPt}

\item
+ASM,24.

This tells PATCHY to establish a new output stream. By default,
all material will be written to the stream 21, which is automatically
initialised and does not require a +ASM directive to establish it.
There are rules/conventions as to which streams are used for what:

\begin{DLtt}{12}
\item[21]Fortran
\item[22]assembler
\item[23]data
\item[24]c

\item[31]diverted Fortran
\end{DLtt}
\ldots

The diverted streams are useful when different compilation options are
required, e.g. static or noopt etc.

\item
The next 3 +USE statements are to select the right sort
of external names, typically for Fortran called C routines.

The convention adopted is

\begin{DLtt}{1234567890}
\item[QX\_SC]external names are postfixed with an underscore, e.g. {\bf hbook1\_}.
Most Unix systems append an underscore to external names in Fortran
routines. On some systems, such as HP/UX and IBM RS6000, one 
must explicitly request this at compile time~\footnote{The options 
for these two systems are {\bf +ppu} and {\bf -qextname} respectively.}.
The trailing underscore is typically used to avoid name clashes
between C and Fortran run-time libraries.
\item[QX\_NOSC]no underscore
\item[QX\_CAPT]no underscore and uppercase
\end{DLtt}

\begin{XMPt}{Examples of styles of external names}

+SELF,IF=QXCAPT.
int CDHSTC(hnf)
+SELF,IF=QXNO_SC.
int cdhstc(hnf)
+SELF,IF=QX_SC.
int cdhstc_(hnf)
+SELF.
char *hnf;

\end{XMPt}

\item
+PAM,11,T=C,A.\$CERN\_ROOT/src/car/hbook

This directive tells PATCHY to read the 'card' format file which 
contains the \HBOOK{} source.

\item
+QUIT.

All done.

\end{enumerate}

\chapter{Installing PATCHY}

Prior to installing the CERNLIB software from scratch, you must install
{\bf PATCHY}. You may obtain the {\bf PATCHY} installation kit from
as shown below.

\section{Unix systems}

\label{sect-UNIXPATCHY}

\subsection{Retrieving the binaries from asisftp.cern.ch}

In most cases, the {\bf PATCHY} binaries can be retrieved
from asisftp.cern.ch as shown below. It may be necessary
to rebuild the modules if there are compiler/shared library
incompatibilities etc.

\begin{XMPt}{Retrieving the modules from asis}

zfatal:/home/cp/jamie (4) ftp asisftp.cern.ch
Connected to asisftp.cern.ch.
220 asisftp FTP server (Version 2.0WU(14) Fri Sep 17 15:39:37 MET DST 1993) ready
.
Name (asisftp.cern.ch:jamie): anonymous
331 Guest login ok, send your complete e-mail address as password.
Password:
230-                   ________________________________________
230-                   Application Software Installation Server
230-                   ________________________________________
230-
230-   Welcome to the ASIS ftp server, developed by the CERN Computing and
230-   Networking Division to serve the High Energy Physics research community.
230-
230-   ftp clients may abort due to improper handling of such introductory
230-   messages. A dash (-) as the first character of your pw will suppress it.
230-
230-   The CERNlib software, located in the "cernlib" directory, is covered by
230-   CERN copyright. Before taking any material from this directory, please
230-   read the copyright notice "cernlib/copyright".
230-
230-   Please contact cernlib@cernvm.cern.ch for site registration. General
230-   support questions should be addressed to asis-support@asis01.cern.ch. 
230-
230 Guest login ok, access restrictions apply.
ftp> cd cernlib/rs_aix32/patchy/4.15/bin
250 CWD command successful.
ftp> ls
200 PORT command successful.
150 Opening ASCII mode data connection for file list.
fcasplit
ycompar
yedit
yfrceta
yindex
yindexb
ylist
ylistb
ypatchy
ysearch
yshift
ytobcd
ytobin
ytoceta
226 Transfer complete.
ftp> 
ftp> 
Local directory now /home/cp/jamie/patchy/bin
ftp> bin
200 Type set to I.
ftp> prompt off
Interactive mode off.
ftp> mget *
200 PORT command successful.
150 Opening BINARY mode data connection for fcasplit (20539 bytes).
226 Transfer complete.
20539 bytes received in 0.8578 seconds (23.38 Kbytes/s)
200 PORT command successful.
150 Opening BINARY mode data connection for ycompar (73414 bytes).
226 Transfer complete.
73414 bytes received in 1.1 seconds (65.17 Kbytes/s)
200 PORT command successful.
150 Opening BINARY mode data connection for yedit (102039 bytes).
226 Transfer complete.
102039 bytes received in 0.6132 seconds (162.5 Kbytes/s)
200 PORT command successful.
150 Opening BINARY mode data connection for yfrceta (96265 bytes).
226 Transfer complete.
96265 bytes received in 0.7846 seconds (119.8 Kbytes/s)
200 PORT command successful.
150 Opening BINARY mode data connection for yindex (1299 bytes).
226 Transfer complete.
1299 bytes received in 0.05103 seconds (24.86 Kbytes/s)
200 PORT command successful.
150 Opening BINARY mode data connection for yindexb (79914 bytes).
226 Transfer complete.
79914 bytes received in 0.9767 seconds (79.91 Kbytes/s)
200 PORT command successful.
150 Opening BINARY mode data connection for ylist (1294 bytes).
226 Transfer complete.
1294 bytes received in 0.02697 seconds (46.85 Kbytes/s)
200 PORT command successful.
150 Opening BINARY mode data connection for ylistb (78498 bytes).
226 Transfer complete.
78498 bytes received in 0.9826 seconds (78.02 Kbytes/s)
200 PORT command successful.
150 Opening BINARY mode data connection for ypatchy (161238 bytes).
226 Transfer complete.
161238 bytes received in 0.9318 seconds (169 Kbytes/s)
200 PORT command successful.
150 Opening BINARY mode data connection for ysearch (88551 bytes).
226 Transfer complete.
88551 bytes received in 0.9307 seconds (92.91 Kbytes/s)
200 PORT command successful.
150 Opening BINARY mode data connection for yshift (93915 bytes).
226 Transfer complete.
93915 bytes received in 1.032 seconds (88.9 Kbytes/s)
200 PORT command successful.
150 Opening BINARY mode data connection for ytobcd (73968 bytes).
226 Transfer complete.
73968 bytes received in 0.4764 seconds (151.6 Kbytes/s)
200 PORT command successful.
150 Opening BINARY mode data connection for ytobin (84577 bytes).
226 Transfer complete.
84577 bytes received in 0.533 seconds (155 Kbytes/s)
200 PORT command successful.
150 Opening BINARY mode data connection for ytoceta (88839 bytes).
226 Transfer complete.
88839 bytes received in 0.4553 seconds (190.5 Kbytes/s)
ftp> quit
221 Goodbye.
zfatal:/home/cp/jamie (5) 

\end{XMPt}


\subsection{Installing PATCHY from the installation kit}

To install PATCHY from the installation kit, first retrieve
the required files from asisftp.cern.ch as shown below.
\index{installing PATCHY}
\index{PATCHY, installation of}

\begin{XMPt}{Retrieving the PATCHY installation kit for Unix systems}

zfatal:/home/cr/cernlib (415) ftp asisftp
Connected to asisftp.cern.ch.
220 asisftp FTP server (Version 2.0WU(14) Fri Sep 17 15:39:37 MET DST 1993) ready
.
Name (asisftp:cernlib): anonymous
331 Guest login ok, send your complete e-mail address as password.
Password:
230-                   ________________________________________
230-                   Application Software Installation Server
230-                   ________________________________________
230-
230-   Welcome to the ASIS ftp server, developed by the CERN Computing and
230-   Networking Division to serve the High Energy Physics research community.
230-
230-   ftp clients may abort due to improper handling of such introductory
230-   messages. A dash (-) as the first character of your pw will suppress it.
230-
230-   The CERNlib software, located in the "cernlib" directory, is covered by
230-   CERN copyright. Before taking any material from this directory, please
230-   read the copyright notice "cernlib/copyright".
230-
230-   Please contact cernlib@cernvm.cern.ch for site registration. General
230-   support questions should be addressed to asis-support@asis01.cern.ch. 
230-
230 Guest login ok, access restrictions apply.
ftp> 
ftp> cd cernlib/rs_aix32/patchy/4.15/src
250-Please read the file README
250-  it was last modified on Tue Nov 30 13:19:59 1993 - 63 days ago
250 CWD command successful.
ftp> 
ftp> ls 
200 PORT command successful.
150 Opening ASCII mode data connection for file list.
README
make_patchy
p4boot.sh0
p4inceta
rceta.sh
226 Transfer complete.
ftp> 
ftp> get README
200 PORT command successful.
150 Opening ASCII mode data connection for README (4844 bytes).
226 Transfer complete.
4963 bytes received in 0.03007 seconds (161.2 Kbytes/s)
ftp> get make_patchy
200 PORT command successful.
150 Opening ASCII mode data connection for make_patchy (4434 bytes).
226 Transfer complete.
4548 bytes received in 0.02672 seconds (166.2 Kbytes/s)
ftp> get p4boot.sh0
200 PORT command successful.
150 Opening ASCII mode data connection for p4boot.sh0 (12888 bytes).
226 Transfer complete.
13335 bytes received in 0.1023 seconds (127.3 Kbytes/s)
ftp> get rceta.sh
200 PORT command successful.
150 Opening ASCII mode data connection for rceta.sh (8598 bytes).
226 Transfer complete.
8877 bytes received in 0.03805 seconds (227.8 Kbytes/s)
ftp> bin
200 Type set to I.
ftp> get p4inceta
200 PORT command successful.
150 Opening BINARY mode data connection for p4inceta (1573200 bytes).
226 Transfer complete.
1573200 bytes received in 8.896 seconds (172.7 Kbytes/s)
ftp> 
ftp> quit
221 Goodbye.
zfatal:/home/cr/cernlib (416) 

\end{XMPt}

We now ensure that the variable {\bf CERN} is correctly defined
and then run {\bf make\_patchy}.

\begin{XMPt}{Running make\_patchy}

chmod +x make_patchy

export CERN=/cernlib/cern

./make_patchy

\end{XMPt}

This will then launch the installation and verification procedure,
resulting in the following files:

\begin{DLtt}{1234567890}
\item[fcasplit]
\item[ycompar]
\item[yedit]
\item[yfrceta]
\item[yindex]
\item[ylist]
\item[ypatchy]
\item[ysearch]
\item[yshift]
\item[ytobcd]
\item[ytobin]
\item[ytoceta]
\end{DLtt}

Only {\bf fcasplit} and {\bf ypatchy} are required for the CERNLIB installation
procedures.

\section{VMS systems}

\label{sect-VMSPATCHY}

\subsection{Copying the PATCHY executables from VXCERN}

The PATCHY modules may be copied from VXCERN as shown below.

\begin{XMPt}{Copying the PATCHY modules from VXCERN}

COPY VXCERN::CERNVAX:[PATCHY.PRO.EXE]*.* * ! VAX versions

COPY VXCERN::CERNAXP:[PATCHY.PRO.EXE]*.* * ! AXP (Alpha) versions

\end{XMPt}
\subsection{Rebuilding PATCHY from the installation kit}

On VAX/VMS systems, the installation kit is available on VXCERN as shown below.

{\bf N.B. an installation kit is only available for VAX systems, i.e. there
is no Alpha installation kit}.

\begin{XMPt}{Copying the PATCHY installation kit from VXCERN}

VSCLIB? set def [.patchy]
VSCLIB? copy vxcern::cernvax:[patchy.src.vaxvms]*.* */log
%COPY-S-COPIED, VXCERN::CERNVAX:[PATCHY.SRC.VAXVMS]P4BOOT.SH0;2 copied to 
 DISK$USER1:[JAMIE.PATCHY]P4BOOT.SH0;2 (23 blocks)
%COPY-S-COPIED, VXCERN::CERNVAX:[PATCHY.SRC.VAXVMS]P4INCETA.CET;8 copied to 
 DISK$USER1:[JAMIE.PATCHY]P4INCETA.CET;8 (3235 blocks)
%COPY-S-COPIED, VXCERN::CERNVAX:[PATCHY.SRC.VAXVMS]PATCHY.COM;4 copied to 
 DISK$USER1:[JAMIE.PATCHY]PATCHY.COM;4 (3 blocks)
%COPY-S-COPIED, VXCERN::CERNVAX:[PATCHY.SRC.VAXVMS]RCETA.SH;3 copied to
 DISK$USER1:[JAMIE.PATCHY]RCETA.SH;3 (16 blocks)
%COPY-S-COPIED, VXCERN::CERNVAX:[PATCHY.SRC.VAXVMS]README.DOC;3 copied to
 DISK$USER1:[JAMIE.PATCHY]README.DOC;3 (10 blocks)
%COPY-S-NEWFILES, 5 files created

\end{XMPt}

Now customise {\bf PATCHY.COM} specifying the source, work and target directories.
The modules are then built by typing {\underline {\bf @PATCHY}}.

\begin{XMPt}{Result of running PATCHY.COM}

Directory DISK$USER1:[JAMIE.PATCHY.EXE]

YCOMPAR.EXE;1       YEDIT.EXE;1         YFRCETA.EXE;1       YINDEX.EXE;1       
YLIST.EXE;1         YPATCHY.EXE;1       YSEARCH.EXE;1       YSHIFT.EXE;1       
YTOBCD.EXE;1        YTOBIN.EXE;1        YTOCETA.EXE;1       

Total of 11 files.

\end{XMPt}

\section{VM systems}

\section{Other systems}

\chapter{Installing CERNLIB software on Unix systems}

\label{sect-UNIXINSTALL}

Below we describe two scenarios. The first is for a Unix system
at CERN, where the asis tree is available via \NFS{} or \AFS{}.
The second is for a remote system or for one where the asis tree
is not available.

\section{Installing CERNLIB when asis is available}

In the following examples, the CERNLIB tree is available via
AFS. The procedure is identical for the case when the CERNLIB
tree is mounted via \NFS{}.

\begin{XMPt}{Accessing the CERNLIB tree via \AFS{}}

\footnotesize{

zfatal:/hepdb/cdchorus (185) ls -l /cern
total 0
lrwxrwxrwx   1 root     system        26 Dec  7 21:02 93c -> /afs/cern.ch/asis/cern/93c
lrwxrwxrwx   1 root     system        26 Dec  7 21:02 93d -> /afs/cern.ch/asis/cern/93d
lrwxrwxrwx   1 root     system        26 Dec  7 21:02 94a -> /afs/cern.ch/asis/cern/94a
lrwxrwxrwx   1 root     system        26 Dec  7 21:02 WWW -> /afs/cern.ch/asis/cern/WWW
lrwxrwxrwx   1 root     system        26 Dec  7 21:02 cmz -> /afs/cern.ch/asis/cern/cmz
lrwxrwxrwx   1 root     system        26 Dec  7 21:02 mad -> /afs/cern.ch/asis/cern/mad
lrwxrwxrwx   1 root     system        26 Dec  7 21:02 man -> /afs/cern.ch/asis/cern/man
lrwxrwxrwx   1 root     system        26 Dec  7 21:02 new -> /afs/cern.ch/asis/cern/new
lrwxrwxrwx   1 root     system        26 Dec  7 21:02 old -> /afs/cern.ch/asis/cern/old
lrwxrwxrwx   1 root     system        29 Dec  7 21:02 patchy -> /afs/cern.ch/asis/cern/patchy
lrwxrwxrwx   1 root     system        28 Dec  7 21:02 phigs -> /afs/cern.ch/asis/cern/phigs
lrwxrwxrwx   1 root     system        26 Dec  7 21:02 pro -> /afs/cern.ch/asis/cern/pro
lrwxrwxrwx   1 root     system        28 Dec  7 21:02 share -> /afs/cern.ch/asis/cern/share

}

\end{XMPt}

Let us assume that we wish to reinstall the CERNLIB software in the {\bf /cernlib/cern}
tree. We first create these directories, and then a subdirectory for the version
that we wish to install. We procede as follows: 

\begin{XMPt}{Setting up the directory tree}

mkdir /cernlib/cern
mkdir /cernlib/cern/93d
mkdir /cernlib/cern/93d/bin
mkdir /cernlib/cern/93d/lib
mkdir /cernlib/cern/93d/log
mkdir /cernlib/cern/93d/src
mkdir /cernlib/cern/93d/doc
\end{XMPt}

We now set up a number of links.

\begin{XMPt}{Creating links into the \AFS{} tree}

cd /cernlib/cern/93d
ln -s /cern/93d/include include
ln -s /cern/93d/mgr mgr
cd src
ln -s /cern/93d/src/car car
ln -s /cern/93d/src/cmz cmz
ln -s car cra

\end{XMPt}

In fact, only the links for the {\bf car} and {\bf cra} directories are required
for what follows. 

\begin{DLtt}{1234567}
\item[car]The CERNLIB source files in {\bf PATCHY} card format. This
directory also contains the {\bf PATCHY} {\it cradles} (*.cra) used
to extract the code.
\item[cmz]The CERNLIB source files in {\bf CMZ} binary format.
\item[cra]A link to the {\bf cra} directory.
\item[doc]Documentation on various Monte Carlo generators.
\item[include]{\it include} files used when the CERNLIB code is called
from C.
\end{DLtt}

We now add the following commands to our profile.

\begin{XMPt}{Tailoring the {\bf .profile} of the cernlib account}

PATH=/cern/pro/bin:$PATH; export PATH

export CERN=/cernlib/cern
export CERN_LEVEL=93d
export PLISTA=DEV
. $CERN/$CERN_LEVEL/mgr/plienv.sh

\end{XMPt}

We then reexecute the {\bf .profile} and switch to the CERN manager directory.

\begin{XMPt}{Preparing to build the CERN software}

. .profile

cd $CERN/$CERN_LEVEL/mgr

\end{XMPt}

We can now build the complete CERN software by typing {\underline {\bf make all}}.
\index{Complete rebuild ! Unix}

\begin{XMPt}{Building the CERN software}

   make -n all

   ...

	makepack -p kerngen
	makepack -s -c kerngen

   ...

	makepack -s -c kernasw
	makepack -l kernlib
	makepack -p cspack
	makepack -s cspack
	makepack -l packlib -c cspack

   ...

	makepack -p isajetd
	rm -r /cernlib/cern/93d/src/cfs/isajetd
	makepack -p pdflibd
	rm -r /cernlib/cern/93d/src/cfs/pdflibd

\end{XMPt}

Various components can be built using the syntax {\underline {\bf make} {\it target}}.
Thus, to build the \PAW{} modules one would type {\underline {\bf make paw}}.
As the standard Unix {\bf make} is employed, all the dependancies are known
and intermediate components only rebuilt if required.

The following extract from the {\bf makefile} indicates which components can
be rebuilt separately or together.

\begin{XMPt}{Extract from cernlib makefile}

# **********************************************************************
# Make definitions                                                     *
# **********************************************************************
# ======================================================================
# >>> General makes
# ======================================================================
# all:        cernset products
  all:        cernset
  cernset:    cernlibs cernpgm userpgm mclibs mcdoc
 
  cernlibs:   kernlib packlib mathlib graflibs pawlib phtools
  cernpgm:    dzedit fatset kuipset paw rzconv flop tree telnetg    \\
              zftp pawserv zserv higzconv f2h hepdbset umlog
  userpgm:    garfield poisson
  mclibs:     ariadne cojets eurodec fritiof herwig isajet        \\
              jetset lepto pdflib photos
  mcdoc:      cojetsd eurodecd fritiofd herwigd isajetd jetsetd   \\
              pdflibd photosd pythiad
  shrlibs:    scernlib smathlib sgraflib sgeant
 
  products:   cmz gks historian nag
# ======================================================================
# >>> Basic Libraries
# ======================================================================
  kernlib:    kernlib.a
  packlib:    packlib.a
  mathlib:    mathlib.a
  phtools:    phtools.a

  graflibs:   graflib grafX11 grafGKS
  graflib:    graflib.a
  grafX11:    grafX11.a
  grafGKS:    grafGKS.a
  grafDGKS:   grafDGKS.a
  grafGL:     grafGL.a
  grafGPR:    grafGPR.a
  pawlib:     pawlib.a

  scernlib:   scernlib.a
  smathlib:   smathlib.a
  sgraflib:   sgraflib.a

  kernlib.a:  $(LIB)/libkernlib.a
  packlib.a:  $(LIB)/libpacklib.a
  mathlib.a:  $(LIB)/libmathlib.a
  phtools.a:  $(LIB)/libphtools.a

  graflib.a:  $(LIB)/libgraflib.a
  grafX11.a:  $(LIB)/libgrafX11.a
  grafGKS.a:  $(LIB)/libgrafGKS.a
  grafDGKS.a: $(LIB)/libgrafDGKS.a
  grafGL.a:   $(LIB)/libgrafGL.a
  grafGPR.a:  $(LIB)/libgrafGPR.a
  pawlib.a:   $(LIB)/libpawlib.a

  scernlib.a: $(LIB)/scernlib.a
  smathlib.a: $(LIB)/smathlib.a
  sgraflib.a: $(LIB)/sgraflib.a

\end{XMPt}

\section{Installing CERNLIB without asis}

If the CERN library directory tree is not accessible over \NFS{}
or  \AFS{}, we must first retrieve the compressed tar files containing
the source (cernsrc.tar.Z) and the installation scripts (cernmgr.tar.Z).~\footnote{Note that files are now packed using \GZIP{} and hence have extension .gz.}

We first connect to the asis server, as shown below.

\begin{XMPt}{Retrieving the CERNLIB sources and installation scripts}

zfatal:/cernlib/tmp (132) ftp asisftp.cern.ch
Connected to asisftp.cern.ch.
220 asisftp FTP server (Version 2.0WU(14) Fri Sep 17 15:39:37 MET DST 1993) ready
.
Name (asisftp.cern.ch:cernlib): anonymous
331 Guest login ok, send your complete e-mail address as password.
Password:
230-                   ________________________________________
230-                   Application Software Installation Server
230-                   ________________________________________
230-
230-   Welcome to the ASIS ftp server, developed by the CERN Computing and
230-   Networking Division to serve the High Energy Physics research community.
230-
230-   ftp clients may abort due to improper handling of such introductory
230-   messages. A dash (-) as the first character of your pw will suppress it.
230-
230-   The CERNlib software, located in the "cernlib" directory, is covered by
230-   CERN copyright. Before taking any material from this directory, please
230-   read the copyright notice "cernlib/copyright".
230-
230-
230-   Please contact cernlib@cernvm.cern.ch for site registration. General
230-   support questions should be addressed to asis-support@asis01.cern.ch. 
230-
230 Guest login ok, access restrictions apply.
ftp>cd cernlib/rs_aix32/93d/tar
250-
250- This directory contains compressed files of CERNlib release 93d for IBM/RS6
000
250-
250- Files ending in .tar.gz have been compressed using gzip. gzip/gunzip for
250- IBM/RS6000 are available in this directory in the gzip.tar file. Get this
250- first and untar in a directory in the search path. Also take a new copy of 

250- plitar; this will use gzip -d ( equivalent to gunzip ) to uncompress files
250- ending in .tar.gz.
250- 
250-
250 CWD command successful.
ftp> 

ftp> get plitar
200 PORT command successful.
150 Opening ASCII mode data connection for plitar (10951 bytes).
226 Transfer complete.
11262 bytes received in 0.06656 seconds (165.2 Kbytes/s)
ftp> get cernsrc.contents
200 PORT command successful.
150 Opening ASCII mode data connection for cernsrc.contents (22467 bytes).
226 Transfer complete.
22799 bytes received in 0.2305 seconds (96.6 Kbytes/s)
ftp> get cernmgr.contents
200 PORT command successful.
150 Opening ASCII mode data connection for cernmgr.contents (4937 bytes).
226 Transfer complete.
5016 bytes received in 0.01622 seconds (302 Kbytes/s)
ftp> bin

ftp> get cernmgr.tar.Z
200 PORT command successful.
150 Opening BINARY mode data connection for cernmgr.tar.Z (111289 bytes).
226 Transfer complete.
111289 bytes received in 0.7157 seconds (151.9 Kbytes/s)
ftp> get cernsrc.tar.Z
200 PORT command successful.
150 Opening BINARY mode data connection for cernsrc.tar.Z (19319023 bytes).
226 Transfer complete.
19319023 bytes received in 118.1 seconds (159.8 Kbytes/s)
ftp> 

\end{XMPt}

Having retrieved the installation kits, we now unpack using \plitar{}.

\begin{XMPt}{Unpacking the installation procedures}

zfatal:/cernlib/tmp (133) ls -l
total 38048
-rw-r--r--   1 cernlib  sys         4937 Feb  1 14:05 cernmgr.contents
-rw-r--r--   1 cernlib  sys       111289 Feb  1 14:06 cernmgr.tar.Z
-rw-r--r--   1 cernlib  sys        22467 Feb  1 14:05 cernsrc.contents
-rw-r--r--   1 cernlib  sys      19319023 Feb  1 14:08 cernsrc.tar.Z
-rw-r--r--   1 cernlib  sys        10951 Feb  1 14:05 plitar

zfatal:/cernlib/tmp (134) chmod +x plitar  # Make 'plitar' executable
zfatal:/cernlib/tmp (135) 
zfatal:/cernlib/tmp (135) plitar           # See if it gives us any help
##======================================================================
##
## PLITAR   93d.02  : CERN Program Library distribution utility
## Last update      : 93/10/28
##
##======================================================================
#
#  Syntax: plitar [ -n ] tar_options tar_file
#
#  plitar combines tar+compress utilities to pack/unpack the files being
#  part of the CERNlib distribution set; the corresponding _readme file
#  describes the contents of each of them. Please read it beforehand.
#  Location of tar files and the target install directory is controlled
#  through environment variables CERN, CERN_LEVEL and PLITMP
#
#  Examples:
#
#  plitar -n xvf cernlib   "non-execute" mode to display the action
#  plitar tvf cmz          examines the CMZ compressed-tar set
#  plitar xvf geant        installs the GEANT compressed-tar set
#
zfatal:/cernlib/tmp (136) echo $CERN
/cernlib/cern
zfatal:/cernlib/tmp (137) export CERN=/cernlib/kern
zfatal:/cernlib/tmp (138) mkdir $CERN
zfatal:/cernlib/tmp (139) echo $CERN_LEVEL
93d
zfatal:/cernlib/tmp (140) echo $PLITMP
/tmp
zfatal:/cernlib/tmp (141) export PLITMP=/cernlib/tmp
zfatal:/cernlib/tmp (145) plitar -n tvf cernmgr.tar.Z
##======================================================================
##
## PLITAR   93d.02  : CERN Program Library distribution utility
## Last update      : 93/10/28
##
##======================================================================
#
# -----------------------------------------------------------------------
# The 2 parameters CERN and PLITMP are environment variables
# which may be changed using setenv (in C-shell) or export (in sh,ksh)
# -----------------------------------------------------------------------
#
# Tar files expected in      PLITMP=/cernlib/tmp
# Target directory           CERN  =/cernlib/kern
#
         uncompress -vc /cernlib/tmp/cernmgr.tar.Z |  tar tvf  -


zfatal:/cernlib/tmp (146) plitar xvf cernmgr.tar.Z   
##======================================================================
##
## PLITAR   93d.02  : CERN Program Library distribution utility
## Last update      : 93/10/28
##
##======================================================================
#
# -----------------------------------------------------------------------
# The 2 parameters CERN and PLITMP are environment variables
# which may be changed using setenv (in C-shell) or export (in sh,ksh)
# -----------------------------------------------------------------------
#
# Tar files expected in      PLITMP=/cernlib/tmp
# Target directory           CERN  =/cernlib/kern
#
         uncompress -vc /cernlib/tmp/cernmgr.tar.Z |  tar xvf  -
x 93d/mgr/93d/irs.names, 4679 bytes, 10 media blocks.
x 93d/mgr/93d/dos.names, 3548 bytes, 7 media blocks.

...

x 93d/mgr/yexpand, 1230 bytes, 3 media blocks.


\end{XMPt}

We can now unpack the source files in the same way.

\begin{XMPt}{Unpacking the source files}
zfatal:/cernlib/tmp (150) plitar xvf cernsrc.tar.Z
##======================================================================
##
## PLITAR   93d.02  : CERN Program Library distribution utility
## Last update      : 93/10/28
##
##======================================================================
#
# -----------------------------------------------------------------------
# The 2 parameters CERN and PLITMP are environment variables
# which may be changed using setenv (in C-shell) or export (in sh,ksh)
# -----------------------------------------------------------------------
#
# Tar files expected in      PLITMP=/cernlib/tmp
# Target directory           CERN  =/cernlib/kern
#
         uncompress -vc /cernlib/tmp/cernsrc.tar.Z |  tar xvf  -
x 93d/src/car/comis.cra, 393 bytes, 1 media blocks.
x 93d/src/car/wylbur.car, 1063629 bytes, 2078 media blocks.
x 93d/src/car/kernvax.car, 196447 bytes, 384 media blocks.
x 93d/src/car/ariadne.cra, 134 bytes, 1 media blocks.

...

x 93d/src/car/pawdemo.car, 784187 bytes, 1532 media blocks.
x 93d/src/car/kernsgi2.car is a symbolic link to kernsgi.car.
x 93d/src/car/isajet72.car, 1226674 bytes, 2396 media blocks.
x 93d/src/car/isajet72.cra, 152 bytes, 1 media blocks.
zfatal:/cernlib/tmp (151) 

\end{XMPt}

We now add the following commands to our profile.

\begin{XMPt}{Tailoring the {\bf .profile} of the cernlib account}

PATH=/cern/pro/bin:$PATH; export PATH

export CERN=/cernlib/cern
export CERN_LEVEL=93d
export PLISTA=DEV
. $CERN/$CERN_LEVEL/mgr/plienv.sh

\end{XMPt}

We then reexecute the {\bf .profile} and switch to the CERN manager directory.

\begin{XMPt}{Preparing to build the CERN software}

. .profile

cd $CERN/$CERN_LEVEL/mgr

\end{XMPt}

We now create 3 directories and 1 link and are then ready to 
start the installation.

\begin{XMPt}{Completing the pre-installation phase}

zfatal:/cernlib/tmp (160) cd $CERN/$CERN_LEVEL    
zfatal:/cernlib/kern/93d (161) mkdir bin
zfatal:/cernlib/kern/93d (162) mkdir lib
zfatal:/cernlib/kern/93d (163) mkdir log
zfatal:/cernlib/kern/93d (164) mkdir doc
zfatal:/cernlib/kern/93d (165) 

zfatal:/cernlib/kern/93d (166) cd src
zfatal:/cernlib/kern/93d/src (165) ln -sf car cra
zfatal:/cernlib/kern/93d/src (166) 

\end{XMPt}

{\bf N.B. the source code of the package} {\it MPA} {\bf is not available
for distribution. For {\underline make all} to work, the dummy link
{\it mpa.car} in the {\it src/car} directory must be removed}.

\begin{XMPt}{Removing dummy mpa.car file}

rm /cernlib/kern/93d/src/car/mpa.car # as root if all else fails
touch /cernlib/kern/93d/src/car/mpa.car

\end{XMPt}

We may now rebuild the entire CERN library using {\underline {\bf make all}}
or one component, as shown below.

\begin{XMPt}{Building the \FATMEN{} module}

zfatal:/cernlib/kern/93d/mgr (181) make -n fatmen
        [ "n" = "n" -o "n" = "bn" ] && make -n -f userlib -f grouplib -f /cernlib/kern/93d/mgr/mkf
/cernlib  -f /cernlib/kern/93d/mgr/mkf/geant fatmen \\
                             || make -n -f userlib -f grouplib -f /cernlib/kern/93d/mgr/mkf/cernlib  -f /cernlib/kern/93d/mgr/mkf/geant fatmen >>/cernlib/kern/93d/log/fatmen 2>&1
        makepack -p kuipc
        makepack -s -c kuipc
        makepack -o kuipc kuipc
        makepack -p fatmen
        makepack -s -c fatmen
        makepack -p cspack
        makepack -s cspack
        makepack -l packlib -c cspack
        makepack -p cdlib
        makepack -s cdlib
        makepack -l packlib -c cdlib
        makepack -p epio
        makepack -s epio
        makepack -l packlib -c epio
        makepack -p fatlib
        makepack -s fatlib
        makepack -l packlib -c fatlib
        makepack -p ffread
        makepack -s ffread
        makepack -l packlib -c ffread
        makepack -p hbook
        makepack -s hbook
        makepack -l packlib -c hbook
        makepack -p kapack
        makepack -s kapack
        makepack -l packlib -c kapack
        makepack -p kuip
        makepack -s kuip
        makepack -l packlib -c kuip
        makepack -p minuit
        makepack -s minuit
        makepack -l packlib -c minuit
        makepack -p zbook
        makepack -s zbook
        makepack -l packlib -c zbook
        makepack -p zebra
        makepack -s zebra
        makepack -l packlib -c zebra
        makepack -p kerngen
        makepack -s -c kerngen
        makepack -p kernnum
        makepack -s -c kernnum
        makepack -p kernasw
        makepack -s -c kernasw
        makepack -l kernlib
        makepack -l packlib
        makepack -o fatmen fatmen

\end{XMPt}

\chapter{Installing CERNLIB software on VMS systems}

CERNLIB software is installed on VMS systems using the 
{\bf CERN\_ROOT:[MGR]MAKE.COM} procedure. The symbol
{\bf MAKE} is defined as {\bf @CERN\_ROOT:[MGR]MAKE}
by the procedure {\bf PLIENV}, described on page~\pageref{sect-DCL}.
The following examples show how one may build
library components, complete libraries or packages.

\section{Building standalone libraries}

Complete libraries may be built using the syntax {\underline {\bf make {\it target}}}.
For example, {\bf KERNLIB} is built as follows:

\begin{XMPt}{Building KERNLIB}

vxcrna:/cernlib > make -n kernlib
        makepack -p KERNASW 
        makepack -s KERNASW 
        makepack -c KERNASW 
        makepack -p KERNNUM 
        makepack -s KERNNUM 
        makepack -c KERNNUM 
        makepack -p KERNGEN 
        makepack -s KERNGEN 
        makepack -c KERNGEN 
        makepack -l KERNLIB

\end{XMPt}

As for the standard Unix make, the option {\bf -n} tells {\bf make}
just to list what it would do and not actually execute the commands.

PACKLIB may be built in a similar manner, as shown below.

\begin{XMPt}{Building PACKLIB}

vxcrna:/cernlib > make -n packlib
        makepack -p CSPACK  
        makepack -s CSPACK  
        makepack -c CSPACK  
        makepack -p EPIO    
        makepack -s EPIO    
        makepack -c EPIO    
        makepack -p FATLIB  
        makepack -s FATLIB  
        makepack -c FATLIB  
        makepack -p FFREAD  
        makepack -s FFREAD  
        makepack -c FFREAD  
        makepack -p HBOOK   
        makepack -s HBOOK   
        makepack -c HBOOK   
        makepack -p KAPACK  
        makepack -s KAPACK  
        makepack -c KAPACK  
        makepack -p KUIP    
        makepack -s KUIP    
        makepack -c KUIP    
        makepack -p MINUIT  
        makepack -s MINUIT  
        makepack -c MINUIT  
        makepack -p ZBOOK   
        makepack -s ZBOOK   
        makepack -c ZBOOK   
        makepack -p ZEBRA   
        makepack -s ZEBRA   
        makepack -c ZEBRA   
        makepack -p CDLIB   
        makepack -s CDLIB   
        makepack -c CDLIB   
        makepack -l PACKLIB

\end{XMPt}

Both KERNLIB and PACKLIB contain a number of components. Let us first
examine how a library containing only one component is built.

\section{Building a simple library}
\begin{XMPt}{Building JETSET}

vxcrna:/cernlib > make -n jetset 
        makepack -p JETSET 
        makepack -s JETSET 
        makepack -c JETSET 
        makepack -l JETSET 

\end{XMPt}

JETSET is built in four steps:

\begin{DLtt}{12}
\item[-p]This invokes the {\bf PATCHY} step, to extract the 
source code from the {\bf .CAR} file.
\item[-s]This invokes the {\bf FCASPLIT} step, to split the
extracted code into separate files, one per routine (strictly,
one per PATCHY DECK).
\item[-c]This invokes the compile step, to compile the individual
routines.
\item[-l]This creates the library out of the object files created
by the previous step.
\end{DLtt}

\section{Building a complex library}

A complex library, such as KERNLIB or PACKLIB, may be built
in one go, as shown above, or component by component.
The former is useful when one wishes to install a new release
of CERNLIB, or install CERNLIB from scratch. The latter
is more appropriate if only one or a few packages have changed.

For example, if a routine in ZEBRA has been modified, we may
rebuild PACKLIB using the following steps:

\begin{enumerate}
\item
{\underline make zebra}
\item
{\underline makepack -l packlib}
\end{enumerate}

The first command will cause the ZEBRA source code to be extracted,
split and compiled. The second will rebuild PACKLIB from all of the
appropriate object files.

\section{Building a module}

Modules are built in a similar manner. For example, the \HEPDB{}
server {\bf CDSERV} is built as follows:

\begin{XMPt}{Building the \HEPDB{} server CDSERV}

vxcrna:/cernlib > make -n cdserv
        makepack -p CDSERV 
        makepack -c CDSERV 
        makepack -o CDSERV  CDSERV 

\end{XMPt}

Here we see that three steps are involved.

\begin{DLtt}{12}
\item[-p]The source code of the server is extract by the PATCHY step.
\item[-c]The code is compiled.
\item[-o]The code is linked to produce an executable module.
\end{DLtt}

\section{Building sets of modules}

One may also build multiple modules in one go. For example,
rather than rebuild all different versions of \PAW{} individually,
one may request that they are all rebuilt, as follows:

\begin{XMPt}{Building multiple modules}

vxcrna:/cernlib > make -n pawall
        makepack -p PAWMDNET
        makepack -s PAWMDNET
        makepack -c PAWMDNET
        makepack -o PAWX11   PAWMDNET
        makepack -p PAWMDNET
        makepack -s PAWMDNET
        makepack -c PAWMDNET
        makepack -o PAWDECW  PAWMDNET
        makepack -p PAWMDNET
        makepack -s PAWMDNET
        makepack -c PAWMDNET
        makepack -o PAWGKS  PAWMDNET
        makepack -p PAWPP   
        makepack -s PAWPP   
        makepack -c PAWPP   
        makepack -o PAWPP    PAWPP   
        makepack -p KXTERM  
        makepack -s KXTERM  
        makepack -c KXTERM  
        makepack -o KXTERM   KXTERM
        makepack -p PAWM    
        makepack -s PAWM    
        makepack -c PAWM    
        makepack -o PAWGKS_T PAWM    
        makepack -p PAWM    
        makepack -s PAWM    
        makepack -c PAWM    
        makepack -o PAWX11_T PAWM    

\end{XMPt}

In this case, KXTERM is also rebuilt as it is required by PAW++.

\section{Handling dependencies}

As we are not using a true {\it make} utility on VMS systems,
the installer must be aware of the dependencies of various
components of CERNLIB. This has the following consequences:

\begin{enumerate}
\item
A complete installation of CERNLIB from scratch must be done
in the correct order.
\item
When reinstalling a particular package, one must take care
to reinstall all components that have changed in the correct
order.
\end{enumerate}

We hope to introduce a {\it make} or make-like utility which
will simplify the installation.

\section{Recommended procedure for installing CERNLIB}

{\bf N.B. if there has been no previous installation of CERNLIB
on your system, see section \ref{sect-VMSSETUP}.}

\begin{enumerate}
\item
Build the libraries in the following order:
\begin{enumerate}
\item
\KERNLIB{}
\item
\MATHLIB{}
\item
\PACKLIB{}
\item
Graphics libraries
\begin{itemize}
\item
\GRAFLIB{}
\item
GRAFGKS
\item
GRAFDGKS
\item
GRAFX11
\end{itemize}
\end{enumerate}
\item
Make PAWLIB
\item
Make the modules (\PAW{}, \FATMEN{}, \HEPDB{} etc.)
\item
Make the Monte Carlo and other stand alone libraries
\begin{itemize}
\item
ARIADNE
\item
COJETS
\item
EURODEC
\item
FRITIOF
\item
HERWIG
\item
ISAJET
\item
JETSET
\item
LEPTO
\item
PDFLIB
\item
PHOTOS
\item
PHTOOLS
\item
TWISTER
\end{itemize}
\end{enumerate}

\subsection{Rebuilding the complete CERN libraries}

All of the libraries and modules can be rebuilt using the following
command file.

\index{MAKEALL}
\index{Complete rebuild ! VMS}

\begin{XMPt}{MAKEALL.COM}
$!
$! Make complete CERNLIB 
$!
$	set noon
$	save_message = f$environment("MESSAGE")
$!
$       warnings_from = "."
$       errors_from   = "."
$       severe_from   = "."
$!
$	cernlib = -
"KERNLIB,MATHLIB,PACKLIB,GRAFLIB,GRAFGKS,GRAFDGKS,GRAFX11,PAWLIB" + -
",CERNPGM,USERPGM,MCLIBS,GEANT321"
$	if p1 .nes. "" then cernlib = p1
$	count   = 0
$ packages:
$	set message 'save_message'
$	package = f$element(count,",",cernlib)
$	if package .eqs. "," then goto end
$	write sys$output "Building ''package' at ''f$time()'"
$	make 'package'
$	wait 0:0:10
$	set message/nofacility/noidentification/noseverity/notext
$ p_wait:
$	show process/nooutput 'package'
$	if $severity .eq. 0 
$	then
$	   write sys$output "''package' complete at ''f$time()'"
$          search cern:[new.log]'package'.log "*** "
$!
$          search/nooutput cern:[new.log]'package'.log "*** WARNING EXIT from"
$	   if $severity .eq. 1 then warnings_from = warnings_from + package + "."
$          search/nooutput cern:[new.log]'package'.log "*** ERROR EXIT from"
$	   if $severity .eq. 1 then errors_from = errors_from + package + "."
$          search/nooutput cern:[new.log]'package'.log "*** SEVERE ERROR EXIT from"
$	   if $severity .eq. 1 then severe_from = severe_from + package + "."
$!
$	   count = count + 1
$	   goto packages
$	endif
$	wait 0:1
$	goto p_wait
$ end:
$	write sys$output "CERNLIB build complete at ''f$time()'"
$	set message 'save_message'
$	if warnings_from .nes. "." then write sys$output -
"Warnings from ''warnings_from'"
$	if errors_from .nes.   "." then write sys$output -
"Errors from ''errors_from'"
$	if severe_from .nes.   "." then write sys$output -
"Severe errors from ''severe_from'"
\end{XMPt}

\subsection{Rebuilding PAW}

All of the PAW modules, libraries and associated packages can be rebuilt
using the following command file.

\index{MAKEPAW}
\index{Rebuilding PAW ! VMS systems}

\begin{XMPt}{Command file to rebuild PAW}
$!
$! Rebuild PAW
$!
$	set noon
$	save_message = f$environment("MESSAGE")
$!
$       warnings_from = "."
$       errors_from   = "."
$       severe_from   = "."
$!
$	cernlib = "PAWLIB,GRAFLIB,GRAFGKS,GRAFDGKS,GRAFX11,KUIP,PACKLIB,PAWALL"
$	if p1 .nes. "" then cernlib = p1
$	count   = 0
$ packages:
$	set message 'save_message'
$	package = f$element(count,",",cernlib)
$!
$! Treat PACKLIB specially
$!
$       if package .eqs. "PACKLIB"
$       then
$          makepack -l packlib
$          count = count + 1
$          goto packages
$       endif
$!
$	if package .eqs. "," then goto end
$	write sys$output "Building ''package' at ''f$time()'"
$	make 'package'
$	wait 0:0:10
$	set message/nofacility/noidentification/noseverity/notext
$ p_wait:
$	show process/nooutput 'package'
$	if $severity .eq. 0 
$	then
$	   write sys$output "''package' complete at ''f$time()'"
$          search cern:[new.log]'package'.log "*** "
$!
$          search/nooutput cern:[new.log]'package'.log "*** WARNING EXIT from"
$	   if $severity .eq. 1 then warnings_from = warnings_from + package + "."
$          search/nooutput cern:[new.log]'package'.log "*** ERROR EXIT from"
$	   if $severity .eq. 1 then errors_from = errors_from + package + "."
$          search/nooutput cern:[new.log]'package'.log "*** SEVERE ERROR EXIT from"
$	   if $severity .eq. 1 then severe_from = severe_from + package + "."
$!
$	   count = count + 1
$	   goto packages
$	endif
$	wait 0:1
$	goto p_wait
$ end:
$	write sys$output "PAW build complete at ''f$time()'"
$	set message 'save_message'
$	if warnings_from .nes. "." then write sys$output -
"Warnings from ''warnings_from'"
$	if errors_from .nes.   "." then write sys$output -
"Errors from ''errors_from'"
$	if severe_from .nes.   "." then write sys$output -
"Severe errors from ''severe_from'"
\end{XMPt}

\subsection{Installing KERNLIB}
\KERNLIB{} can be built using the command {\underline {\bf make kernlib}}.
\index{KERNLIB}
\subsection{Installing MATHLIB}
\MATHLIB{} can be built using the command {\underline {\bf make mathlib}}.
This will execute the following commands:
\index{MPA}
\index{LAPACK}
\index{BVSL}
\begin{XMPt}{Building MATHLIB}

vxcrna:/cernlib > make -n mathlib
        makepack -p LAPACK  
        makepack -s LAPACK  
        makepack -c LAPACK  
        makepack -p BVSL    
        makepack -s BVSL    
        makepack -c BVSL    
        makepack -p MPA     
        makepack -s MPA     
        makepack -c MPA     
        makepack -p GEN     
        makepack -s GEN     
        makepack -c GEN     
        makepack -l MATHLIB

\end{XMPt}
\subsection{Installing PACKLIB}
\index{PACKLIB}
\PACKLIB{} can be built using the command {\tt make packlib}.
PACKLIB currently consists of the following packages:
\begin{itemize}
\item
\CSPACK{}~\cite{bib-CSPACK}
\item
EPIO~\cite{bib-EPIO}
\item
\FATMEN{}~\cite{bib-FATMEN}
\item
FFREAD~\cite{bib-FFREAD}
\item
\HBOOK{}~\cite{bib-HBOOK}
\item
\HEPDB{}~\cite{bib-HEPDB}
\item
KAPACK~\cite{bib-KAPACK}
\item
KUIP~\cite{bib-KUIP}
\item
MINUIT~\cite{bib-MINUIT}
\item
ZBOOK~\cite{bib-ZBOOK}
\item
ZEBRA~\cite{bib-ZEBRA}
\end{itemize}
\subsection{Installing graphics libraries}
The graphics libraries are divided into a kernel library, \GRAFLIB{},
and package specific libraries:
\begin{DLtt}{12345678}
\item[GRAFGKS]GTS-GRAL GKS specific routines
\item[GRAFDGKS]DEC GKS specific routines
\item[GRAFX11]X11 specific routines
\end{DLtt}

They may be installed using the command {\underline {\bf make GRAFLIB GRAFGKS GRAFDGKS GRAFX11}}.

\subsection{Building the Monte Carlo and other stand alone libraries}
These may be made using the command {\underline {\bf make {\it target}}}. Note
that some of the packages are available in several versions.
For example, versions 6.3, 7.3 and 7.4 of {\bf JETSET} are all available
at the time of writing. The available versions can be found as shown
below.

\begin{XMPt}{Listing the available versions of a given package}

vxcrna:/cernlib > dir cern:[new.src.car]jetset*.car

Directory CERN:[NEW.SRC.CAR]

JETSET63.CAR;1      JETSET73.CAR;1      JETSET74.CAR;1     

Total of 3 files.

\end{XMPt}

Version 7.4 of {\bf JETSET} is installed by typing {\underline {\bf make jetset74}}.

In a number of cases, the documentation is extracted by typing
{\underline {\bf make targetD}}, e.g. {\underline {\bf make herwig54D}}. If this is the case,
you will find an appropriate cradle in the {\bf CERN\_LEVEL:[SRC.CAR]}
area, e.g. {\bf herwig54D.cra}.

\subsection{Installing PAWLIB}

The \PAW{} library \PAWLIB{} is required if you wish to build 
\PAW{} or if you intend
to link your own applications with the \PAW{} routines. It is built
using the command {\underline {\bf make pawlib}}. This will extract and compile
the various components, as shown below.

\begin{XMPt}{Building PAWLIB}

vxcrna:/cernlib > make -n pawlib
        makepack -p PAW     
        makepack -s PAW     
        makepack -c PAW     
        makepack -p COMIS   
        makepack -s COMIS   
        makepack -c COMIS   
        makepack -p SIGMA    
        makepack -s SIGMA    
        makepack -c SIGMA    
        makepack -l PAWLIB 

\end{XMPt}

\subsection{Building the CERNLIB modules}

All of the following modules can be built using the syntax
{\underline {\bf make {\it target}}} except where indicated. Some
modules can be built together, e.g. the various versions of \PAW{}.

\begin{DLtt}{1234567890}
\item[AKMULT]Program to split a file containing multiple Macro routines
into individual files. (Obsolete?)
\item[HTONEW]Program to convert \HBOOK{} files. Not built with CERNLIB procedures.
\end{DLtt}

\begin{itemize}

\item
Sundry packages
\begin{DLtt}{1234567890}
\item[ASTUCE]Extract source from Historian/Update file. Only supported for MAD.
\item[CERNLIB]The CERNLIB command.
\item[FCONV]Convert a file between different formats.
\item[FLOP]Fortran coding convention checker.
\item[HIGZCONV]Convert HIGZ RZ metafiles into postscript or GKS format.
\item[SYSREQ]The SYSREQ command (used at CERN to access the Tape Management System)
\item[TREE]Show calling tree of a Fortran program.
\item[TNX11\_M]TELNETG program linked with Multinet TCP/IP. See the \CSPACK{} manual
for details~\cite{bib-CSPACK}.
\item[WYLBUR]A portable extended Wylbur like editor.
\item[XBANNER]Program to write text in large letters.
\end{DLtt}

\item
Zebra bank documentation programs. {\bf make dzedit} builds
DZEX11 and DZEGKS.
\begin{DLtt}{1234567890}
\item[DZDOC]Zebra bank documentation package.
\item[DZEX11]Interactive program linked with X11.
\item[DZEGKS]As above, linked with GTS-GRAL GKS.
\end{DLtt}

\item
\FATMEN{} programs. Can be built individually or collectively using
{\underline {\bf make fatset}}. See also the \FATMEN{} manual~\cite{bib-FATMEN}
for information on configuring the \FATMEN{} servers.
\begin{DLtt}{1234567890}
\item[FATMEN]The \FATMEN{} shell. 
\item[FATNEW]Program to build a new \FATMEN{} catalogue.
\item[FATSEND]Program to transfer \FATMEN{} updates between servers.
\item[FATSRV]The \FATMEN{} server.
\end{DLtt}

\item
Garfield programs.
\begin{DLtt}{1234567890}
\item[GARFIELD]
\item[GARFRUN]
\end{DLtt}

\item
\HEPDB{} programs. Can be built individually or collectively using
{\underline {\bf make hepdbset}}. See also the \HEPDB{} manual~\cite{bib-HEPDB}
for information on configuring the \HEPDB{} servers.
\begin{DLtt}{1234567890}
\item[HEPDB]The interactive interface.
\item[CDSERV]The \HEPDB{} server.
\item[CDMAKE]Program to make a new database.
\item[CDMOVE]Program to move updates between server queues.
\end{DLtt}

\item
KUIP programs. Can be built individually or collectively using
{\underline {\bf make kuipset}}.
\begin{DLtt}{1234567890}
\item[KUIPC]The KUIP compiler.
\item[KUESVR]The KUIP server.
\item[KXTERM]The KUIP terminal used in applications such as {\bf paw++}. Can
also be built by typing {\underline {\bf make paw++}} or {\underline {\bf make pawall}}.
\end{DLtt}

\item
The \PAW{}~\cite{bib-PAW} clients and server. Can be built individually (as shown) or
collectively, with the exception of PAWSERV, via {\underline {\bf make pawall}}, 
which also builds {\bf KXTERM}.
{\underline {\bf make pawset}} can be used to make all \PAW{} clients (i.e. not PAWSERV)
except PAW++.
{\bf N.B. the modules linked with GTS-GRAL GKS are only available on VAX systems}.
\begin{DLtt}{1234567890}
\item[PAWPP]PAW++ (as from 1/1/95, without TCP/IP)
\item[PAWPP_M]PAW++ linked with Multinet TCP/IP (as from 1/1/95)
\item[PAWPP_U]PAW++ linked with UCX TCP/IP (as from 1/1/95)
\item[PAWSERV]The \PAW{} server.
\item[PAWX11]\PAW{} linked with the X11 libraries, no TCP/IP.
\item[PAWX11\_DECW]\PAW{} linked with the old DECWindows libraries, no TCP/IP.
{\it (No longer built as from release 94B)}
\item[PAWX11\_M]\PAW{} linked with the X11 libraries and Multinet TCP/IP.
\item[PAWGKS]\PAW{} linked with GTS-GRAL GKS.
\item[PAWGKS\_M]\PAW{} linked with GTS-GRAL GKS and Multinet TCP/IP.
\end{DLtt}

\item
Programs to permit transfer of RZ files. These may be built individually,
e.g. {\underline {\bf make rfra}}, or collectively using {\underline {\bf make rzconv}}.
\begin{DLtt}{1234567890}
\item[RFRA]Convert an RZ file from FZ alpha exchange format
\item[RFRX]Convert an RZ file from FZ binary exchange format
\item[RTOA]Convert an RZ file into FZ alpha exchange format
\item[RTOX]Convert an RZ file into FZ binary exchange format
\end{DLtt}

\item
Poisson suite of programs. Can be built individually or collectively
using {\underline {\bf make poisson}}. Used to solve Poisson's or Laplace's equation
in 2 dimensional regions (magnetostatic, electrostatic or static
temperature problems).
\begin{DLtt}{1234567890}
\item[FORCCR]A solver to calculate the forces.
\item[LATTCR]Lattice defintion program.
\item[POISCR]Solver for Poisson's or Laplace's equation.
\item[TRIPCR]Postprocessor to generate a GKS metafile.
\end{DLtt}

\item
Zebra file transfer program and associated server. See also the \CSPACK{} manual~\cite{bib-CSPACK}
for information on configuring the servers.
\begin{DLtt}{1234567890}
\item[ZFTP]The client program.
\item[ZSERV]The server program.
\end{DLtt}

\item
{\bf VAXTAP} commands. Cannot be built using {\bf make}. See the {\bf VAXTAP}
manual~\cite{bib-VAXTAP} for installation details.
\begin{DLtt}{1234567890}
\item[EINIT]Initialise a tape with a VOL1 label written in EBCDIC. 
\item[LABELDUMP]Dump the VOL1 label of a tape.
\item[SETUP]Mount a tape, optional STK and / or TMS support.
\item[STAGE]Stage command.
\item[STAGECLN]Stage space manager.
\item[TAPECOPY]Copy a tape.
\item[WRTAPE]Write a disk file to tape with ASCII or EBCDIC labels.
\item[XTAPE]Examine the contents of a tape.
\end{DLtt}

\item
Monitoring utilities. 
\begin{DLtt}{1234567890}
\item[UMCOM]
\item[UMLOG]
\item[UMON]
\end{DLtt}

\end{itemize}

\section{TCP/IP considerations}

Some of the CERNLIB modules require TCP/IP socket libraries. The list
of these modules is defined by the symbol {\bf need\_tcp} in 
{\bf makepack.com}. At the time of writing, this is defined as shown below.

\begin{XMPt}{List of modules requiring TCP/IP}

$ need_tcp = ".ZFTP.ZSERV.PAWM.PAWPP.PAWSERV.TELNETG.SYSREQ.FATMEN."

\end{XMPt}

The CERNLIB installation procedures attempt to chose the correct version
of TCP/IP and act accordingly. This is done in the command file {\bf f\$tcpip.com}
as follows:

\begin{XMPt}{Determining the TCP/IP version}

$ If F$TRNLNM("TWG$ETC").nes.""     Then tcpip_var="Wollongong WINTCP"
$ If F$TRNLNM("MULTINET").nes.""    Then tcpip_var="MultiNet TCPIP"
$ If F$TRNLNM("UCX$NETWORK").nes."" Then tcpip_var="UCX TCPIP"

\end{XMPt}

If you do not have one of these systems installed, then you will need to
modify {\bf f\$tcpip.com} and {\bf makepack.com} accordingly.

There are 3 areas that might require modification:

\begin{enumerate}
\item
The selection of the appropriate include files.
\item
Definitions for the C preprocessor.
\item
The appropriate options file for linking.
\end{enumerate}

\begin{XMPt}{Selecting the appropriate include files}

$ If pack.eqs."TELNETG"
$ Then If tcppg.eqs."_W"
$      Then TCPDIR="TWG$TCP:[netdist.include"
$           assign/user_mode 'TCPDIR'],'TCPDIR'.sys],'TCPDIR'.net],-
                             'TCPDIR'.netinet],sys$library vaxc$include
$           assign/user_mode 'TCPDIR'.net]     net
$           assign/user_mode 'TCPDIR'.arpa]    arpa
$      Endif
$      If tcppg.eqs."_M"
$      Then TCPDIR="MULTINET_ROOT:[Multinet.include"
$           assign/user_mode 'RDIR'.src.cfs.cspack] arpa
$      Endif
$      If tcppg.eqs."_U"
$      Then TCPDIR="UCX??"
$      Endif
$           assign/user_mode 'TCPDIR'.sys]     sys
$           assign/user_mode 'TCPDIR'.netinet] netinet
$ Endif

\end{XMPt}

\begin{XMPt}{Definitions for the C preprocessor}

$      If tcppg.eqs."_W"   Then cco=cco+"/DEF=TWG"
$      If tcppg.eqs."_M"   Then cco=cco+"/DEF=TGV"

\end{XMPt}

\begin{XMPt}{Link options file for Multinet, UCX and Wollongong}

::::::::::::::
vmslib_m.opt
::::::::::::::
SYS$LIBRARY:vaxcrtl/share
multinet_socket_library/share


::::::::::::::
vmslib_u.opt
::::::::::::::
SYS$LIBRARY:ucx$ipc/lib
SYS$LIBRARY:vaxcrtl/share

::::::::::::::
vmslib_w.opt
::::::::::::::
CERN_ROOT:[lib]VMSLIB/lib
SYS$LIBRARY:vaxcrtl/share

\end{XMPt}

\

\chapter{Installing CERNLIB software on VM systems}

This chapter describes how CERNLIB software is installed on systems
running VM/CMS. It assumes that the environment has been set up
as described in section \ref{sect-CERNVM}.

The actual software installation is performed in batch and is
controlled by a service machine. Commands are sent to this
service machine using the {\bf SLIB} exec.
The commands used are similar to those used on VMS systems,
except that they are prefixed by {\bf SLIB}.

\section{Components of the CERN libraries on VM systems}

The following diagram shows the structure and components of
the CERN libraries on VM systems. This structure is reflected
in the {\bf MAKELIB NAMES} file.

\begin{XMPt}{CERNLIB structure on VM}
 //CERN/CNDIV/CERNLIB
                     /ALL
                         /CERNLIBS
                                  /KERNLIB
                                          /KERNASW
                                          /KERNNUM
                                          /KERNGEN
                                  /PACKLIB
                                          /EPIO
                                          /FFREAD
                                          /IOPACK
                                          /KAPACK
                                          /KUIP
                                          /HBOOK
                                          /MINUIT
                                          /VMIO
                                          /ZBOOK
                                          /ZEBRA
                                          /CDLIB
                                          /FATLIB
                                  /MATHLIB
                                          /LAPACK
                                          /MPA
                                          /BVSL
                                          /GEN
                                  /GRAFLIBS
                                           /GRAFLIB
                                           /GRAFGKS
                                           /GRAFGDDM
                                           /GRAFX11
                                  /PAWLIB
                                         /PAW
                                         /COMIS
                                         /SIGMA
                         /MCLIBS
                                /COJETS
                                       /COJETS
                                       /COJETSD
                                /EURODEC
                                        /EURODEC
                                        /EURODECD
                                /HERWIG
                                /ISAJET
                                /JETSET
                                       /JETSET
                                       /PYTHIA
                                /LEPTO
                                /PHOTOS
                                /PDFLIB
                                /TWISTER
                                /FRITIOF
                                /ARIADNE
                                /MCDOC
                                      /COJETSD
                                      /EURODECD
                                      /HERWIGD
                                      /ISAJETD
                                      /JETSETD
                                      /PYTHIAD
                                      /PHOTOSD
                                      /PDFLIBD
                                      /FRITIOFD
                                      /ARIADNED
                         /CERNPGM
                                 /PAWM
                                      /PAWGKS
                                      /PAWGDDM
                                      /PAWX11
                                 /KUIPC
                                 /FATSET
                                        /FATMEN
                                        /FATNEW
                                        /FATSRV
                                        /FATSEND
                                 /RZCONV
                                        /RTOA
                                        /RFRA
                                        /RTOX
                                        /RFRX
                                 /FLOP
                                 /TREE
                                 /ZSERV
                                 /PAWSERV
                                 /ZFTP
                                 /TELNETG
                                 /DZEDIT
                                        /DZEGKS
                                        /DZEGDDM
                                        /DZEX11
                                 /HIGZCONV
                                 /GRTREE
                                        /GRTGKS
                                        /GRTPS
                         /USERPGM
                                 /BANNER
                                 /GARFIELD
                                 /MAGNET
                                 /POISSON
                                 /TRSPRT
                                 /TURTLE
 
\end{XMPt}

\section{Building standalone libraries}

Complete libraries are built using the syntax {\underline {\bf make {\it target}}}.
For example, {\bf KERNLIB} is built by typing {\underline {\bf SLIB MAKE KERNLIB}}.
Again, this is similar to the VMS case, except that the {\bf SLIB} exec sends
the command to the {\bf LIBSERV} service machine. This service machine
runs the appropriate {\bf MAKELIB} job in batch. {\bf MAKELIB} is controlled by a 
configuration file {\bf MAKELIB NAMES} on the {\bf PUBCR 197}.

To explain how this works, let us examine the case of {\bf KERNLIB} in more detail.

The command {\bf MAKE KERNLIB} corresponds to an entry in the {\bf MAKELIB NAMES}
file as shown below.

\begin{XMPt}{MAKE\_KERNLIB entry in MAKELIB NAMES file}

************************************************************************
* KERNLIB Library                                                      *
************************************************************************
:NICK.MAKE_KERNLIB
:JOB.KERNLIB   :BOPT.TIME  0:10 JOBID KERNLIB
               :QSUB.
:TXTLIB.KERNASW KERNNUM KERNGEN
:FROM.  KERNASW KERNNUM KERNGEN

\end{XMPt}

In this case, {\underline {\bf make kernlib}} will rebuild KERNLIB from the
individual TXTLIBs KERNASW, KERNNUM and KERNGEN. To rebuild
the individual TXTLIBs, {\underline {\bf make from kernlib}} should be used.
In this case, one must ensure that the jobs terminate correctly before
rebuilding KERNLIB from its components.

PACKLIB is built in a similar manner. The entry in MAKELIB names is as follows:

\begin{XMPt}{MAKE\_PACKLIB entry in MAKELIB NAMES file}

************************************************************************
* PACKLIB Library                                                      *
************************************************************************
:NICK.MAKE_PACKLIB
:JOB.PACKLIB   :BOPT.TIME  0:15 JOBID PACKLIB
               :QSUB.
:TXTLIB.CSPACK EPIO FFREAD IOPACK KAPACK KUIP HBOOK MINUIT VMIO
        ZBOOK ZEBRA KERNASW KERNNUM KERNGEN CDLIB FATLIB
:FROM.  CSPACK EPIO FFREAD IOPACK KAPACK KUIP HBOOK MINUIT VMIO
        ZBOOK ZEBRA CDLIB FATLIB

\end{XMPt}

\section{Building the CERNLIB TXTLIBs}

\begin{itemize}

\item
Basic libraries
\begin{DLtt}{1234567890}
\item[\KERNLIB{}]Use {\underline {\bf slib make from kernlib}} to build the components
and then {\underline {\bf slib make kernlib}} to assemble the library.
\item[\MATHLIB{}]Use {\underline {\bf slib make from mathlib}} to build the components
and then {\underline {\bf slib make mathlib}} to assemble the library.
\item[\PACKLIB{}]Use {\underline {\bf slib make from packlib}} to build the components
and then {\underline {\bf slib make packlib}} to assemble the library.
\item[\PAWLIB{}]Use {\underline {\bf slib make from pawlib}} to build the components
and then {\underline {\bf slib make pawlib}} to assemble the library.
\item[VKERNLIB]As {\bf KERNLIB}, but with vectorised code where available.
Use {\underline {\bf slib make from vkernlib}} to build the components
and then {\underline {\bf slib make vkernlib}} to assemble the library.
\item[VMATHLIB]As {\bf MATHLIB}, but with vectorised code where available.
Use {\underline {\bf slib make from vmathlib}} to build the components
and then {\underline {\bf slib make vmathlib}} to assemble the library.
\item[VPACKLIB]As {\bf KERNLIB}, but with vectorised code where available.
Use {\underline {\bf slib make from vpacklib}} to build the components
and then {\underline {\bf slib make vpacklib}} to assemble the library.
\end{DLtt}

\item
Graphics libraries. Use {\underline {\bf slib make from graflibs}}
and then {\underline {\bf slib make graflibs}} 
to build all or {\underline {\bf slib make} {\it target}} to build
individual components.
\begin{DLtt}{1234567890}
\item[GRAFGDDM]GDDM interface. 
\item[GRAFGKS]GKS interface. 
\item[\GRAFLIB{}]Graphics kernel.
\item[GRAFX11]X11 interface.
\end{DLtt}

\item
GEANT
\begin{DLtt}{1234567890}
\item[GEANT314]Geant version 3.14
\item[GEANT315]Geant version 3.15
\item[GEANT316]Geant version 3.16
\item[GEANT321]Geant version 3.21
\end{DLtt}

\item
Monte Carlo generators and other libraries, built
using {\underline {\bf slib make} {\it target}}. 
Can be built collectively using {\underline {\bf slib make mclibs}}.
\begin{DLtt}{1234567890}
\item[ARIADNE]
\item[COJETS]
\item[EURODEC]
\item[FRITIOF] 
\item[HERWIG56]
\item[ISAJET65]
\item[ISAJET72]
\item[JETSET63]
\item[JETSET73]
\item[LEPTO61]
\item[PDFLIB]
\item[PHOTOS]
\item[TWISTER]
\end{DLtt}

\item
Sundry
\begin{DLtt}{1234567890}
\item[CKERNEL]Emulation of C system routines.
\item[PHTOOLS]Collection of physics tools, e.g. {\bf FOWL}, {\bf GENBOD} etc.
\end{DLtt}

\end{itemize}

\section{Building the CERNLIB modules}

\begin{itemize}

\item
Sundry packages
\begin{DLtt}{1234567890}
\item[CMZ]
\item[HTOLIB]
\item[FCASPLIT]
\item[BANNER]Write text in large letters.
\item[MAKEDECK]
\item[FLOP]Fortran coding convention checker.
\item[HIGZCONV]Convert HIGZ RZ metafiles into postscript or GKS format.
\item[TREE]Show calling tree of a Fortran program.
\item[GRTGKS]Obsolete? (Convert tree to GKS metafile?)
\item[GRTPS]Obsolete? (Convert tree to postscript file?)
\end{DLtt}


\item
Zebra bank documentation programs.
\begin{DLtt}{1234567890}
\item[DZEX11]Interactive program linked with X11.
\item[DZEGKS]As above, linked with GTS-GRAL GKS.
\item[DZEGDDM]As above, linked with IBM's GDDM software.
\end{DLtt}

\item
\FATMEN{} programs. Can be built individually or collectively using
{\underline {\bf slib make fatset}}. See also the \FATMEN{} manual~\cite{bib-FATMEN}
for information on configuring the \FATMEN{} servers.
\begin{DLtt}{1234567890}
\item[FATMEN]The \FATMEN{} shell. 
\item[FATNEW]Program to build a new \FATMEN{} catalogue.
\item[FATSEND]Program to transfer \FATMEN{} updates between servers.
\item[FATSRV]The \FATMEN{} server.
\end{DLtt}

\item
GARFIELD programs.
\begin{DLtt}{1234567890}
\item[GARFIELD]
\item[GARFRUN]
\end{DLtt}

\item
\HEPDB{} programs. Can be built individually or collectively using
{\underline {\bf make hepdbset}}. See also the \HEPDB{} manual~\cite{bib-HEPDB}
for information on configuring the \HEPDB{} servers.
\begin{DLtt}{1234567890}
\item[HEPDB]The interactive interface.
\item[CDSERV]The \HEPDB{} server.
\item[CDMAKE]Program to make a new database.
\item[CDMOVE]Program to move updates between server queues.
\end{DLtt}

\item
KUIP programs.
\begin{DLtt}{1234567890}
\item[KUIPC]The KUIP compiler.
\end{DLtt}

\item
The \PAW{} clients and server. Can be built individually or collectively,
with the exception of {\bf PAWSERV}, using {\underline {\bf make pawm}}.
\begin{DLtt}{1234567890}
\item[PAWX11]\PAW{} linked with the X11 libraries and the TCP/IP libraries.
\item[PAWGDDM]\PAW{} linked with IBM's GDDM graphics package and the TCP/IP libraries.
\item[PAWGKS]\PAW{} linked with GTS-GRAL GKS and the TCP/IP libraries. 
\item[PAWSERV]The \PAW{} server
\end{DLtt}

\item
Programs to permit transfer of RZ files. These may be built individually,
e.g. {\underline {\bf make rfra}}, or collectively using {\underline {\bf make rzconv}}.
\begin{DLtt}{1234567890}
\item[RFRA]Convert an RZ file from FZ alpha exchange format
\item[RFRX]Convert an RZ file from FZ binary exchange format
\item[RTOA]Convert an RZ file into FZ alpha exchange format
\item[RTOX]Convert an RZ file into FZ binary exchange format
\end{DLtt}

\item
Poisson suite of programs. Can be built individually or collectively
using {\underline {\bf make poisson}}. Used to solve Poisson's or Laplace's equation
in 2 dimensional regions (magnetostatic, electrostatic or static
temperature problems). These programs are on the {\bf MAGNET}
disk on CERNVM ({\underline {\bf GIME MAGNET}}).
\begin{DLtt}{1234567890}
\item[FORCCR]A solver to calculate the forces.
\item[LATTCR]Lattice defintion program.
\item[POISCR]Solver for Poisson's or Laplace's equation.
\item[TRIPCR]Postprocessor to generate a GKS metafile.
\end{DLtt}

\item
Zebra file transfer program and associated server. See also the \CSPACK{} manual~\cite{bib-CSPACK}
for information on configuring the servers.
\begin{DLtt}{1234567890}
\item[BZFTP]The client program, {\it batch} version. Program
will exit with a unique return code in case of problems..
\item[ZFTP]The client program.
\item[ZSERV]The server program.
\end{DLtt}

\end{itemize}

\chapter{Installing CERNLIB software on MSDOS systems}

\chapter{Installing CERNLIB software on Windows/NT systems}

\chapter{Installing CERNLIB software on MVS systems}

\chapter{Installing CERNLIB on a Unix system that is not already supported}

\index{kernxxx}

{\bf If you try to install the CERN Program Library on a system
that is not currently supported, you will get an error from
makepack as shown below. Should this occur, follow the procedure
below. Please remember to provide your feedback to the CERN Program
Library office so that the work is not lost and can be shared by
others.}

\begin{XMPt}{Typical errors from makepack on unsupported Unix systems}
zfatal:/cernlib/cern/94a/mgr (59) make -n kernlib  

...

Make: make: 1254-002 Cannot find a rule to create target 
     /cernlib/cern/94a/src/car/kernxxx.car from dependencies.   <---
   Stop.
\end{XMPt}

\begin{itemize}
\item
Modify the script {\bf cernsys} to set the variables {\bf PLIUWC},
\PLINAME{} and {\bf PLISYS} as appropriate. The names
chosen for \PLINAME{} and {\bf PLIUWC} should be agreed
with the CERN Program Library office, if possible.
\begin{DLtt}{1234567890}
\item[PLIUWC]Three letter code indicating the machine type.
This code is used to find the system dependant part of {\bf KERNLIB}.
e.g., for the RS6000, {\bf PLIUWC} is set to {\bf irs} and the
RS6000 specific part of {\bf KERNLIB} is in {\bf kernirs.car}.
\item[PLINAME]The appropriate {\bf PATCHY} flag. This will
be automatically selected by the CERNLIB installation jobs.

If we were installing the libraries under a flavour of Unix known
as {\bf OBELIX}, we would add the following line to the {\bf cradle}.

\begin{XMPt}{Selecting Unix code on the OBELIX system}

+USE,UNIX,IF=OBELIX.

\end{XMPt}

\item[PLISYS]This should be set to {\bf SYSTEMV}, {\bf BSD}, {\bf MACH}
or {\bf UNKNOWN} as appropriate.
\end{DLtt}
\item
Chose a machine specific KERNLIB pam for a similar system and copy it, e.g.
to {\bf kernobx.car} in the case of {\bf OBELIX}.
\end{itemize}
\chapter{Installing CERNLIB software on other systems}

Should you wish to install the CERN program library on a machine
to which it has not already been ported, the following tips may
prove useful.

\section{Starting point}

Start from a system as close as possible to the new system.
For example, if you were porting the library to Alpha/VMS,
an appropriate starting point would be the VAX/VMS version.

\section{File naming conventions}

Most Unix systems use {\bf .f} for Fortran files, although the Apollo uses {\bf .ftn}.

\section{Compiler name and options}

The Fortran compiler is typically invoked using the {\bf f77} command on Unix
systems, although the RS6000 uses {\bf xlf} and the Convex {\bf fc}.

\section{Porting {\bf PATCHY}}

As the CERNLIB installation procedures currently use {\bf PATCHY}, you
will either have to port {\bf PATCHY} and possibly also the
splitting program {\bf FCASPLIT}, or extract the code on
a system to which these programs have already been ported.\footnote{The
latter technique was used to install the CERN libraries on Windows/NT.
{\bf PATCHY} was run on a PC running MSDOS and the output files
written to a Novell server, from where they could be accessed
from a Windows/NT system.}

If it is necessary to modify the compiler and/or options, one
should also remove the check of the file {\bf p4boot.sh} against
{\bf p4boot.sh0}. If there is a mismatch, the installation procedure
will exit.

Fortran installation packages. It may be necessary to make modifications
to the files {\bf rceta.f} or {\bf fcasplit.f}

\section{Likely areas of incompatibility}

There are a number of areas where incompatibilities between machines are likely
to arise. These include:

\begin{enumerate}
\item
Fortran {\bf OPEN} statements. Modifications are likely to routines such
as {\bf KUOPEN}, {\bf RZOPEN}, {\bf FMOPEN} etc. In addition to various
language extensions, such as the {\bf READONLY} and {\bf SHARED}
attributes in VAX Fortran, the units in which the record length of 
direct access files often varies (typically bytes or words).
\item
The syntax for file and directory names is likely to differ.
This will affect packages such as \CSPACK{}, \FATMEN{} and \HEPDB{}
amongst others.  
\item
Data representation. The majority of new systems support IEEE floating
point. If your system does not support IEEE floating point format,
then you will need to modify the KERNLIB package {\bf IE3CONV}.
If your system uses a floating point format that already exists,
then you should find the appropriate code in one of the KERNLIB 
pam files. For example, the routines to convert to and from IBM
floating point representation can be found in the {\bf KERNIBM}
pam file.
\item
Byte order. Most systems are {\it big endian}, which corresponds
to the way that we write numbers in every day life (i.e. the
left most bit has the highest significance. Some systems,
in particular DEC systems (VAX, Alpha, Ultrix) and IBM PCs
and compatibles, are {\it little endian}.
\item
Interface to the system. Routines in the KERNLIB package
{\bf CINTF} will probably require modification.
\item
The graphics packages may require heavy modification depending
on the graphics facilities on the target machine.
\end{enumerate}

\section{Porting the CERN libraries from Sun OS to Solaris}

The following modifications were required to port the
CERN libraries from Sun OS to Solaris.

\begin{XMPt}{Cradle for KERNGEN}

+EXE.
+USE,*KERNGEN,$PLINAME.
+ASM,22    ,IF=CRAY,IBMVM,VAXVMS.
+ASM,23,T=A,IF=IBMVM.kerngensh.sh
+ASM,24.
+ASM,31,T=A.:kerngen2F.f
+USE,QSYSBSD ,T=I,IF=SOLARIS.
+USE,QENVBSD ,T=I,IF=SOLARIS.
+USE,QSIGJMP   ,IF=SOLARIS.
+USE,QGETCWD   ,IF=SOLARIS.
+USE,QSIGPOSIX ,IF=SOLARIS.
+DIV,P=TCGEN,D=UCOPY2,IF=SOLARIS.
+DEL,P=SUNGS,D=JUMPAD,C=1-9,IF=SOLARIS.
+DEL,P=SUNGS,D=JUMPX2,C=1-46,IF=SOLARIS.
+PAM,11,T=C,A.$CERN_ROOT/src/car/kerngen
+PAM,12,T=C,A.$CERN_ROOT/src/car/kern$PLIUWC
+PAM,13,T=C,A,IF=IBMVM.$CERN_ROOT/src/car/kerncms
+PAM,14,T=C,A.$CERN_ROOT/src/car/kernfor
+QUIT.

\end{XMPt}

The above cradle has been slightly simplified for clarity.
However, we see that the main changes have been the selection
of certain flags that characterise the operating system.
These flags are described in appendix \ref{sect-FLAGS}
on page \pageref{sect-FLAGS}.

We repeat those selected for Solaris below.

\begin{DLtt}{12345678}

\item[QSYSBSD]   Unix system BSD (system 5 otherwise)
\item[QSIGJMP]   Posix sigsetjmp/siglongjmp for setjmp/longjmp
\item[QENVBSD]   BSD setenv is available
\item[QSIGBSD]   signal handling with BSD   sigvec
\item[QSIGPOSIX] signal handling with Posix sigaction

\end{DLtt}

\section{Porting CERNLIB to FACOM VPX series}

The FACOM VPX series run a Unix System V system.
However, the floating point representation
is that of IBM mainframes.

We start with the Sun Solaris versions of the libraries,
e.g. {\bf KERNSUN} with the flag {\bf SOLARIS}.

In \ZEBRA{}, we must ensure that data is correctly converted
on input and output. 

For IBM mainframes, the required definitions are in
the deck {\bf IBM} of patch {\bf FQ} in the Zebra pam file.
The conversion of data on input and output is performed
in the routines {\bf FZICV} and {\bf FZOCV} respectively.
The conversion is performed by sequences as shown below
(plus the corresponding sequences for input). For the FACOM,
the following is probably sufficient:

\begin{XMPt}{Selecting correction input/output conversion}

+USE, FQIE3FSC.    use default CALL IE3FOS for output single prec.
+USE, FQIE3FDC.    use default CALL IE3FOD for output double prec.
+USE, FQIE3TSC.    use default CALL IE3FOS for input single prec.
+USE, FQIE3TDC.    use default CALL IE3FOD for input double prec.

\end{XMPt}

which will call the {\bf KERNLIB} conversion routines (which must
of course be provide) for floating point data and copy as is
for all other data. (The VPX series uses the ASCII character
set and is big endian).

In this respect, the PATCH {\bf IBX}, which is for AIX on
IBM mainframes, and {\bf KERNIBX CAR}, which contains
Fortran versions of the floating point conversion routines,
may work directly on the FACOM.

\begin{DLtt}{1234567890}
\item[FZOCVFB]Output conversion of bit strings
\item[FZOCVFI]Output conversion of integer data
\item[FZOCVFF]Output conversion of single precision data
\item[FZOCVFD]Output conversion of double precision data
\item[FZOCVFH]Output conversion of hollerith data
\end{DLtt}
\chapter{Rebuilding components of the library {\it by hand}}

On occasion, one may need to rebuild a component of the CERN
library. This typically happens when there is a different
version of the operating system, compiler or shared library
on the local system to that used at CERN. Alternatively, 
one may wish to link to a licensed product that is either
not available at CERN or cannot be distributed. A typical
example is PAW.

\section{Rebuilding \PAW{} on VMS systems}

\index{Building PAW modules ! with UCX}
\index{Building PAW modules ! with Multinet}
\index{UCX}
\index{MULTINET}

\begin{XMPt}{Example of rebuilding \PAW{} on VMS systems}

$!
$! Build various versions of \PAW{}
$!
$ set noon
$!
$ architecture=f$edit(f$getsyi("ARCH_NAME"),"UPCASE")
$!
$ if architecture .eqs. "ALPHA" 
$    then 
$       cc:==cc/standard=vaxc
$       link:==link/nonative
$    endif
$!
$  ypatchy cern:['cern_level'.src.car]paw.car pawmain.for :go
+USE,CZ.
+USE,MAIN.
+USE,P=PAW,D=0PAMAIN.
+EXE.
+KEEP,PAWSIZ.
      PARAMETER (NWPAW=500000)
+PAM,T=C.
+QUIT.
$!
$  ypatchy cern:['cern_level'.src.car]paw.car pawpp.for :go
+USE,CZ.
+USE,MAIN,MOTIF.
+USE,P=PAW,D=0PAMAINM.
+EXE.
+KEEP,PAWSIZ.
      PARAMETER (NWPAW=500000)
+PAM,T=C.
+QUIT.
$!
$  create czdummy.for
      subroutine czdummy
      entry czopen
      entry czclos
      entry czputa
      entry czgeta
      entry czputc
      entry czgetc
      entry cztcp
      entry CONNECT
      entry GETHOSTBYNAME
      entry GETSERVBYNAME
      entry HTONS
      entry INET_ADDR
      entry RECV
      entry SELECT
      entry SEND
      entry SETSOCKOPT
      entry SHUTDOWN
      entry SOCKET
      entry multinet_get_socket_errno_addr
      entry socket_close
      entry socket_ioctl
      entry socket_perror
      end
$!
$  create gethostname.c
/*
 * return the node name
 */
#include <descrip.h>
#include <lnmdef.h>
int
gethostname( node, len )
	char *node;
	int len;
{
  $DESCRIPTOR( tabnam, "LNM$SYSTEM" );
  $DESCRIPTOR( lognam, "SYS$NODE" );
  int length = 0;
  struct {
    short buffer_length;
    short item_code;
    char  *buffer_address;
    int    *return_length;
    int    item_list_end;
  } itmlst;     /* Disabled auto initialization = { len - 1, 
		   LNM$_STRING, node, &length, 0 }; */
  char *p = node;

    /* Manual initialization code inserted by CRL on 931206 */
  itmlst.buffer_length   = (len - 1);
  itmlst.item_code       = LNM$_STRING;
  itmlst.buffer_address  = node;
  itmlst.return_length   = &length;
  itmlst.item_list_end   = 0;

  sys$trnlnm( 0, &tabnam, &lognam, 0, &itmlst );

  while( p[0] != '\\0' && p[0] != ':' )
    p++;
  p[0] = '\\0';

  return( 0 );
}
$!
$! Compile the main program(s) if found
$!
$  if f$search("PAWMAIN.FOR") .nes. ""
$     then
$     write sys$output "Compiling PAWMAIN..."
$     fortran pawmain
$  endif
$!
$  if f$search("PAWPP.FOR") .nes. ""
$     then
$     write sys$output "Compiling PAWPP..."
$     fortran pawpp
$  endif
$!
$  if f$search("GETHOSTNAME.C") .nes. ""
$     then
$     write sys$output "Compiling GETHOSTNAME..."
$     cc gethostname
$  endif
$!
$  if f$search("CZDUMMY.FOR") .nes. ""
$     then
$     write sys$output "Compiling CZDUMMY (dummy TCP/IP routines)..."
$     fortran czdummy
$  endif
$!
$! Linking of PAW/GKS version
$!
$! Licensed software required: GTS-GRAL GKS (installed and distributed by CERN)
$!
$   if architecture .eqs. "ALPHA"
$      then
$         write sys$output "PAW/GKS only available on VAX/VMS"
$      else
$         write sys$output "Link GTS-GRAL GKS version of PAW..."
$         cernlib pawlib,mathlib,packlib,graflib,packlib
$         link/exe=pawgks pawmain,czdummy,'LIB$
$      endif
$! 
$! Linking of PAW/DEC-GKS
$! Licensed software required: DEC-GKS from Digital
$!
$   write sys$output "Link DEC-GKS version of PAW..."
$   cernlib pawlib,mathlib,packlib,graflib/dgks,packlib
$   link/exe=pawdgks pawmain,czdummy,'LIB$
$!
$! Linking of PAW/X11
$! Licensed software required: Motif 1.1
$!
$   write sys$output "Link PAW/X11..."
$   cernlib pawlib,mathlib,packlib,graflib/x11,packlib
$   link/exe=pawx11 pawmain,gethostname,czdummy,'LIB$
$!
$! Linking of PAW/X11_M
$! Licensed software required: Motif 1.1, Multinet TCP/IP
$!
$   write sys$output "Link PAW/X11_M..."
$   cernlib pawlib,mathlib,packlib,graflib/x11,packlib
$   if architecture .eqs. "ALPHA"
$      then
$         link/exe=pawx11_m -
             pawmain,'LIB$',sys$input/opt
             multinet_socket_library/share
             sys$library:decc$shr/share
$      else
$         link/exe=pawx11_m -
             pawmain,'LIB$',sys$input/opt
             multinet_socket_library/share
             sys$library:vaxcrtl/share
$   endif
$!
$! Linking of PAW/X11_U
$! Licensed software required: Motif 1.1, DEC TCP/IP (UCX)
$!
$   write sys$output "Link PAW/X11_U..."
$   cernlib pawlib,mathlib,packlib,graflib/x11,packlib
$   if architecture .eqs. "ALPHA"
$      then
$         link/exe=pawx11_u -
             pawmain,'LIB$',sys$library:ucx$ipc/lib,sys$input/opt
             sys$library:ucx$ipc_shr/share
             sys$library:decc$shr/share
$      else
$         link/exe=pawx11_u -
             pawmain,'LIB$',sys$library:ucx$ipc/lib,sys$input/opt
             sys$library:ucx$ipc_shr/share
             sys$library:vaxcrtl/share
$   endif
$!
$! Linking of PAW/X11_DECW
$! Licensed software required: DECWindows
$!
$   write sys$output "Link PAW/X11_DECW..."
$   if architecture .eqs. "ALPHA"
$      then
$         write sys$output "This option only available on VAX/VMS"
$      else
$         lib$:==CERN:['cern_level'.LIB]PAWLIB/LIB,PACKLIB/LIB,MATHLIB/LIB,-
GRAFLIB/LIB,GRAFX11/LIB,PACKLIB/LIB,KERNLIB/LIB"
$         link/exe=pawx11_decw -
             pawmain,gethostname,czdummy,'LIB$',sys$input/opt
             cern:[decw]decw$xlibshr/share
             cern:[decw]decw$dwtlibshr/share
             cern:[decw]decw$transport_common/share
             sys$library:vaxcrtl/share
$   endif
$!
$! Linking of PAW/X11_DECW_M
$! Licensed software required: DECWindows, Multinet
$!
$   write sys$output "Link PAW/X11_DECW_M..."
$   if architecture .eqs. "ALPHA"
$      then
$         write sys$output "This option only available on VAX/VMS"
$      else
$         lib$:==CERN:['cern_level'.LIB]PAWLIB/LIB,PACKLIB/LIB,MATHLIB/LIB,-
GRAFLIB/LIB,GRAFX11/LIB,PACKLIB/LIB,KERNLIB/LIB"
$         link/exe=pawx11_decw_m -
             pawmain,'LIB$',sys$input/opt
             multinet_socket_library/share
             cern:[decw]decw$xlibshr/share
             cern:[decw]decw$dwtlibshr/share
             cern:[decw]decw$transport_common/share
             sys$library:vaxcrtl/share
$   endif
$!
$! Linking of PAW++ with Multinet
$!
$   write sys$output "Link PAW++ with Multinet..."
$   cernlib pawlib,mathlib,packlib,graflib/motif,packlib
$   if architecture .eqs. "ALPHA"
$      then
$         link/exe=pawpp -
             pawpp,'LIB$',sys$input/opt
             multinet_socket_library/share
             sys$library:decc$shr/share
$      else
$         link/exe=pawpp -
             pawpp,'LIB$',sys$input/opt
             multinet_socket_library/share
             sys$library:vaxcrtl/share
$   endif
$!
$! Linking of PAW++ with DEC TCP/IP (UCX)
$!
$   write sys$output "Link PAW++ with UCX..."
$   cernlib pawlib,mathlib,packlib,graflib/motif,packlib
$   if architecture .eqs. "ALPHA"
$      then
$         link/exe=pawpp_u -
             pawpp,'LIB$',sys$input/opt
             sys$library:ucx$ipc_shr/share
             sys$library:decc$shr/share
$      else
$         link/exe=pawpp_u -
             pawpp,'LIB$',sys$library:ucx$ipc/lib,sys$input/opt
             sys$library:ucx$ipc_shr/share
             sys$library:vaxcrtl/share
$   endif
$!
$!
$! Linking of PAW++ without Multinet
$!
$   write sys$output "Link PAW++ without Multinet..."
$   cernlib pawlib,mathlib,packlib,graflib/motif,packlib
$   if architecture .eqs. "ALPHA"
$      then
$         link/exe=pawpp_dnet -
             pawpp,czdummy,gethostname,'LIB$',sys$input/opt
             sys$library:decc$shr/share
$      else
$         link/exe=pawpp_dnet -
             pawpp,czdummy,gethostname,'LIB$',sys$input/opt
             sys$library:vaxcrtl/share
$   endif
\end{XMPt}

\section{Relinking PAW on VM/CMS systems}

The following section describes how to relink \PAW{}
on systems running VM/XA or VM/ESA. It assumes that the 
CERN libraries, e.g. PACKLIB, PAWLIB, are already installed on
your system.
 
\subsection{General requirements}
 
\begin{itemize}
\item
VS-FORTRAN 2.x    (our libraries are generated with 2.5.0)
\item
IBM C/370  2.1.0  (not compatible with Waterloo C)
\item
TCP/IP 2.1        (if you want to use the "rlogin" facility in PAW)
\item
GKS, GDDM or X11 graphics libraries
\item
C run time library
\end{itemize}


\subsection{Extracting the main program}

The main program can be extracted as shown below:

\begin{XMPt}{Extracting the \PAW{} main program}

/* Extract PAWMAIN */
/* PAWMAIN EXEC    */
queue "+USE,CZ."
queue "+USE,MAIN."
queue "+USE,P=PAW,D=0PAMAIN."
queue "+EXE."
queue "+KEEP,PAWSIZ."
queue "      PARAMETER (NWPAW=500000)"
queue "+PAM,11,T=C."
queue "+QUIT."
exec        ypatchy,
            'pam="PAW CAR *"',
            'asm="PAWMAIN FORTRAN"'

\end{XMPt}

This results in the following Fortran file:

\begin{XMPt}{\PAW{} main program}

*CMZ :  2.04/08 30/11/93  14.07.07  by  Rene Brun
*-- Author :    Rene Brun   03/01/89
      PROGRAM PAMAIN
*
*        MAIN Program for basic PAW
*
      PARAMETER (NWPAW=500000)
*
      COMMON/PAWC/PAWCOM(NWPAW)
*
      CALL PAW(NWPAW,IWTYP)
*
      CALL KUWHAG
*
      CALL PAEXIT
*
      STOP
      END
      SUBROUTINE QNEXT
      END

\end{XMPt}


The ENDMODU routine shown below is used in order to reduce the size of the PAW
module. 
 
\begin{XMPt}{ENDMODU FORTRAN}

      BLOCK DATA ENDMODU
      END

\end{XMPt}
 
\subsection{Building the GKS version of \PAW{}}

This requires the GTS-GRAL GKS software, which is installed and distributed 
by CERN.


\begin{XMPt}{Building the GKS version of \PAW{}}

 VFORT PAWMAIN
 VFORT ENDMODU 
 CERNLIB PAWLIB GRAFLIB ( GTS2D LINK
 LOAD PAWMAIN ( CLEAR NOAUTO
 INCLUDE ENDMODU
 GENMOD PAWGKS  ( FROM PAMAIN TO ENDMODU RMODE ANY AMODE ANY
 
\end{XMPt}

\section{Building the GDDM version of \PAW{}}

This requires the GDDM software from IBM.

 
\begin{XMPt}{Building the GDDM version of \PAW{}}

 VFORT PAWMAIN
 VFORT ENDMODU
 CERNLIB PAWLIB GRAFLIB ( GDDM LINK
 LOAD PAWMAIN ( CLEAR NOAUTO
 INCLUDE ENDMODU
 GENMOD PAWGDDM ( FROM PAMAIN TO ENDMODU RMODE ANY AMODE ANY
 
\end{XMPt}

\section{Building the X11 version of \PAW{}}

This requires IBM's X11 software, which is bundled together with TCP/IP.

\begin{XMPt}{Building the X11 version of \PAW{}}

 VFORT PAWMAIN
 VFORT ENDMODU
 CERNLIB PAWLIB GRAFLIB ( X11 LINK
 LOAD PAWMAIN ( CLEAR NOAUTO
 INCLUDE ENDMODU
 GENMOD PAWX11  ( FROM PAMAIN TO ENDMODU RMODE ANY AMODE ANY
 
\end{XMPt}

