%%%%%%%%%%%%%%%%%%%%%%%%%%%%%%%%%%%%%%%%%%%%%%%%%%%%%%%%%%%%%%%%%%%
%                                                                 %
%  GEANT manual in LaTeX form                                     %
%                                                                 %
%  Michel Goossens (for translation into LaTeX)                   %
%  Version 1.00                                                   %
%  Last Mod. Jan 24 1991  1300   MG + IB                          %
%                                                                 %
%%%%%%%%%%%%%%%%%%%%%%%%%%%%%%%%%%%%%%%%%%%%%%%%%%%%%%%%%%%%%%%%%%%
\Authors{F.Carminati, S.Giani}   \Origin{Same}
\Submitted{09.11.92}   \Revised{09.11.92}
\Version{Geant 3.15}\Routid{DRAW500}
\Makehead{The interactive commands}

Some of the graphics functionality of GEANT 3.16 is available only
from the interactive version. To run the interactive version the user
should link the application using as main program the \Rind{GXINT316}
code, which usually is in the same directory where the GEANT library
is. For more information on this subject the user is referred to the 
{\tt [XINT]} section.

We give here a short description of these commands which do not
correspond to a callable interface.

\section{{\tt LENS} command}
Once that a drawing (detectors, tracks or hits) is in a view bank, it is 
possible to scan it in great detail via the {\tt LENS} facility (with X11):
clicking on the left button of the mouse, it is possible to define a frame
(lens) on the drawing through which it will appear zoomed; clicking then on
the central button will translate the lens over the drawing, while clicking
the left one will change the magnification power of the lens; clicking once
on the right button is for changing action, clicking twice is to quit.

\section{{\tt EDITV} command}
This command which helps in the interactive debugging or tuning of the 
detector geometry is the interactive command {\tt EDITV}, {\tt [XINT130]},
by which it is possible to modify interactively some geometrical parameters
set by the user routines defining the detector geometry.

\section{{\tt MEASURE} command}
This command allows the user to measure the
distance between two points on the screen
and to translate this to the real distance in centimeters in the real 
coordinate system.

\section{{\it Hidden line removal} commands}
The following commands are only effective when the {\it hidden-line 
removal} mode is active ({\tt HIDE} option set to {\tt ON} via the
\Rind{GDOPT} routine or the {\tt DOPT} command).

\subsection{{\tt CVOL} menu}
The hidden-line removal technique is necessary to visualize properly
very complex detectors. At the same time, it can be useful to visualize
the inner elements of a detector in detail. For this purpose, the
commands menu {\tt CVOL} has been developed: these commands allow
subtractions (via boolean operation) of given shapes from any part of
the detector, thereby showing its inner contents. It is possible
to clip each different volume by means of a different shape ({\tt BOX,
TUBE, CONE, SPHE} are available). If '*' is given as the name of the
volume to be clipped, all volumes are clipped by the given shape.
A volume can be clipped at most twice (even by
different shapes); if a volume is explicitely clipped
twice, the '*' will not act on it anymore. Giving '.' as the name
of the volume to be clipped will reset the clipping.

\subsection{{\tt MOVE} command}
The interactive
{\tt MOVE} command with the option {\tt T} gives the possibility
to rotate, zoom and translate in real time the tracks stored in the
{\tt JXYZ} data structure and the hits for one event.
The option {\tt H} of the interactive
\Rind{MOVE} command gives the possibility
to rotate, zoom and translate in real time the hits stored for one event.
In this 3D display of the physical event it is
possible to obtain a printout of the content of a {\it hit} by picking
it with the mouse.

\subsection{{\tt BOMB} command}
It may be nice to obtain an {\it exploded} view of the detector. This can
be obtained via the {\tt BOMB} command. It has one {\it real} argument
{\it boom}. If {\it boom} is positive (values $< 1$ are suggested, but any 
value is possible) all the volumes are shifted by a distance proportional 
to {\it boom} along the direction between their centre and the origin of 
the {\tt MARS}; the volumes which are symmetric with respect to this 
origin are simply not shown. {\it boom} equal to 0 resets the normal mode.
A negative ($ > -1$) value of {\it boom} will cause an {\it implosion}; 
for even lower values of {\it boom} the volumes' positions will be 
reflected respect to the origin. This command can be useful to improve 
the 3D effect for very complex detectors.

\subsection{{\tt SHIFT} command}
It is possible to draw a volume shifted from its initial position.
It can be useful if the user wants to extract a
volume or some volumes from the detector to show them more clearly.
The last requested {\tt SHIFT} for each volume
is performed. Moreover, the {\tt SHIFT} of
each volume will be performed starting from where its mother has
been shifted, so that it's easier to {\tt SHIFT} nicely sets
of volumes using the mother-daughter relationships.
If '.' is given as the name of the volume
to be shifted, the shifts for all volumes will be reset.
