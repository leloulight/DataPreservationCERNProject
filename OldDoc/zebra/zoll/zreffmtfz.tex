% file : zebrafmt.tex
%
%  Set the page size  A4  =  210mm by 297 mm
%
%  width of text
%
%
%  vertical space
%
%  \setlength{\voffset}{-18mm}
   \setlength{\topmargin}{-15mm}
   \setlength{\headheight}{5mm}
   \setlength{\headsep}{10mm}
   \setlength{\textheight}{245mm}
   \setlength{\baselineskip}{13pt}
   \setlength{\footskip}{15mm}
   \setlength{\footheight}{5mm}
   \parskip 6pt plus 1pt
   \parindent 0pt
%
%  horizontal space
%
%  \setlength{\hoffset}{-13mm}
   \setlength{\textwidth}{160mm}
   \setlength{\oddsidemargin}{-5mm}
   \setlength{\evensidemargin}{3mm}
   \setlength{\marginparsep}{5mm}
   \setlength{\marginparpush}{20mm}
   \setlength{\marginparwidth}{15mm}
%
%  macro "in" - move left margin
%
   \def\in#1{\advance\leftskip by #1}
%
%  macro "lile" - line-length, move left and right margins
%
   \def\lile#1{\advance\leftskip by #1 \advance\rightskip by #1}
%
%  macro "spcomp" - inter-line spacing compact
%
   \def\spcomp{\setlength{\baselineskip}{12pt}}
%
%  macro "spnorm" - inter-line spacing normal
%
   \def\spnorm{\setlength{\baselineskip}{13pt}}
%
%  Command \Func - box for Fortran FUNCTION - obsolete, use ROUTA
%
   \newcommand{\Func}[1]{\par\vspace*{2mm}
    \framebox[\textwidth]{\rule[-3mm]{0mm}{8mm}\large\tt#1}
    \par\vspace*{1mm}}
%
%  Command \Subr - box for Fortran SUBROUTINE - obsolete, use ROUTA
%
   \newcommand{\Subr}[1]{\par\vspace*{2mm}
    \framebox[\textwidth]{\rule[-3mm]{0mm}{8mm}\large\tt#1}
    \par\vspace*{1mm}}
%
%  Command \ROUTA - box for high-lighting one routine
%
   \newcommand{\ROUTA}[1]{\par\vspace*{2mm}
    \framebox[\textwidth]{\rule[-3mm]{0mm}{8mm}\large\tt#1}
    \par\vspace*{1mm}}
%
%  Commands \ROUTB S M L - boxes for high-lighting two routines
%
   \newcommand{\ROUTBS}[2]{\ROUTBX{40}{#1}{#2}}
   \newcommand{\ROUTBM}[2]{\ROUTBX{20}{#1}{#2}}
   \newcommand{\ROUTBL}[2]{\ROUTBX{10}{#1}{#2}}
%
%        Command \ROUTBX - box for high-lighting two routines
%
   \newcommand{\ROUTBX}[3]{\par%
     \setlength{\unitlength}{1mm}%
     \begin{picture}(160,19)%
       \put(0,0){\framebox(160,16){ }}%
       \put(#1,9){\large\tt #2}%
       \put(#1,3){\large\tt #3}%
     \end{picture}%
     \par\vspace*{1mm}}
%
%  Commands \ROUTC S M L - boxes for high-lighting three routines
%
   \newcommand{\ROUTCS}[3]{\ROUTCX{40}{#1}{#2}{#3}}
   \newcommand{\ROUTCM}[3]{\ROUTCX{20}{#1}{#2}{#3}}
   \newcommand{\ROUTCL}[3]{\ROUTCX{10}{#1}{#2}{#3}}
%
%        Command \ROUTCX - box for high-lighting three routines
%
   \newcommand{\ROUTCX}[4]{\par%
     \setlength{\unitlength}{1mm}%
     \begin{picture}(160,25)%
       \put(0,0){\framebox(160,22){}}%
       \put(#1,15){\large\tt #2}%
       \put(#1,9){\large\tt #3}%
       \put(#1,3){\large\tt #4}%
     \end{picture}%
     \par\vspace*{1mm}}
%%
%%%%    -- keep for safety --
%% %
%% %  Command \Func - box for Fortran FUNCTION
%% %
%%    \newcommand{\Func}[1]{\par\vspace*{3mm}\framebox[.98\textwidth]
%%    {\rule[-3mm]{0mm}{8mm}\large\tt#1}\par\vspace*{1mm}%
%%    \def\cut##1= ##2 ##3 {\gdef\subnam{##2}}
%%    \setbox77\hbox{\cut#1 A }%Placeholder for case without argument
%%    \protect\label{SR_\subnam}% mark the Function definition
%%    \protect\index{Function \subnam}% Enter Function definition in index
%%    }% ***** end of \newcommand{\Func}
%% %
%% %  Command \Subr - box for Fortran SUBROUTINE
%% %
%%    \newcommand{\Subr}[1]{\par\vspace*{3mm}\framebox[.98\textwidth]
%%    {\rule[-3mm]{0mm}{8mm}\large\tt#1}\par\vspace*{1mm}%
%%    \def\cut##1 ##2 ##3 {\gdef\subnam{##2}}
%%    \setbox77\hbox{\cut#1 A }%Placeholder for case without argument
%%    \protect\label{SR_\subnam}% mark the subroutine definition
%%    \protect\index{Routine \subnam}% Enter subroutine def in index
%%    }% ***** end of \newcommand{\Subr}
%% %%
