%%%%%%%%%%%%%%%%%%%%%%%%%%%%%%%%%%%%%%%%%%%%%%%%%%%%%%%%%%%%%%%%%%%%%%%%%%%%%%%%
%                                                                              %
%   HIGZ  User Guide -- LaTeX Source                                           %
%                                                                              %
%   Chapter: The miscellaneous functions                                       %
%                                                                              %
%   Editor: Olivier Couet / CN-AS                                              %
%   Last Mod.: 9 July 1993 oc                                                  %
%                                                                              %
%%%%%%%%%%%%%%%%%%%%%%%%%%%%%%%%%%%%%%%%%%%%%%%%%%%%%%%%%%%%%%%%%%%%%%%%%%%%%%%%
\Filename{H1miscellaneousfunctions}
\chapter{miscellaneous functions}
\index{miscellaneous functions}

User routines, whose functionality is often needed (e.g. displaying a message),
but which cannot be classified easily in any of the previous chapters will be
described in this chapter.

\Filename{H2Displayamessageonthescreen}
\section{Display a message on the screen}
\index{message on the screen}
\Shubr{IGMESS}{(N,CHMESS,CHTIT,CHOPT)}
\Action
This routine allows to display a message. The \X11{} version of \HIGZ{} displays
the message in a separated window.
\Pdesc
\begin{DLtt}{1234567}
\item[N] Number of lines in the message.
\item[CHMESS(N)] Message to be displayed.
\item[CHTIT] Window title.
\item[CHOPT] Options.
\begin{DLtt}{12345}
\item['P'] Print the array \Rarg{CHMESS} and open the message window
if necessary.
\item['C'] Close the message window.
\item['T'] Print the array \Rarg{CHMESS} on standard output.
\item['D'] Delete the message window.
\end{DLtt}
\end{DLtt}
\Filename{H2Displayacolourmap}
\section{Display a colour map}
\index{display!colour map}
\Shubr{IGCOLM}{(X1,X2,Y1,Y2,IC1,IC2,ZMIN,ZMAX,CHOPT)}
\Action
This routine allows to display a colour map on the screen.
\Pdesc
\begin{DLtt}{1234567}
\item[X1] X coordinate of 1st corner of the rectangle in \WC.
\item[X2] X coordinate of 2nd corner of the rectangle in \WC.
\item[Y1] Y coordinate of 1st corner of the rectangle in \WC.
\item[Y2] Y coordinate of 2nd corner of the rectangle in \WC.
\item[IC1] First colour index.
\item[IC2] Last colour index
\item[ZMIN] Minimum Z value.
\item[ZMAX] Maximum Z value.
\item[CHOPT] Options.
\begin{DLtt}{12345}
\item['C'] Draw the levels with \Em{C}olours.
\item['B'] Draw the levels with \Em{B}oxes.
\item['A'] Draw the \Em{A}xis.
\item['H'] Draw the map \Em{H}orizontally (default is vertically).
\end{DLtt}
\end{DLtt}

\newpage

\Filename{H2ConversionbetweenColoursystems}
\section{Conversion between Colour systems}
\index{colour!systems!HLS}
\index{colour!systems!RGB}

\subsection{RGB to HLS}
\Shubr{IGRTOH}{(CR,CB,CG,CH*,CL*,CS*)}
\Action
This routine convert a RGB colour into an HLS colour.
\Pdesc
\begin{DLtt}{1234567}
\item[CR] Red value \Lit{0.\(\leq\)CR\(\leq\)1.}
\item[CG] Green value \Lit{0.\(\leq\)CG\(\leq\)1.}
\item[CB] Blue value \Lit{0.\(\leq\)CB\(\leq\)1.}
\item[CH] Hue value \Lit{0.\(\leq\)CH\(\leq\)360.}
\item[CL] Light value \Lit{0.\(\leq\)CL\(\leq\)1.}
\item[CS] Saturation value \Lit{0.\(\leq\)CS\(\leq\)1.}
\end{DLtt}

\subsection{HLS to RGB}
\Shubr{IGHTOR}{(CH,CL,CS,CR*,CB*,CG*)}
\Action
This routine convert a HLS colour into an RGB colour.
\Pdesc
\begin{DLtt}{1234567}
\item[CH] Hue value \Lit{0.\(\leq\)CH\(\leq\)360.}
\item[CL] Light value \Lit{0.\(\leq\)CL\(\leq\)1.}
\item[CS] Saturation value \Lit{0.\(\leq\)CS\(\leq\)1.}
\item[CR] Red value \Lit{0.\(\leq\)CR\(\leq\)1.}
\item[CG] Green value \Lit{0.\(\leq\)CG\(\leq\)1.}
\item[CB] Blue value \Lit{0.\(\leq\)CB\(\leq\)1.}
\end{DLtt}

\newpage
 
\Filename{H2Conversionbetweencharacterstringandnumbers}
\section{Conversion between character string and numbers}
\index{character!conversion to number}
\index{number!conversion to character}

Often it is necessary to convert a \FORTRAN{} character string into
a number (integer or real) or vice versa. For example, routine \Rind{IGMENU}
returns some parameters as character strings and it is often necessary to convert
these into numbers. 
Also, to print graphically the result of a computation with
\Rind{ITX} it is necessary to convert a number into a character string. 
The routines described in this paragraph allow these kinds of conversions.

\subsection{Character to integer}
\Shubr{IZCTOI}{(CHVAL,IVAL*)}
\Action
Converts the character string {\tt CHVAL} into the integer {\tt IVAL}.
\Pdesc
\begin{DLtt}{1234567}
\item[CHVAL] Character string.
\item[IVAL] Integer.
\end{DLtt}

\subsection{Character to real}
\Shubr{IZCTOR}{(CHVAL,RVAL*)}
\Action
Converts the character string {\tt CHVAL} into the real {\tt RVAL}.
\Pdesc
\begin{DLtt}{1234567}
\item[CHVAL] Character string.
\item[RVAL] Real.
\end{DLtt}
\subsection{Integer to character}
\Shubr{IZITOC}{(IVAL,CHVAL*)}
\Action
Converts the integer {\tt IVAL} into character string {\tt CHVAL}.
\Pdesc
\begin{DLtt}{1234567}
\item[IVAL] Integer.
\item[CHVAL] Character string.
\end{DLtt}
\subsection{Real to character}
\Shubr{IZRTOC}{(RVAL,CHVAL*)}
\Action
Converts the real {\tt RVAL} into character string {\tt CHVAL}.
\Pdesc
\begin{DLtt}{1234567}
\item[RVAL] Real.
\item[CHVAL] Character string.
\end{DLtt}
