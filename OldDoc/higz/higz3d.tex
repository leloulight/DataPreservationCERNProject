%%%%%%%%%%%%%%%%%%%%%%%%%%%%%%%%%%%%%%%%%%%%%%%%%%%%%%%%%%%%%%%%%%%%%%%%%%%%%%%%
%                                                                              %
%   HIGZ  User Guide -- LaTeX Source                                           %
%                                                                              %
%   Chapter: 3-dimensional routines                                            %
%                                                                              %
%   This document needs the following external EPS files:                      %
%     none                                                                     %
%                                                                              %
%   Editor: Michel Goossens / AS-MI                                            %
%   Last Mod.: 16 January 1992 14:25 mg                                        %
%                                                                              %
%%%%%%%%%%%%%%%%%%%%%%%%%%%%%%%%%%%%%%%%%%%%%%%%%%%%%%%%%%%%%%%%%%%%%%%%%%%%%%%%
\Filename{H1Aproposalforthe3Droutines}
\chapter{A proposal for the 3D routines}
The current implementation of \HIGZ, as described in the previous
chapters, is only 2D.
3D applications must cohabitate with 2D applications based
on \HIGZ. For example an event scanning program should be able to
display histograms as well.
\par
We see \HIGZ{} 3D as being essentially a direct interface
to the local graphics package (e.g. \GKS3D, WAND,
\index{GMR}
\index{GSR}
\index{DI3000}
\index{Apollo}
GMR-3D, GSR-3D or DI3000).
In the 3D case, graphics structures will be manipulated
directly by the device software for obvious reasons of
efficiency. The IZ part of \HIGZ{} could nevertheless be useful to save
a picture into the data base and make usage of the
\index{picture data base}
Graphics Editor.
\par
A version of \HIGZ{} 3D interfaced to the GMR package on Apollo
and the WAND package for Megatek has been written by the L3 group
from NIKHEF (contact person is G.Massaro). These routines
are not yet included in the \HIGZ{} standard file. The list of routines
currently implemented is shown below. For more details, contact
the authors.
\begin{center}\begin{tabular}{>{\ttsc}ll}
  ISVIEW(NT, THETA, PHI, PSI)             &Set view representation\\
  ISVP3(NT,XMIN,XMAX,YMIN,YMAX,ZMIN,ZMAX) &Set viewport\\
  ISWN3(NT,XMIN,XMAX,YMIN,YMAX,ZMIN,ZMAX) &Set window\\
  ISOPEN(ISNUM, CLEAR)                    &Open a structure\\
  ISCLOS                                  &Close a structure\\
  IML3(N, X, Y, Z)                        &Multiline\\
  IGML3(N, XYZ)                           &Multiline\\
  IPL3(N, X, Y, Z)                        &Polyline\\
  IGPL3(N, XYZ)                           &Polyline\\
  IPM3(N, X, Y, Z)                        &Polymarker\\
  IGPM3(N, XYZ)                           &Polymarker\\
  ITX3(X, Y, Z, TEXT)                     &Text\\
\end{tabular}\end{center}
