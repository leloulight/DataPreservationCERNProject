%%%%%%%%%%%%%%%%%%%%%%%%%%%%%%%%%%%%%%%%%%%%%%%%%%%%%%%%%%%%%%%%%%%
%                                                                 %
%   MINUIT User Guide -- LaTeX Source                             %
%                                                                 %
%   Chapter 1                                                     %
%                                                                 %
%   The following external EPS files are referenced:              %
%                                                                 %
%   Editor: Michel Goossens / CN-AS                               %
%   Last Mod.: 17 Mar. 1992 10:00 mg                              %
%                                                                 %
%%%%%%%%%%%%%%%%%%%%%%%%%%%%%%%%%%%%%%%%%%%%%%%%%%%%%%%%%%%%%%%%%%%
 
\chapter{Minuit Basic Concepts}

\section{The Organization of Minuit.}
The Minuit package acts on a multiparameter Fortran function to which we
give the generic name \Rind{FCN}, although the actual name may be chosen by the user.
This function must be defined and supplied by the user (or by an intermediate
program such as HBOOK\cite{bib-HBOOK} or PAW\cite{bib-PAW}, 
in case Minuit is being used under the
control of such an intermediate program).
The value of \Rind{FCN} will in general depend on one or more variable parameters
whose meaning is defined by the user (or by the intermediate program),
but whose trial values are determined by Minuit according to what the user
requests should be done to \Rind{FCN} (usually minimize it).
 
To take a simple example, suppose the problem is to fit a polynomial through
a set of data points.
Then the user would write an \Rind{FCN} which calculates the chisquare between a
polynomial and the data; the variable parameters of \Rind{FCN} would be the
coefficients of the polynomials.  Using Minuit commands, the user would request
Minuit to minimize \Rind{FCN} with respect to the parameters, that is, find those
values of the coefficients which give the lowest value of chisquare.
 
The user must therefore supply, in addition to the function to be analyzed,
a set of commands to instruct Minuit what analysis is wanted.
The commands may be given in several different forms:

\begin{UL}
\item As a data file, corresponding to the traditional ``data cards'', for
      batch processing;
\item Typed in at execution time at a terminal, for interactive running;
\item Coded in Fortran in the calling program, which allows looping,
      conditional execution, and all the other possibilities of Fortran, but not
      interactivity, since it must be compiled before execution.
      This is sometimes known as running Minuit in ``slave mode''.
      HBOOK and PAW use Minuit in this way.
\end{UL}

It is also possible to mix any of the above forms, for example starting off
a fit with a standard command file, then turning it over to the interactive
user for the final command steps.

\section{Internal and External Parameters.}

\index{internal parameter}
\index{external parameter}
\index{parameters!external}
\index{parameters!internal}
Each of the parameters to \Rind{FCN} is defined by the user as belonging to
one of the following types:

\begin{DL}{Variable with limits:\ }
\item[Freely variable:]      allowed to take on any value.
\item[Variable with limits:] allowed to vary only between two limits specified by the user.
\item[Fixed:]                originally defined as variable, but now taking on only the
                             value the parameter had at the moment it was fixed, 
                             or a value later assigned by the user.
\item[Constant:]             taking on only one value as specified by the user.
\item[Undefined:]            never defined by user.
\end{DL}
 
The user, in \Rind{FCN}, must of course be able to ``see'' all types of
defined parameters,
and he therefore has access to what we call the
{\it external parameter list}, that is, the parameters as he
defined them.
On the other hand, the internal Minuit minimizing routines only want to ``see''
variable parameters without limits, and so they have access only to the
{\it internal parameter list} which is created from the external list
by the following transformation:

\begin{OL}
\item Squeeze out all parameters that are not variable.
\item Transform all variable parameters with limits, so that the transformed
      parameter can vary without limits.
      (See the next section for details concerning this transformation.)
      Because this transformation is non-linear, it is recommended to avoid
      putting limits on parameters where they are not needed.
\end{OL}

As an example, suppose that the user has defined the following parameters:
\begin{UL}
\item Parameter 1, constant.
\item Parameter 3, freely variable.
\item Parameter 10, variable with limits.
\item Parameter 11, constant.
\item Parameter 22, freely variable.
\item All others undefined.
\end{UL}
Then the internal parameter list would be as follows:
\begin{UL}
\item Internal parameter 1 = external parameter 3.
\item Internal parameter 2 = external parameter 10, transformed appropriately.
\item Internal parameter 3 = external parameter 22.
\end{UL}
\index{internal parameter}
\index{external parameter}
\index{parameters!external}
\index{parameters!internal}
In the above example, Minuit considers that the number of external parameters
is 22 (the highest external parameter number defined), and the number of
internal parameters is 3.  The latter number is passed as \texttt{NPAR} to \Rind{FCN}.
This is the number which determines, for example, the size of the error matrix
of the parameters, since only variable parameters have errors.
 
An important feature of Minuit is that parameters are allowed to change
types during a Minuit run. Several Minuit commands are available to make
variable parameters fixed and vice-versa; to impose, change, or remove limits
from variable parameters; and even to define completely new parameters at any
time during a run. In addition, some Minuit routines (notably the \Cind[MINOs]{MINOS} error
analysis) cause one or more variable parameters to be temporarily fixed during
the calculation.  Therefore, the correspondence between external and internal
parameter lists is in general a dynamic one, and the value of \texttt{NPAR} is not
necessarily constant.

\subsection{The transformation for parameters with limits.}

For variable parameters with limits, Minuit uses the following transformation:
$$
\begin{array}{l@{\hspace{3cm}}l}
P_{\mathrm{int}} = \arcsin
        \left( 2\: \frac{\Tstm P_{\mathrm{ext}}-a\Rule}{\Tstm b-a} - 1 \right)       &
P_{\mathrm{ext}} = a + \frac{\Tstm b - a}{\Tstm 2}
        \left( \sin P_{\mathrm{int}} + 1 \right)                  \\
\end{array}
$$
so that the internal value $P_{\mathrm{int}}$ can take on any value, while
the external value $P_{\mathrm{ext}}$ can take on values only between the lower
limit $a$ and the upper limit $b$.
Since the transformation is necessarily non-linear, it would transform a
nice linear problem into a nasty non-linear one, which is the reason why
limits should be avoided if not necessary. 
In addition, the transformation
does require some computer time, so it slows down the computation a little
bit, and more importantly, it introduces additional numerical inaccuracy into
the problem in addition to what is introduced in the numerical calculation
of the \texttt{FCN} value.  
The effects of non-linearity and numerical roundoff both
become more important as the external value gets closer to one of the limits
(expressed as the distance to nearest limit divided by distance between limits).
The user must therefore be aware of the fact that, for example,
if he puts limits of $(0,10^{10})$ on a parameter, then the values $0.0$ 
and $1.0$ will be indistinguishable to the accuracy of most machines.

The transformation also affects the parameter error matrix, of course,
so Minuit does a transformation of the error matrix (and the 
``parabolic'' parameter errors) when there are parameter limits.
Users should however realize that the transformation is only a linear
approximation, and that it cannot give a meaningful result if one or more
parameters is very close to a limit, where
$\partial P_{\mathrm{ext}} / \partial P_{\mathrm{int}} \approx 0$.
Therefore, it is recommended that:
\begin{UL}
\item Limits on variable parameters should be used only when needed in order
to prevent the parameter from taking on unphysical values.
\item When a satisfactory minimum has been found using limits, the limits
should then be removed if possible, in order to perform or re-perform the
error analysis without limits.
\end{UL}

Further discussion of the effects of parameter limits may be
found in the last chapter.

\section{Minuit Strategy.}

At many places in the analysis of the user function, Minuit must
decide whether to be ``safe'' and waste a few function calls in order
to know where it is, or to be ``fast'' and attempt to get the requested
results with the fewest possible calls at a certain risk of not
obtaining the precision desired by the user.
In order to allow the user to influence these decisions, there is an
internal Minuit parameter \texttt{ISTRAT} which can be set by the user
through the command \Cind{SET STRategy}.  
In the current release,
this parameter can take on three integer values (0, 1, 2), and the
default value is 1.  Value 0 indicates to Minuit that it should
economize function calls; it is intended for cases where there are
many variable parameters and/or the function takes a long time
to calculate and/or the user is not interested in very precise
values for parameter errors.  On the other hand, the value 2 indicates
that Minuit is allowed to waste function calls in order to be sure
that all values are precise; it is intended for cases where the function
is evaluated in a very short time and/or where the parameter errors
must be calculated reliably

\section{Parameter Errors.}

Minuit is usually used to find the ``best'' values of a set of parameters,
where ``best'' is defined as those values which minimize a given function, \Rind{FCN}.
The width of the function minimum, or more generally, the shape of the
function in some neighbourhood of the minimum, gives information about
the {\it uncertainty} in the best parameter values, often called by
physicists the {\it parameter errors}.
An important feature of Minuit is that it offers several tools to analyze
the parameter errors.
\subsection{FCN Normalization and the ERRor definition.}
Whatever method is used to calculate the parameter errors, they will depend
on the overall (multiplicative) normalization of \Rind{FCN}, in the sense that if
the value of \Rind{FCN} is everywhere multiplied by a constant $\beta$, then the errors
will be decreased by a factor $\sqrt{\beta}$.
Additive constants do not change the parameter
errors, but may imply a different goodness-of-fit confidence level.
 
Assuming that the user knows what the normalization of his \Rind{FCN} means, and
also that he is interested in parameter errors, the \Cind{SET ERRordef} command
allows him to define what he means by one ``error'', in terms of the
change in \Rind{FCN} value which should be caused by changing one parameter by
one ``error''.  
If the \Rind{FCN} is the usual chisquare function (defined below),
then \Cind[SET ERRordef]{ERRordef} should be set to 1.0 
(the default value anyway) if the user
wants the usual one-standard-deviation errors.
If \Rind{FCN} is a negative-log-likelihood
function, then the one-standard-deviation value for 
\Cind[SET ERRordef]{ERRORDEF} is 0.5.
If \Rind{FCN} is a chisquare, but the user wants two-standard-deviation errors, then
\Cind[SET ERRordef]{ERRORDEF} should be = 4.0, etc.
 
Note that in the usual case where Minuit is being used to perform a fit to
some experimental data, the parameter errors will be proportional to the
uncertainty in the data, and therefore meaningful parameter errors cannot
be obtained unless the measurement errors of the data are known.  In the common
case of a least-squares fit, \Rind{FCN} is usually defined as a chisquare:

\begin{equation}
\chi^2 (\alpha) = \sum_{i=1}^{n} \frac{f(x_i,\alpha) - e_i)^2}{\sigma_i^2}
\end{equation}

where $\alpha$ is the vector of free parameters being fitted, and
the $\sigma_i$ are the uncertainties in the individual measurements $e_i$.
If these uncertainties are not known, and are simply left out of the calculation,
then the fit may still have meaning, but not the quantitative values of the
resulting parameter errors.
(Only the relative errors of different parameters with
respect to each other may be meaningful.)
 
If the $\sigma_i$ are all overestimated by a factor $\beta$, then the resulting
parameter errors from the fit will be overestimated by the same factor $\beta$.

\subsection{The Error Matrix.}

The Minuit processors \Cind[MIGrad]{MIGRAD} and \Cind[HESse]{HESSE} 
normally produce an error matrix.
This matrix is the inverse of the matrix of second derivatives of \Rind{FCN},
transformed if necessary into external coordinate space%
\footnote{The {\it internal error matrix} maintained by Minuit is transformed
for the user into {\it external coordinates}, but the numbering
of rows and columns is of course still according to internal parameter
numbering, since one does not want rows and columns corresponding to
parameters which are not variable. The transformation therefore
affects only parameters with limits; if there are no limits, internal and
external error matrices are the same.},
and multiplied by the square root of \Cind[SET ERRordef]{ERRORDEF}.
Therefore, errors based on the Minuit error matrix take account of all the
parameter correlations, but not the non-linearities. That is, from the error
matrix alone, two-standard-deviation errors are always exactly twice as big
as one-standard-deviation errors.
 
When the error matrix has been calculated (for example by the successful
execution of a command \Cind{MIGrad} or \Cind{HESse}) then the parameter errors
printed by Minuit are the square roots of the diagonal elements of this
matrix. The commands \Cind{SHOw COVariance} and \Cind{SHOw CORrelations}
allow the user to see the off-diagonal elements as well.
The command \Cind{SHOw EIGenvalues} causes Minuit to calculate and print
out the eigenvalues of the error matrix, which should all be
positive if the matrix is positive-definite (see below on Migrad and
positive-definiteness).
 
The effect of correlations on the individual parameter errors can be
seen as follows. When parameter \texttt{N} is \Cind{FIX}ed, 
Minuit inverts the error
matrix, removes the row and column corresponding to parameter \texttt{N}, and
re-inverts the result. The effect on the errors of the other parameters
will in general be to make them smaller, since the component due to
the uncertainty in parameter \texttt{N} has now been removed. (In the limit
that a given parameter is uncorrelated with parameter \texttt{N}, its error will
not change when parameter \texttt{N} is fixed.)
However the procedure is not reversible, since Minuit forgets the
original error matrix, so if parameter \texttt{N} is then \Cind{RELease}d, 
the error matrix is considered as unknown and has to be recalculated with
appropriate commands.

\subsection{MINOS Errors.}
\index{errors}

The Minuit processor \Cind[MINOs]{MINOS} was probably the first, and may still be the only,
generally available program to calculate parameter errors taking into
account both parameter correlations and non-linearities.
The \Cind[MINOs]{MINOS} error intervals are in general assymmetric, and may be expensive
to calculate, especially if there are a lot of free parameters and the
problem is very non-linear.
 
\Cind[MINOs]{MINOS} can only operate after a good minimum has already been found, and
the error matrix has been calculated, so the \Cind[MINOs]{MINOS} command will
normally follow a \Cind[MIGrad]{MIGRAD} command.
The \Cind[MINOs]{MINOS} error for a given parameter is defined as the change in the
value of that parameter which causes $F'$ to increase by the amount \texttt{UP},
where $F'$ is the minimum of \Rind{FCN} with respect to all {\it other}
free parameters, and \texttt{UP} is the ERRordef value specified by the
user (default = 1.).
 
The algorithm for finding the positive and negative \Cind[MINOs]{MINOS} errors for
parameter \texttt{N} consists of varying parameter \texttt{N}, each time minimizing
\Rind{FCN} with respect to all the other \texttt{NPAR-1} variable parameters,
to find numerically the two values of parameter \texttt{N} for which the
minimum of \Rind{FCN} takes on the values \texttt{FMIN+UP}, where \texttt{FMIN} is the
minimum of \Rind{FCN} with respect to all \texttt{NPAR} parameters.
In order to make the procedure as fast as possible, \Cind[MINOs]{MINOS} uses the
error matrix to predict the values of all parameters at the
various sub-minima which it will have to find in the course of the
calculation, and in the limit that the problem is nearly linear,
the predictions of \Cind[MINOs]{MINOS} will be nearly exact, requiring very
few iterations.  On the other hand, when the problem is very
non-linear (i.e., \Rind{FCN} is far from a quadratic function of its
parameters), that is precisely the situation when \Cind[MINOs]{MINOS} is needed
in order to indicate the correct parameter errors.

\subsection{Contour Plotting}

Minuit currently offers two very different procedures for finding
\Rind{FCN} contours.  They will be identified by the corresponding
command names: \Cind{CONtour} and \Cind{MNContour}.
 
\subsubsection{CONtour}

This procedure is designed for a lineprinter or alphanumeric
terminal as output device, and gives a static picture of \Rind{FCN}
as function of the two parameters specified by the user, that is,
all the other variable parameters (if any) are considered as
temporarily fixed at their current values.  First a range is
chosen, by default two current standard deviations on either
side of the current best value of each of the two parameters,
and a grid size n is chosen, by default 25 by 25 positions
for the full range of each parameter.  Contour zero is defined
as the current best function value $\displaystyle F_{\mathrm{min}}$
(presumably the minimum), and then the $\displaystyle i^{\mathrm{th}}$
contour is defined as where \Rind{FCN} has the value
$\displaystyle F_{\mathrm{min}} + i^2 * \mbox{\tt UP}$.
The procedure then simply evaluates \Rind{FCN} at the four corners of
each of the $n^2$ grid positions (which makes $(n+1)^2$ evaluations)
to determine whether the $\displaystyle i^{\mathrm{th}}$ contour passes through
it. The method, although not very efficient or precise,
is very robust, and capable of revealing unexpected
multiple valleys.
 
\subsubsection{MNContour}

The contour calculated by \Cind{MNContour} is dynamic, in the sense
that it represents the minimum of \Rind{FCN} with respect to all the
other \texttt{NPAR-2} parameters (if any). In statistical terms, this
means that \Cind{MNContour} takes account of the correlations between
the two parameters being plotted, and all the other variable
parameters, using a procedure analogous to that of \Cind[MINOs]{MINOS}.
(If this feature is not wanted, then the other
parameters must be \Cind{FIX}ed before calling \Cind{MNContour}.)
\Cind{MNContour} provides the actual coordinates of the points around
the contour, suitable for plotting with a graphics routine or
by hand. The points are given in counter-clockwise order
around the contour.
Only one contour is calculated per command (or Fortran call),
and the level is $\displaystyle F_{\mathrm{min}} + \mbox{\tt UP}$.
where \texttt{UP} is the
\Cind[SET ERRordef]{ERRordef} specified by the user, or 1.0 by default.
The number of points to be calculated is chosen by
the user (Default is 20 for the data-driven mode.).
As a by-product, \Cind{MNContour} provides the \Cind[MINOs]{MINOS} errors of the
two parameters in question, since these are just the extreme
points of the contour (Use \Cind{SHOw MINos} to see them).
In command-driven mode, a rough (alphanumeric, not graphic)
plot of the points is given 
(if \Cind[SET PRIntout]{PRIntlevel$\geq0$}) 
and the numerical values of the coordinates are printed 
(if \Cind[SET PRIntout]{PRIntlevel$\geq1$}).
In Fortran-callable mode, the user gets Fortran
access to the vector of point coordinates through
\Rind[MNCONT]{SUBROUTINE MNCONT}.
